
\documentclass[b5paper, openany]{ctexbook}

\usepackage[colorlinks=true, linkcolor=black]{hyperref}
\usepackage[margin=2.5cm]{geometry}



% \usepackage{physics}
\usepackage{pifont}
\usepackage[perpage,symbol*]{footmisc}
\DefineFNsymbols{circled}{{\ding{192}}{\ding{193}}{\ding{194}}
{\ding{195}}{\ding{196}}{\ding{197}}{\ding{198}}{\ding{199}}{\ding{200}}{\ding{201}}}
\setfnsymbol{circled}



\usepackage{amsmath,amsfonts,mathrsfs,amssymb}
\usepackage{graphicx}

\usepackage[font=bf,labelfont=bf,labelsep=quad]{caption}

\usepackage{tikz, pstricks}
%\usepackage{nicematrix}
\usepackage{bookmark}
\usepackage{ntheorem}
\theoremseparator{\;}



\usepackage{blkarray}
\usepackage{bm}


\usepackage{enumerate}


\theoremstyle{plain}
\theoremheaderfont{\normalfont\bfseries} 
\theorembodyfont{\normalfont}


\usepackage[framemethod=tikz]{mdframed}

\usepackage{circuitikz}

\newtheorem{example}{\bf 例}[chapter]
\newenvironment{solution}{\noindent {\bf 解:}}{}  %{\hfill $\clubsuit$\par}
\newenvironment{analyze}{\noindent {\bf 分析:}}{}
\newenvironment{rmk}{\noindent {\bf 注意:}}{}
\newenvironment{note}{\noindent {\bf 说明:}}{}



\renewcommand{\proofname}{\bf 证明:}
\newenvironment{proof}{{\noindent \bf 证明:}}{}%{\hfill $\square$\par}

\newcommand{\E}{\mathbb{E}}
\renewcommand{\Pr}{\mathbb{P}}
\newcommand{\EP}{\mathbb{E}^{\mathbb{P}}}
\newcommand{\EQ}{\mathbb{E}^{\mathbb{Q}}}
\newcommand{\dif}{\,{\rm d}}
\newcommand{\Var}{{\rm Var}}
\newcommand{\Cov}{{\rm Cov}}
\newcommand{\x}{\times}

 \usepackage{tcolorbox}
 \tcbuselibrary{breakable}
 \tcbuselibrary{most}



\newtcolorbox{ex}[1][]
  {colback = white, colframe = cyan!75!black, fonttitle = \bfseries,
    colbacktitle = cyan!85!black, enhanced,
    attach boxed title to top center={yshift=-2mm},breakable, 
    title=练习, #1}

\newtcolorbox{blk}[2][]
  {colback = white, colframe = magenta!75!black, fonttitle = \bfseries,
    colbacktitle = magenta!85!black, enhanced,
    attach boxed title to top left={xshift=5mm, yshift=-2mm},breakable, 
    title=#2, #1}


\setcounter{tocdepth}{2}

\setcounter{secnumdepth}{3}



\ctexset {
section = {
	name = {第,节},
 	number = \chinese{section}},
subsection = {
	name = {,、\hspace{-1em}},
	number = \chinese{subsection}
},
subsubsection = {
	name = {(,)\hspace{-1em}},
	number = \chinese{subsubsection},
}
}



\renewcommand{\contentsname}{目~~录}

\newcommand{\poly}{\polynomial[reciprocal]}



\usepackage{mathtools}

\setlength{\abovecaptionskip}{0.cm}
\setlength{\belowcaptionskip}{-0.cm}

\usetikzlibrary{decorations.pathmorphing, patterns}
\usetikzlibrary{calc, patterns, decorations.markings}
\usetikzlibrary{positioning, snakes}

\newcommand{\Lim}{\displaystyle\lim}

\usepackage{yhmath}
\usepackage{longdivision}
\usepackage{polynom}
\usepackage{polynomial}
\usepackage{multicol}

\renewcommand{\frac}{\dfrac}
\newcommand{\oc}{$^{\circ}{\rm C}$}
\usepackage{longtable}
\usepackage{tkz-euclide, tkz-base}
\newcommand{\NC}{\text{N}/\text{C}}
\newcommand{\ms}{\text{m}/\text{s}}
\newcommand{\cms}{\text{cm}/\text{s}}
\newcommand{\msq}{\text{m}/\text{s}^2}
\newcommand{\cmsq}{\text{cm}/\text{s}^2}
\newcommand{\kmh}{\text{km}/\text{h}}
\newcommand{\Int}{\displaystyle\int}
\usepackage{cases}

\newcommand{\R}{\mathbb{R}}
\newcommand{\N}{\mathbb{N}}
\newcommand{\Z}{\mathbb{Z}}

\newcommand{\VEC}{\overrightarrow}
\renewcommand{\Pr}{\mathbb{P}}

\newcommand{\comb}[2]{{\rm C}^{#1}_{#2}}
\newcommand{\perm}[2]{{\rm P}^{#1}_{#2}}
\newcommand{\rep}[2]{{\rm H}^{#1}_{#2}}
\newcommand{\pc}[1]{\cos{#1}+i\sin{#1}}
\newcommand{\pcx}[1]{\cos\left(#1\right)+i\sin\left(#1\right)}




\begin{document}
%\fontsize{10.5}{11}\selectfont














\title{\Huge\bfseries 中学数学实验教材\\第六册(上)}



\author{\Large 中学数学实验教材编写组编}
\date{\Large 1986年5月}

\maketitle




\frontmatter

% \input{preface.tex}
\tableofcontents


\mainmatter

\chapter{复数系}

在数学中,数是最基础、应用最广泛的工具之一,我们已经在自然数的基础上,逐步学习了整数、有理数、实数等概念和它们的运算及性质.这一章,我们将进一步学习复数的概念及其运算、性质.

\section{复数的概念}
\subsection{数的概念的发展}

数的概念是中学数学的主要内容之一,是从实践中逐步形成和发展起来的.早在有文字记载的历史以前,由于计数和排序的需要,人类就建立起自然数(正整数)的概念,处理着各种常见的、有天然单元的可数量. 自然数的全体构成了自然数集$\mathbb{N}$.

随着社会生产和科学技术的发展,数的概念相应地得到了不断发展.

为了处理实践中遇到的测量、分割、分配等可以细分的量,人们引进了正分数;为了解决生产实践中的各种具有相反意义的量的表示和运算问题,人们又引进了零及负数,并把自然数看作正整数. 这样一来,把正整数、零、负整数合并在一起,称为整数集$\mathbb{Z}$;并把整数与分数(正分数、负分数)合称为有理数集$\mathbb{Q}$.

任一有理数,可以表示成分数形式$\frac{m}{n}$,其中$m,n\in\mathbb{N}$; 还可以表示成循环小数(包括循环节为0的小数).

有理数系的形成,使得许多人曾认为这就足以解决可度量的量,特别是在古希腊,他们曾认为:任意两条线段都是可公度的(即:任意两条线段长度的比都是有理数). 但实际上这种认识是不正确的. 例如,正方形的边长和对角线长,正五边形的边长和对角线长都是不可公度的. 为了解决这些量与量之比值不能用有理数表示的矛盾,于是又引进了无理数的概念,无理数就是无限不循环小数. 无理数与有理数统称为实数,构成了实数集$\mathbb{R}$.

实数集$\mathbb{R}$对加、减、乘、除(零不作除数)四则运算是封闭的,且满足运算通性,即满足加、乘法的交换律、结合律和分配律,数零与1保持在运算中的特性.

实数集$\mathbb{R}$与数轴上的点集之间,可以建立起一一对应的关系. 因此,从度量的角度来说,实数系是一个完整无缺的数系,它足以满足度量的实际需要.

但是,从代数运算和解方程的角度来看,实数集并不完整无缺,因为,在实数范围内,开方运算并不封闭,实系数方程(如$ax^2+bx+c=0$)在实数系内未必有解.

事实上,数的概念的逐步扩充,与代数运算和解方程的需要也是相适应的,例如:

由于自然数集$\mathbb{N}$对于减法、除法运算不封闭,因而方程$x+7=2$, $7x=5$等在自然数集中无解,但在有理数集$\mathbb{Q}$中就有解$x=-5$, $x=\frac{5}{7}$. 

由于有理数集$\mathbb{Q}$对于开方运算不封闭,因而方程$x^2-3
=0$在$\mathbb{Q}$中无解,但在实数集$\mathbb{R}$中就有两个解$x=\pm\sqrt{3}$.

同样,实数集$\mathbb{R}$对开方运算仍不封闭,因而象方程$x^2+1=0$在$\mathbb{R}$中还是无解,因为,没有一个实数的平方等于$-1$.要解决这一类矛盾,数的概念需要进一步发展,这就需要引进新数,并讨论它的运算.

\begin{ex}
\begin{enumerate}
    \item 试叙述实数集$\mathbb{R}$中有哪些数系运算通性?
\item 试分别在自然数集$\mathbb{N}$.整数集$\mathbb{Z}$、有理数集$\mathbb{Q}$和实数集$\mathbb{R}$中,解方程$(x-2)(3x+1)(x+7)(x^2+5)=0$.
\end{enumerate}
\end{ex}

\subsection{复数及其几何表示}
如上所述,由于解方程的需要,早在十六世纪,人们就引进了一个新数$i$,使得方程有解,并把它叫做\textbf{虚数单位},并规定:
\begin{enumerate}[(1)]
\item $i^2=-1$;
\item 它与实数进行四则运算时,运算通性及数0与1的运算特性仍然保持. 即,满足加、乘法的交换律、结合律、分配律;满足
\[\begin{split}
    0+i&=i+0=i,\\
0\cdot i&=i\cdot 0=0,\\
1\cdot i&=i\cdot 1=i.
\end{split}\]
\end{enumerate}

在这样的规定下,$i$可以与实数$b$相乘,再与实数$a$相加,从而得到$a+bi$.

这样,数的概念扩大了,我们就把形如$a+bi\; (a,b\in\mathbb{R})$的数,叫做复数.全体复数构成的集合,称为\textbf{复数集},
一般记作$\mathbb{C}$.

对于复数$a+bi\; (a,b\in\mathbb{R})$,当$b=0$时,就是实数$a$;当$b\ne 0$时,这一复数叫做虚数,特别地,当$a=0$且$b\ne 0$时,叫做纯虚数.

$a$,$b$分别叫做复数$z=a+bi$的\textbf{实部}与\textbf{虚部},并且记作:${\rm Re}(z)=a$, ${\rm Im}(z)=b$.

显然,实数集$\mathbb{R}$是复数集$\mathbb{C}$的真子集,即$\mathbb{R}\subset \mathbb{C}$.

例如,$3+2i$, $-\frac{1}{2}-\sqrt{3}i$, $-0.25i$都是虚数. 其中$-0.25i$是纯虚数.它们的实部和虚部分别为:
\[\begin{split}
&{\rm Re}(3+2i)=3,\qquad {\rm Im}(3+2i)=2;\\
&{\rm Re}\left(-\frac{1}{2}-\sqrt{3}i\right)=-\frac{1}{2},\qquad {\rm Im}\left(-\frac{1}{2}-\sqrt{3}i\right)=-\sqrt{3};\\
&{\rm Re}(-0.25i)=0,\qquad {\rm Im}(-0.25i)=-0.25.\\
\end{split}\]

我们还规定,当且仅当两个复数的实部和虚部分别相等时,这两个复数相等,即
\[a+bi=c+di\Longleftrightarrow a=c\text{且}b=d\]
或
\[z_1=z_2\Longleftrightarrow {\rm Re}(z_1)={\rm Re}(z_2)\quad \text{且} \quad {\rm Im}(z_1)={\rm Im}(z_2)\]
特别地,我们可以得出
\[a+bi=0\Longleftrightarrow a=b=0\]

\begin{example}
    若$3x+2yi-1=(x-1)i+y$,其中$x,y\in\mathbb{R}$. 求$x$与$y$的值.
\end{example}

\begin{solution}
将已知等式整理得
\[(3x-1)+2yi=y+(x-1)i\]
根据复数相等的定义,可得方程组$\begin{cases}
    3x-1=y\\
    2y=x-1
\end{cases}$,
解得
\[x=\frac{1}{5},\qquad y=-\frac{2}{5}\]
\end{solution}

当两个复数的实部相等、虚部互为相反数时,这两个复数叫做\textbf{互为共轭复数},即
:复数$a+bi$与$a-bi$叫做互为共轭复数. 一般地,复数$z=a+bi$的共轭复数表示为:
$$\overline{z}=a-bi$$

显然,实数$a$(即虚部为零的复数)的共轭复数仍是它自身$a$.

我们已经知道,实数集与数轴上的点之间可建立一一对应关系. 引进虚数,构成复数集之后,是否有相应的对应关系呢?

首先,虚数单位$i$在实数轴上没有立足之地,它不属于实数集;其次,当$b\ne 0$时,复数$a+bi$也不可能在实数轴上找出对应之点,它仍不属于实数集,因此,我们必须另找途径来建立对应关系.

从复数及其相等的定义中,我们知道,任何一个复数
$z=a+bi$,都由一个有序实数对$(a,b)$唯一确定,而每一个有序实数对,又借助平面的坐标化可以唯一确定平面上一个点,因此,只要在平面上建立起直角坐标系,就可以用平面上的点$Z(a,b)$表示出复数$z=a+bi$,如图1.1所示.

\begin{minipage}{.45\textwidth}
\centering
\begin{tikzpicture}[>=stealth]
\draw[->](-1,0)--(3,0)node[right]{$x$};
\draw[->](0,-1)--(0,3)node[right]{$y$};
\draw[dashed](0,1.5)node[left]{$b$}--(2,1.5)node[above]{$Z(a,b)$}--(2,0)node[below]{$a$};
\node[below right]{$O$};
\draw[fill](2,1.5)circle(1pt);
\end{tikzpicture}
\captionof{figure}{ }
\end{minipage}
\hfill
\begin{minipage}{.45\textwidth}
\centering
    \begin{tikzpicture}[>=stealth]
  \draw[->](-1,0)--(3,0)node[right]{$x$};
\draw[->](0,-2)--(0,2)node[right]{$y$};
\draw[dashed](0,1.25)node[left]{$b$}--(2,1.25)node[above]{$Z(a,b)$}--(2,0)node[below right]{$a$}--(2,-1.25)node[below]{$\overline{Z}(a,-b)$}--(0,-1.25)node[left]{$-b$};
\node[below right]{$O$};  
\draw[fill](2,1.25)circle(1pt);
\draw[fill](2,-1.25)circle(1pt);
    \end{tikzpicture}
\captionof{figure}{ }
\end{minipage}


这个建立了坐标系来表示复数的平面,叫做\textbf{复平面},其中$X$轴叫做\textbf{实轴},$Y$轴(不含原点)叫做\textbf{虚轴}.

这样一来,每一个复数,在复平面内都有唯一一点和它对应,反之,复平面内的每一个点,都有唯一的一个复数和它对应,因此,复数集$\mathbb{C}$和复平面上的点集是一一对应的. 特别地,实数集和复平面实轴上的点集一一对应;纯虚数集合和复平面虚轴(不含原点)上的点集一一对应.

显然,互为共轭的两个复数$z=a+bi$与$\bar z=a-bi$在复平面上对应的两点$Z(a,b)$与$\overline{Z}(a,-b)$关于实轴对称(图1.2).

\begin{ex}
\begin{enumerate}
    \item 将下列给出的各数,分类填入表中:
\[1-\sqrt{2},\quad 0.618,\quad -1+i,\quad 0,\quad i,\quad i^2,\quad 3+7\sqrt{2}i\]
\[i-1,\quad (\sqrt{2}+\sqrt{3})i,\quad \sqrt{2}i-\sqrt{2},\quad -\frac{\sqrt{3}}{2}i\]
\begin{tabular}{|c|p{.7\textwidth}|}
\hline
    实数&\\
\hline
虚数&\\\hline
纯虚数&\\    \hline
\end{tabular}

\item 填写下列表格:
\begin{center}
\begin{tabular}{|c|p{.1\textwidth}|p{.1\textwidth}|p{.1\textwidth}|p{.1\textwidth}|p{.1\textwidth}|}
\hline 
复数$z$& ${\rm Re}(z)$& ${\rm Im}(z)$&$\bar z$& ${\rm Re}(\bar z)$& ${\rm Im}(\bar z)$\\
\hline
$-\sqrt{3}+i$ &&&&&\\
$-(1+\sqrt{2})$&&&&&\\
$\frac{i-\sqrt{2}}{4}$&&&&&\\
$0$&&&&&\\
$i$&&&&&\\
$a-bi$&&&&&\\
\hline
\end{tabular}
\end{center}
\item 试求出满足下列等式的实数$x$,$y$:
\begin{enumerate}[(1)]
\item $(3x+2y)+(5x-y)i=17-2i$;
\item $3x+2y-7=(2x+3y-8)i$;
\item $7x+(4y-1)i-1=0$.
\end{enumerate}

\item 试说明对任意复数$z$,都有$\overline{\bar z}=z$.
\item 在复平面上,画出下列复数及它们的共轭复数所对应的点:
\[2-2i;\quad -2+2i;\quad -2i; \quad \sqrt{2};\quad 0.\]
\end{enumerate}
\end{ex}

\subsection{复数的模与幅角}
我们已经知道,复数集与复平面上的点集之间是一一对应的,即每一复数$z=a+bi\; (a,b\in\R)$都对应着复平面
上唯一一点$Z(a,b)$,反之亦然。

如果我们在复平面上连结原点$O$与表示复数$z=a+bi$的$Z(a,b)$,就得到一个向量(从$O$点指向$Z$点),记作$\VEC{OZ}$,这样就可以在复数同向量(从原点出发的向量)之间
建立起联系. 不难看出,任一复数$z$可唯一确定复平面上一点$Z$,进而唯一确定一个向量$\VEC{OZ}$;反之,复平面上任一向量$\VEC{OZ}$,可唯一确定一点$Z$,进而唯一确定一个复数$z$,因此,复数集$\mathbb{C}$与复平面内所有以原点$O$为起点的向量(位置向量)集合也是一一对应的。

这样一来,我们就可以很方便地把复数$z=a+bi\; (a,b\in\R)$说成点$Z(a,b)$,或说成向量$\VEC{OZ}$. 研究复数的运算、性质就可以和研究向量的运算、性质互通了.

如图1.3所示,我们把向量$\VEC{OZ}$的模($OZ$的长度)$r$叫做复数$z=a+bi$的模(或绝对值),记作
\[r=|z|\quad \text{或}\quad r=|a+bi|\]
\begin{figure}[htp]
    \centering
\begin{tikzpicture}[>=stealth]
\draw[->](-1,0)--(3,0)node[right]{$x$};
\draw[->](0,-1)--(0,3)node[right]{$y$};
\draw[dashed](0,1.5)node[left]{$b$}--(2,1.5)node[above]{$Z(a,b)$}--(2,0)node[below]{$a$};
\draw[very thick,->](0,0)--node[left]{$r$}(2,1.5);
\draw(.5,0) arc (0:35:.5)node[right]{$\theta$};
\node[below right]{$O$};
\end{tikzpicture}
    \caption{}
\end{figure}

显然
\[r=|z|=|a+bi|=\sqrt{a^2+b^2}\ge 0\]

特别地,当$b=0$时,$z$是一个实数,这时
\[r=|z|=|a|\qquad  \text{(与实数的绝对值概念一致)
}\]

又以实轴$OX$的正半轴为始边,向量$\VEC{OZ}$所在的射线为终边的角$\theta$,叫做复数$z=a+bi$的\textbf{幅角}.

由于任一不等于零的复数,按以上规定可有无限多个幅
角(终边相同的角)的值,因而,我们进一步约定:

满足$0<\theta<2\pi$的幅角$\theta$的值,叫做幅角的\textbf{主值},并记作$\arg z$,即$0\le \arg z<2\pi$. 
因此,任一非零复数$z$的幅角$\theta=\arg z+2k\pi\; (k\in\Z)$

这样一来,每一个不等于零的复数,就只有唯一的模与唯一的幅角主值,并可被它的模与幅角主值所唯一确定。

应该指出,由于复数$z=0$对应于复平面上的零向量(一个点),而且零向量的方向是任意的,因此,复数$z=0$的模$|z|=0$,它的幅角却是任意的。

\begin{example}
试比较复数$z_1=1+\sqrt{2}i$,$z_2=-\frac{1}{2}-i$的模的大小.
\end{example}

\begin{solution}
$\because\quad |z_1|=\sqrt{1+2}=\sqrt{3},\quad |z_2|=\sqrt{\frac{1}{4}+1}=\frac{1}{2}\sqrt{5},\qquad \sqrt{3}>\frac{1}{2}\sqrt{5}$

$\therefore\quad |z_1|>|z_2|$.
\end{solution}

一般来说,两个复数中,至少有一个不是实数,就不能比较它们的大小,但任意两个复数的模,总是可以比较大小的.

\begin{example}
    如果$a$,$b$都是正实数,试求:
\begin{multicols}{4}
\begin{enumerate}[(1)]
    \item $\arg a$
    \item $\arg (-a)$
    \item $\arg (bi)$
    \item $\arg (-bi)$
\end{enumerate}
\end{multicols}
\end{example}

\begin{solution}
由于$a$,$b$是正实数,因此:
复数$a$,$-a$分别对应着复平面中正、负半实轴上的点(不包含原点). 所以
\begin{multicols}{2}
    \begin{enumerate}[(1)]
        \item $\arg (a)=0$
        \item $\arg (-a)=\pi$
    \end{enumerate}
    \end{multicols}
复数$bi$,$-bi$分别对应着复平面中正、负半虚轴上的点(不包含原点). 所以
\begin{multicols}{2}
    \begin{enumerate}[(1)]\setcounter{enumi}{2}
        \item $\arg (bi)=\frac{\pi}{2}$
        \item $\arg (-bi)=\frac{3\pi}{2}$
    \end{enumerate}
    \end{multicols}
\end{solution}

\begin{example}
    设$z\in\mathbb{C}$,试说明满足下列条件的点Z的集合组成什么样的图形?
\begin{multicols}{2}
\begin{enumerate}[(1)]
    \item $|z|=a$
    \item $2<|z|\le 4$
\end{enumerate}
\end{multicols}
\end{example}

\begin{solution}
\begin{enumerate}[(1)]
    \item 复数$z$的模等于$a\ge 0$,就是说它对应于复平面上的向量$\VEC{OZ}$的长度为$a$,也就是说它对应于复平面上的点$Z$到原点的距离为$a$. 所以,满足条件$|z|=a$的点的集合,当$a>0$时组成一个以$O$为圆心、以$a$为半径的圆;当$a=0$时,就是一个点(原点)。
\item 先将条件$2<|z|\le 4$化为$\begin{cases}
    |z|>2\\ |z|\le 4
\end{cases}$

\begin{minipage}{.45\textwidth}
 再分别考虑:满足$|z|>2$的点集组成圆$|z|=2$的外部;满足$|z|\le 4$的点集组成圆$|z|=4$的内部(包含边界)。同时满足以上两条件的点集,就是已求两集合的交集,也就是所要求的集合.

因此,满足$2<|z|\le 4$的点的集合是以$O$为圆心,以2及4为半径的两图所夹的圆环,其中包含外边界,但不包括内边界(图1.4).   
\end{minipage}\hfill
\begin{minipage}{.45\textwidth}
    \centering
\begin{tikzpicture}[>=stealth, scale=.4]

\draw[ thick, pattern=north east lines](0,0) circle (4);
\fill[color=white](0,0) circle (2);    
\draw[dashed,  thick](0,0) circle (2);    
\draw[->](-6,0)--(6,0)node[right]{$x$};
\draw[->](0,-6)--(0,6)node[right]{$y$};
\node [below right]{$O$};
\node at (2,0)[below right]{2};
\node at (4,0)[below right]{4};
\node at (0,2)[below right]{2};
\node at (0,4)[above right]{4};

\end{tikzpicture}
    \captionof{figure}{ }
\end{minipage}

\end{enumerate}    
\end{solution}

\begin{ex}
\begin{enumerate}
    \item 求出下列复数的模,并按其大小用不等号连起来:
\[\sqrt{2}-i,\quad -2+i,\quad 7,\quad 2i,\quad -1-i\]
  \item   设$|z|=r$, $\arg z=a$,试求出$|\bar z|$, $\arg \bar z$.
  \item 求证:复数$z_1=-3i$, $z_2=\sqrt{2}+\sqrt{7}i$, $z_3=\sqrt{7}-\sqrt{2}i$, $z_4=-2+\sqrt{5}i$所对应的四个点共圆,并写出个这圆周上的所有点应满足的条件.
\item   你能指出满足下列条件的复数所对应的点集组成什么图形吗?
\begin{multicols}{2}
\begin{enumerate}[(1)]
    \item $\arg z=\frac{\pi}{4}$
    \item $\frac{\pi}{4}\le \arg z\le \frac{\pi}{2}$
\end{enumerate}
\end{multicols}
\end{enumerate}

\end{ex}

\subsection*{习题1.1}
\begin{enumerate}
    \item 设复数$z= (a^{2}- 2a- 3 )+( a^{2}+ 2a- 3) i$, $a\in \mathbb{R} $, 
试问在什么条件下,这一复数是
\begin{multicols}{4}
\begin{enumerate}[(1)]
    \item 实 数 ?
    \item 零?
    \item 虚数?
    \item 纯虚数?
\end{enumerate}
\end{multicols}

\item 实数$a,b$取何值时,才能使复数$z_1=z_2$:
\begin{enumerate}[(1)]
    \item $z_{1}= \left(\frac{1}{2}a+ b \right)+\left( 5a+\frac{2}{3}b \right) i,\quad z_{2}= - 4+ 16i$ ;
    \item $z_{1}=(a+b)+24i,\quad z_{2}=-(5+abi)$;
    \item $z_{1}=8+2abi,\quad z_{2}=6i+a^{2}-b^{2}$;
    \item $z_{1}=2a^{2}-5a+2,\quad z_{2}=(b^{2}+b-2)i$;
    \item $z_{1}= 0,\quad z_{2}= (a- 1)^{2}+ (b^{2}+ 2b)i$.
\end{enumerate}
\item 求出下列各复数的共轭复数,并在复平面上描出表示这些互为共轭复数的点。
\[1- i, \quad - 2+ \sqrt {2}i, \quad - \sqrt {3}i, \quad 2, \quad i, \quad \sqrt {3}+ i\]
\item 求出上题中各复数的模及幅角的主值(可用反三角函数表示).
\item 实数$x,y$取何值时,才能使$z_1=\bar z_2$:
\begin{enumerate}[(1)]
    \item $z_{1}=( x-y )+5i,\quad  z_{2}=1-( 2x+y ) i$;
    \item $z_{1}=xy-( x+y)i,\quad  z_{2}=5+24 i$; 
    \item $z_{1}=( x^{2}-3x-2 )+( y^{2}-5y-6 ) i,\quad z_{2}=2$;
    \item $z_1=0,\quad z_2=(2x^2-5x+2)+i(y^2+y-2)$.
\end{enumerate}


\item 求证:两个复数互为共轭的必要条件是这两个复数
的模相等。
\item  在下列各题中,要使等式$| z_1|=2| z_2|$成立,试
求实数$x,y$的值:
\begin{enumerate}[(1)]
    \item $z_{1}=x-1-2i,\quad  z_{2}=1+( x+1 )i$;
    \item $z_{1}=( x-y+1 )+2 ( x+y-3 )i,\quad z_{2}=\frac{x-y+1}{2\sqrt{2}}i $
\end{enumerate}

\item  设$a,b\in\mathbb{R}$, 它们取何值时,表示复数$z=(a-b-1)+(2a-b-5)i$在复平面上的点,才能在$y=x$直线上?
\item 试求满足下列条件的复数$x+yi$在复平面上对应的点
的轨迹:
\begin{multicols}{3}
\begin{enumerate}[(1)]
    \item $|x+yi|=3$
    \item $|x+yi|<3$
    \item $|x+yi|\ge 3$
    \item $2\le |x+yi|\le 3$
    \item $0<|x+yi|<3$
\end{enumerate}
\end{multicols}
\item 设$z=a+bi,\; a,b\in\mathbb{R}$. 问:满足下列条件的点
$Z(a,b)$的集合组成什么图形?
\begin{multicols}{2}
\begin{enumerate}[(1)]
    \item $0\le |a|<1$
    \item $a>0,\; b>0,\quad a^2+b^2>4$
\end{enumerate}
\end{multicols}

\item 想想看:若已知一个非零复 数$z=a+bi$, 你能求
出它的幅角来吗?它的主值又如何求出?举例说明。

\item 满足条件$\frac{\pi}{6}\le \arg z\le \frac{\pi}{2}$的复数,在复平面上对应着什么样的点集?

\end{enumerate}

\section{复数的四则运算}
本节讨论复数的四则运算,我们将根据引进的新数——虚数单位$i$的特征$i^2=-1$和希望它遵从“数系运算通性”的要求,合理地提出运算法则,并进一步应用这些法则,有效地进行运算,同时,根据复数的几何意义,我们将相应地给出复数运算的几何意义.

\subsection{复数的加、减法}
\subsubsection{加法}
复数的加法运算按照以下规定的法则进行:设$z_1=a+bi$, $z_2=c+di$是任意两个复数,则
\[z_1+z_2=(a+bi)+(c+di)=(a+c)+(b+d)i\]

可见,任意两个复数的和仍然是一个复数.

特别地,根据这一规定我们可求出任一个实数$a$和任一纯虚数$bi$的和:
\[(a+0i)+(0+bi)=(a+0)+(0+b)i=a+bi\]

由此可知,复数$a+bi$,既可认定为一个数,又可认定它为实数$a$与纯虚数$bi$的和,今后,对于表达复数$a+bi$中的“$+$”号与表达加法运算的“$+$”号,我们将认为是一样的。

不难验证,复数加法法则的规定满足“运算通性”。
即 : 设 $z_1, z_2, z_3\in \mathbb{C}$, 则
\begin{align}
    z_{1}+ z_2 &= z_{2}+ z_{1}\tag{交换律成立}\\
    z_{1}+\left( z_{2}+z_{3} \right)&=\left( z_{1}+z_{2} \right)+z_{3}=z_{1}+z_{2}+z_{3}\tag{结合律成立}\\
    z_{1}+ 0&= 0+ z_{1}= z_{1}\tag{零的运算特性保持}
\end{align}

\begin{example}
    试求下列复数的和:
\begin{enumerate}[(1)]
    \item $(5-3i)+(-\sqrt{2}+i)$
    \item $( 9+ 6i) + ( 9- 6i)$
    \item $( 1- i) + ( - 1+ 2i) + 7+ \sqrt {2}i$
\end{enumerate}
\end{example}

\begin{solution}
\begin{enumerate}[(1)]
    \item $( 5- 3i) + ( - \sqrt {2}+ i) = 5- \sqrt {2}$
    \item $( 9+ 6i) + ( 9- 6i) = 18+ 0i= 18$
    \item $( 1- i) + ( - 1+ 2i) +7+ \sqrt {2}i=(1-1+7)+(-1+2+\sqrt{2})i=7+(1+\sqrt{2})i$
\end{enumerate}
\end{solution}


由于复数可以用复平面上的“位置”向量来表示,因此,复数加法的几何意义就是:复数的加法可以按照向量的“平行四边形”法则来进行,即,先将求和的两个复数所对应的向量$\overrightarrow{OZ_{1}}$, $\overrightarrow{OZ_{2}}$画出,若这两个向量不共线,再以这两个向量为邻边作平行四边形$OZ_1ZZ_2$, 则这个平行 四 边形的对角线$OZ$所示的向量$\overrightarrow{OZ}$, 就 对 应 着 已 知 两 个 复 数 的 和(如图1.5).

\begin{figure}[htp]
    \centering
\begin{tikzpicture}[>=stealth, scale=.8]
\draw[->](-1,0)--(4,0)node[right]{$x$};
\draw[->](0,-1)--(0,4)node[right]{$y$};
\node [below right]{$O$};
\draw[very thick, ->](0,0)--(.5,2)node[above]{$Z_2$};
\draw[very thick, ->](0,0)--(2.5,1)node[right]{$Z_1$};
\draw[very thick, ->](0,0)--(3,3)node[above]{$Z$};
\draw[dashed](.5,2)--(3,3)--(2.5,1);


\end{tikzpicture}
    \caption{}
\end{figure}

$\overrightarrow{OZ_{1}}$ 与$\overrightarrow{OZ_{2}}$共线,只要在$\overrightarrow{OZ_{1}}$的方向上紧接着延长一个$\overrightarrow{Z_{1}Z}$,使$\overrightarrow{Z_{1}Z}=\overrightarrow{OZ_{2}}$. 这时同样得到向量$\overrightarrow{OZ}$, 它对应着已识两复数的和(相当于作一个退缩成一条线段的平行四边形).

显然,两个共轭复数的和是一个实数,它对应着实轴上的一个向量. 

\subsubsection{减法}
复数的减法仍定义为加法
的逆运算,即:满足$(c+di)+(x+yi)=a+bi$的复数$x+yi$,就叫做复数$a+b$减去$c+di$的差,记作
\[x+yi=(a+bi)-(c+di)\]

根据复数加法法则和复数相等的定义,可以得出复数的减法法则
\[(a+bi)-(c+di)=(a-c)+(b-d)i\]

可见,两复数的差,仍是一个确定的复数,就是说,复数集对于减法也是封闭的.

同样,复数的减法的几何意义就是用向量减法的“三角
形法则”来表示,即:两个复数的差
$z_1-z_2$(向量$\VEC{OZ_1}-\VEC{OZ_2}$)与连结两个向量终点并
指向被减向量终点的向量对应
(图1.6),即
\[\VEC{OZ_1}-\VEC{OZ_2}=\VEC{Z_2Z_1}\]

\begin{figure}[htp]
    \centering
\begin{tikzpicture}[>=stealth, scale=.8]
\draw[->](-1,0)--(4,0)node[right]{$x$};
\draw[->](0,-1)--(0,4)node[right]{$y$};
\node [below right]{$O$};
\draw[ thick, ->](0,0)--(3,3)node[above]{$Z_2$};
\draw[ thick, ->](0,0)--(2.5,1)node[right]{$Z_1$};
\draw[very thick, ->](3,3)--(2.5,1);


\end{tikzpicture}
    \caption{}
\end{figure}

综上所述,复数的加、减法与我们熟悉的多项式加、减法是类似的,就是把复数的实部、虚部分别相加、减,即
\[(a+bi)\pm (c+di)=(a\pm c)+(b\pm d)\]

\begin{example}
    计算$(3-4i)+(-2+i)-(5+2i)$
\end{example}

\begin{solution}
 $$(3-4i)+(-2+i)-(5+2i)
=(3-2-5)+(-4+1-2)i=-4-5i.$$   
\end{solution}

\begin{example}
    设复数$z_1=a+bi$, $z_2=c+di$, 试求这两复数在复平面上所描述的两点之间的距离.
\end{example}

\begin{solution}
    由于$z_1$与$z_2$在复平面上所描述的两点$Z_1$, $Z_2$之间的距离$d$,就是向量$\VEC{Z_1Z_2}$的模,因而,只要求出向量$\VEC{Z_1Z_2}$所对应的复数,并求出它的模即可。

    $\because\quad \VEC{Z_1Z_2}=\VEC{OZ_2}-\VEC{OZ_1}$

$\therefore\quad \VEC{Z_1Z_2}$对应于复数$z_2-z_1$,于是
\[\begin{split}
d&=|\VEC{Z_1Z_2}|=|z_2-z_1|\\
&=|(c+di)-(a+bi)|=|(c-a)+(d-b)i|\\
&=\sqrt{(c-a)^2+(d-b)^2}    
\end{split}\]
\end{solution}

\begin{example}
    根据复数的几何意义及向量表示,试求复平面内以$P(a,b)$点为圆心,$r$为半径的圆的方程.
\end{example}

\begin{figure}[htp]
    \centering
\begin{tikzpicture}[>=stealth, scale=1]
\draw[->](-1,0)--(4,0)node[right]{$x$};
\draw[->](0,-1)--(0,4)node[right]{$y$};
\node [below right]{$O$};
\draw[ thick, ->](0,0)--(1.5,3)node[above]{$Z$};
\draw[ thick, ->](0,0)--(2,2)node[right]{$P$};
\draw[thick](2,2)circle(1.12);
\draw[->, very thick](2,2)--node[right]{$r$}(1.5,3);

\end{tikzpicture}
    \caption{}
\end{figure}


\begin{solution}
    设圆心$P(a,b)$与复数$p=a+bi$对应,圆上任一点$Z(x,y)$与复数$z=x+yi$对应,由于已知圆的半径为$r$,因
此,向量$\VEC{PZ}$的模就是定长$r$. 即
$|\VEC{PZ}|=r$.

又由于$\VEC{PZ}=\VEC{OZ}-\VEC{OP}$,因而向量$\VEC{PZ}$就对应着复数$z-p=(x+yi)-(a+bi)=(x-a)+(y-b)i$.

所以,圆上任一点$Z$应满足$|\VEC{PZ}|=|\VEC{OZ}-\VEC{OP}|=r$,即
\[|z-p|=r\]
这就是复平面上以$P$为圆心,$r$为半径的圆的方程.

若进一步求出$|z-p|=|(x-a)+(y-b)i|
=\sqrt{(x-a)^2+(y-b)^2}$
代入圆方程,并化简。就得出用实数表示的圆的一般方程:
\[(x-a)^2+(y-b)^2=r^2\]
特别地,当$P$点在原点时,圆的方程就变成为
$$|z|=r$$
或
$$x^2+y^2=r^2$$
\end{solution}

\begin{ex}
\begin{enumerate}
    \item 利用复数加法法则和实数、虚数单位运算通性,验证:复数加法的交换律和结合律都成立.
    \item 计算(分别求和,再求差):
\begin{multicols}{2}
\begin{enumerate}[(1)]
    \item $(2+3i)\pm (1+i)$
    \item $(2+4i)\pm (1-i)$
    \item $(-3i-2)\pm (-1-i)$
    \item $-2i\pm (-4+i)$
    \item $(2-3i)\pm (-5i)\pm(-1+7i)$
\end{enumerate}
\end{multicols}
\item 将上题中的各小题,用向量法求和.
\item 求证:两个互为共轭的复数之和为实数;两互为共轭的复数之差为纯虚数或零.
\item 想想看:在复平面上,以$P(a,b)$为圆心、以r为半径的圆内所有点的集合,可用什么样的条件给定?
\end{enumerate}
\end{ex}

\subsection{复数的乘、除法}
\subsubsection{乘法}
复数的乘法运算按照以下规定的法则进行:设$z_1=a+bi$, $z_2=c+di$是任意两复数,则
\[z_1\cdot z_2=(a+bi)\cdot (c+di)=(ac-bd)+(ad+bc)i\]
其中,由于我们要求乘法运算仍应保持“运算通性”,因而,这一规定说明复数的乘法与多项式乘法是类似的,只要将结果中的$i^2$换成$-1$,分别合并实部与虚部即可.

可见,任意两个复数的乘积仍然是一个复数,也就是说,复数集对于乘法也是封闭的.

特别地,根据这一规定,对两个互为共轭复数$z$与$\bar z$,我们有
\[z\cdot \bar z=(x+yi)\cdot (x-yi)=x^2+y^2\]

因此,互为共轭的两复数之积是一个实数,它等于每一复数的模的平方,即
\[z\cdot \bar z=|z|^2=|\bar z|^2\]

容易验证,复数乘法运算也具有“数系运算通性”,即对于任意$z_1,z_2,z_3\in\mathbb{C}$,都有
\begin{align}
z_1\cdot z_2  &= z_2\cdot z_1 \tag{乘法交换律成立}\\
z_1\cdot (z_2\cdot z_3)  &= (z_1\cdot z_2)\cdot z_3=z_1\cdot z_2\cdot z_3 \tag{乘法结合律成立}\\
z_1\cdot (z_2+ z_3)  &= z_1\cdot z_2+ z_1\cdot z_3 \tag{乘法对加法的分配律成立}\\
z_1\cdot 0  &= 0\cdot z_1=0 \tag{零的运算特性保持}\\
z_1\cdot 1  &= 1\cdot z_1=z_1 \tag{1的运算特性保持}
\end{align}

还应该指出,复数乘方运算、幂的概念与实数完全相同,而且由于复数的乘法满足交换律、结合律,所以在实数集$\R$中的指数运算律,在复数集$\mathbb{C}$中仍然成立,即

对任意$z_1,z_2,z\in\mathbb{C}$及$m,n\in\N$, 都有
\[\begin{split}
    z^m\cdot z^n&=z^{m+n}\\
    (z^m)^n&=z^{m\cdot n}\\
    (z_1\cdot z_2)^n&=z^{n}_1\cdot z^{n}_2\\
\end{split}\]

这样,我们由定义$i^2=-1$,运用指数运算律可以得出:
\[\begin{split}
i^3&=i\cdot i^2=-i\\
i^4&=i^2\cdot i^2=(-1)\cdot (-1)=1\\
i^5&=i^4\cdot i=i\\
\cdots&\cdots
\end{split}\]

一般地,对任意$n\in\mathbb{N}$,我们可归纳出:
\[\begin{split}
    i^{4n}&=(i^4)^n=1^n=1\\
    i^{4n+1}&=(i^4)^n\cdot i=i\\
    i^{4n+2}&=(i^4)^n\cdot i^2=-1\\
    i^{4n+3}&=(i^4)^n\cdot i^3=-i\\
\end{split}\]
即,对于数$n\in\N$,则
\[i^{4n}=1,\qquad i^{4n+1}=i,\qquad i^{4n+2}=-1,\qquad i^{4n+3}=-i\]

\begin{example}
    计算$(5-4i)(1-i)(-2+3i)$.
\end{example}

\begin{solution}
\[(5-4i)(1-i)(-2+3i)=(1- 9i) (- 2+ 3i) = 25+ 21i\]
\end{solution}

\begin{example}
    计算$\left(-\frac12+\frac{\sqrt{3}}{2}i\right)^6$
\end{example}

\begin{solution}
\[\begin{split}
    \left(-\frac12+\frac{\sqrt{3}}{2}i\right)^6&=\left[\left(-\frac12+\frac{\sqrt{3}}{2}i\right)^2\right]^3\\
&= \left(\frac{1}{4}-\frac{\sqrt{3}}{2}i+\frac{3i^{2}}{4}\right)^{3}=\left(-\frac{1}{2}-\frac{\sqrt{3}}{2}i\right)^3\\
&=\left( - \frac 12\right) ^{3}- 3\left( \frac {- 1}{2}\right)^{2}\cdot \frac {\sqrt {3}}2i+3\left(-\frac{1}{2}\right)\left(\frac{\sqrt{3}}{2}i\right)^{2}-\left(\frac{\sqrt{3}}{2}i\right)^{3}\\
&=-\frac{1}{8}-\frac{3\sqrt{3}}{8}i+\frac{9}{8}+\frac{3\sqrt{3}}{8}i=1
\end{split}\]
\end{solution}


\subsubsection{除法}

复数的除法同样定义为乘法的逆运算,即:满足$(c+di)(x+yi)=a+bi\; (c+di\neq0)$的复数$x+yi$叫做复数$a+bi$除以$c+di$的商,记作
$$x+yi=\frac{a+bi}{c+di}\quad \text{或}\quad( a+bi )\div( c+di ) $$

根据两个共轭复数的乘积是一个实数以及乘法运算法
则,可以得出复数的除法法则:

当$c+ di\neq 0$ 时,
\[\frac{a+bi}{c+di}=\frac{(a+bi)(c-di)}{(c+di)(c-di)}=\frac{ac+bd}{c^2+d^2}+\frac{bc-ad}{c^2+d^2}i\]
其中由于$c+di\ne 0$,因而$c$、$d$不同时为零,所以,$c^2+d^2\ne 0$.可见这样求得的商是唯一确定的一个复数,也就是说,复数集$\mathbb{C}$对于除法(除数不为零)也是封闭的.

\begin{example}
    计算$(2-3i)\div (-3+4i)$
\end{example}

\begin{solution}
\[\begin{split}
    (2-3i)\div (-3+4i)&=\frac{2-3i}{-3+4i}=\frac{(2-3i)(-3-4i)}{(-3+4i)(-3-4i)}\\
    &=\frac{-18+i}{25}=-\frac{18}{25}+\frac{1}{25}i
\end{split}\]
\end{solution}

可以看出,复数乘、除法法则的公式并不需要硬记,乘法类似于多项式乘法,除法可视为分式的分子、分母同乘以除数的共轭数,并注意正确应用运算通性及$i^2=-1$,就可得出结果.

\begin{ex}
\begin{enumerate}
    \item 验证复数乘法对于加法满足分配律.
    \item 计算下列各式:
\begin{multicols}{2}
\begin{enumerate}[(1)]
\item $(4-3i)(-\sqrt{3}i)$
\item $\left(\frac{1}{2}-\frac{\sqrt{3}}{2}i\right)(1+i)$
\item $\left(\frac{1}{2}-\frac{\sqrt{3}}{2}i\right)^3\cdot \left(\frac{1}{2}+\frac{\sqrt{3}}{2}i\right)^3$
\item $(-\sqrt{77}+\sqrt{112}i)\cdot 0$
\end{enumerate}
\end{multicols}

\item  (口答)计算:
\[i^{13},\quad i^{22},\quad i^{36},\quad i^{43},\quad i^{70},\quad i^{101},\quad i^{355},\quad i^{404} \]
\[\frac{1}{i},\quad \frac{1}{i^2},\quad \frac{1}{i^3},\quad \frac{1}{i^4},\quad \frac{1}{i^{11}}\]
\item     计算:
\begin{multicols}{3}
\begin{enumerate}[(1)]
    \item $\frac{1}{1+i}$
    \item $\frac{1}{\sqrt{2}i}$
    \item $\frac{2i}{1-i}$
    \item $\frac{2+i}{7+4i}$
    \item $\frac{1}{(9+2i)^2}$
    \item $\frac{(1-i)^2}{(1+i)^3}$
\end{enumerate}
\end{multicols}
\end{enumerate}
\end{ex}

\subsection{复数集内的多项式与方程}
我们已经学习过实数范围内的多项式,而且明确多项式的元(未知数)只要求它满足“运算通性”,因此,对于复系数的多项式运算或未知数(元)取复数值的多项式求值,都可以与实系数多项式同样处理.

\begin{enumerate}
    \item 复系数多项式的四则运算举例
    
例如$f(x)=x+i$, $g(x)=2x+1-i$,则
\[\begin{split}
f(x)+g(x)&=3x+1\\
f(x)-g(x)&=-x-1+2i\\
f(x)\cdot g(x)&=2x^2+(1+i)x+(1+i)\\
g(x)&=2\cdot f(x)+1-3i    
\end{split}\]
这就是说,$g(x)$除
以$f(x)$可得商2,余$1-3i$(由综合除法得).

\item 多项式求值举例

例如$f(x)=x^2+x+1$,
则
\[\begin{split}
  f(i)&=i^2+i+1=i\\
f(1+i)&=(1+i)^2+(1+i)+1=2+3i.  
\end{split}\]

\item 复系数多项式的因式分解举例

例如:在复数范围内,$f(x)=x^2+1$可以进一步分解
\[x^2+1=x^2-(-1)=x^2-i^2=(x+i)(x-i)\]

\item 复数范围内的方程与方程组,均可按实数范围时的方法求解

\item 由带余除法导出的多项式的余式定理,对于复系数
的多项式仍然适用. 例如
$f(x)=x^4+2x^3-3x^2+5$除以$x-2i$所得的余式为:
\[R=f(2i)=(2i)^4+2(2i)^3-3(2i)^2+5
=33-16i\]
\end{enumerate}

\begin{example}
    在复数范围内分解因式$x^4-16$
\end{example}

\begin{solution}
\[x^{4}-16=(x^{2}-4)(x^{2}+4)=(x- 2) (x+ 2) (x- 2i) (x+ 2i)\]
\end{solution}

\begin{example}
    解方程$x^2-6x+10=0$
\end{example}

\begin{solution}
    利用求根公式,得
$$x=\frac{6\pm\sqrt{36-40}}{2}=\frac{6\pm\sqrt{-4}}{2}=3\pm i$$
$\therefore\quad x_{1}= 3+ i,\quad x_{2}= 3- i$
\end{solution}

\begin{example}
     解方程
$\begin{cases}
    2x+(1+i)y=6-2i\\
    (3-i)x+2i y=13-i
\end{cases}$
\end{example}

\begin{solution}
   用行列式法解,由于
\[\begin{split}
    \Delta&=\begin{vmatrix}
    2& 1+i\\
    3-i& 2i
\end{vmatrix}=4i-(1+i)(3-i)=-4+2i\\
\Delta_{1}&=\begin{vmatrix}
   6- 2i& 1+i\\
    13-i& 2i
\end{vmatrix}=-10\\
\Delta_{2}&=\begin{vmatrix}
    2& 6-2i\\
     3-i& 13-i
 \end{vmatrix}=10+10i\\
\end{split}\]
$\therefore\quad x= \frac {\Delta_{1}}{\Delta} = - \frac {- 10}{- 4+ 2i}= 2+ i,\quad 
y=\frac{\Delta_{2}}{\Delta}=\frac{10+10i}{-4+2i}=-1-3i$

因此,方程组的解集为$\{(x,\; y)\}=\{(2+i,\;  -1-3i)\}$.
\end{solution}

\begin{ex}
\begin{enumerate}
    \item 已知$f(x)=2x^2+(1+i)x+(1-i)$,试求$f(i)$, $f(-i)$, $f(1+i)$和$f(0)$的值.
    \item 在复数范围内分解因式$2x^4-5x^2-3$.
    \item 解方程组$\begin{cases}
        ix-y=2(i-1)\\
        x-iy=0
    \end{cases}$
\end{enumerate}
\end{ex}

\subsection*{习题1.2}

\begin{enumerate}
    \item 计算:
\begin{enumerate}[(1)]
    \item $\left(\frac{2}{3}+i\right)+\left(1-\frac{2}{3}i\right)-\left(\frac{1}{2}+\frac{3}{4}i\right)$
    \item $[(a+b)+(a-b)i]-[(a-b)-(a+b)i]$
\end{enumerate}

\item 设复数$z_1=3+2i$, $z_2=-2+5i$在复平面上分别对应着点$Z_1,Z_2$. 试求向量$\VEC{Z_1Z_2}$与$\VEC{Z_2Z_1}$所表示的复数.
\item 试求复平面上点$P(2,1)$与点$Q(3,-1)$之间的距离.
\item 求证:两个复数和的共轭复数,等于这两个复数的共轭复数之和.
\item 设$z_1,z_2$是非零复数,用几何方法证明:
\begin{enumerate}[(1)]
    \item $\big||z_1|-|z_2|\big|\le |z_1\pm z_2|\le |z_1|+|z_2|$
    \item $|z_1+z_2|^2+|z_1-z_2|^2=2|z_1|^2+2|z_2|^2$
\end{enumerate}
\item  写出复平面内任意两点连线段的垂直平分线的方程。
\item  设复平面内有两定点$F_1(-\sqrt{5},0)$, $F_2(\sqrt{5},0)$,试求到这两定点的距离之和等于6的点的轨迹方程(椭圆方程).
\item  计算:
\begin{enumerate}[(1)]
    \item $(\sqrt{a}+\sqrt{b}i)(\sqrt{a}-\sqrt{b}i)\quad (a,b\in\mathbb{R})$
    \item $(1-2i)(-0.2+0.3i)(0.5-0.4i)$
    \item $(1+i)+(2-i^{3})+(3-i^{5})+(4-i^{7})$
    \item $(\sqrt{2}-\sqrt{2}i)^{2}\cdot\left(\frac{1}{4}+\frac{1}{4}i\right)$
    \item $(a+bi)^{3}+(a-bi)^{3}$
    \item $\begin{vmatrix}
        1-i&i\\1+i&-i
    \end{vmatrix}+\begin{vmatrix}
        i&1&0\\1&2i&1\\0&1&3i
    \end{vmatrix}$
\end{enumerate}

\item 利用公式$(a+bi)(a-bi)=a^2+b^2$, 把下列各式分解因式:
\begin{multicols}{2}
 \begin{enumerate}[(1)]
    \item $a^{4}-b^{4}$
    \item $x^{2}+2x+3$
    \item $a^{2}+2ab+b^{2}+c^{2}$
    \item $x^{3}-2x+4$
\end{enumerate}   
\end{multicols}


\item 设$\omega = - \frac 12+ \frac {\sqrt {3}}2i$, 求证:
\begin{enumerate}[(1)]
    \item $1+\omega+\omega^{2}=0$
    \item $\omega^{3}=1$
    \item $a^{3}+b^{3}+c^{3}-3abc=(a+b+c)(a+\omega b+\omega^{2}c)(a+\omega^{2}b+\omega c)$
\end{enumerate}
\item 计算:
\begin{multicols}{2}
\begin{enumerate}[(1)]
    \item $\frac{1}{2-4i}$
    \item $\frac{3-2i}{1+i}$
    \item $\frac{1+i}{(1-i)^{2}}$
    \item $\frac{1+2i}{2-i^3}$
    \item $\frac{(1-2i)^2}{3-4i}-\frac{(2+i)^2}{4-3i}$
    \item $\frac{\sqrt{3}-\sqrt{2}i}{\sqrt{3}+\sqrt{2}i}-\frac{\sqrt{2}+\sqrt{3}i}{\sqrt{2}-\sqrt{3}i}$
\end{enumerate}
\end{multicols}

\item 若$z_1,z_2\in\mathbb{C}$, 且$z_1\cdot z_2=0$, 则$z_1,z_2$中至少有一个等于0.

\item 已知$\frac1z=\frac{1}{z_{1}}+\frac{1}{z_{2}}$, 且$z_1=2+3i$, $z_{2}=1-i$,
求$z$.

\item 已知$z^2=5-12i$, 求$z$.

\item 证明下列各命题:
\begin{enumerate}[(1)]
    \item 作两个复数的和、差、积、商(除数不为零)的
共轭复数,分别等于这两个复数的共轭复数的和、差、积、商.
\item 若$f(x)$是一个一元多项式,则对于任意复数
$z$, 有$\overline{f(z)}=f(\bar{z})$.
\end{enumerate}

\item 已知$f(z)=\frac{z^2-z+1}{z^2+z+1}$, 求$f(2+3i)$, $f(1-i)$的值.

\item  如果规定$i^0=1$, $i^{-m}=\frac{1}{i^m}\; (m\in\mathbb{N})$, 证明:对一切整数$n$,以下等式成立:
$$i^{4n}={1},\quad i^{4{n+1}}=i,\quad  i^{4{n+2}}=-1,\quad  i^{4n+3}=-i.$$
\item 根据$i^2=-1$及要求复数运算满足“通性”,试说
明复 数 加 法 、 乘 法 法 则 规 定 是 合 理 的。

\item 怎样的复数$z$, 才能满足条件:
\begin{multicols}{2}
\begin{enumerate}[(1)]
    \item $\overline {z}= z$
    \item $\overline {z}= -z$
    \item $z\cdot \bar z=[{\rm Re}(z)]^2$
    \item $z-\bar z=2[{\rm Im}(z)]$
\end{enumerate}    
\end{multicols}

\item 应用余式定理,求下列各余式;
\begin{enumerate}[(1)]
    \item $f( x) = x^{4}- 3x^{3}+ 2x^{2}- x+ i$
除以$g(x)=x+i$
\item $f( x) = x^{4}- 3x^{3}+ 2x^{2}- x+ i$
除以$h(x)=(1+i)x-1+i$
\end{enumerate}
\item 应用综合除法,将多项式
$f\left(x\right)=x^{3}+3\left({1+i}\right)x^{2}-2(5-3i)x-4\left(1+2i\right)$
展开成$(x+i)$的幂的代数和形式.

\item 解方程组$\begin{cases}
    x_1+2x_2=1+i\\
    ix_2+x_3=2i\\
    ix_1-ix_3=1-i
\end{cases}$
\end{enumerate}

\section{复数的三角形式}

我们已经建立了复数集
\[\mathbb{C}=\{x+yi\mid x,y\in\R,\; i^2=-1\}\]
与复平面上的点集以及复平面上的全体向量(由原点出发)集合间的一一对应关系,并且明确了复数的模与幅角、幅角的主值等概念,这一节我们将学习复数的另一种表达形式. 就是由它的模与幅角所确定的三角形式,这将有助于复数乘法、除法、乘方、开方运算的进一步简便。

\subsection{复数的三角形式}

设$z=a+bi,\; (a,b\in\R)$,对应着复平面上的向量$\VEC{OZ}$(图1.8),并设它的模$|\VEC{OZ}|=r$,幅角为$\theta$. 若$z=a+bi\ne 0$,则可得
\[a=r\cdot \cos\theta,\qquad b=r\cdot \sin\theta\]
\begin{figure}[htp]
    \centering
\begin{tikzpicture}[>=stealth, scale=.7]
\draw[->](-1,0)--(4,0)node[right]{$x$};
\draw[->](0,-1)--(0,3)node[right]{$y$};
\node[below left]{$O$};
\draw[->, very thick](0,0)--(3,2.5)node[above]{$Z$};
\draw[dashed](3,0)--node[right]{$b$}(3,2.5);
\node at (1.5,0)[below]{$a$};
\draw[->](1,0)node[above right]{$\theta$} arc (0:41:1);
\end{tikzpicture}
    \caption{}
\end{figure}


$\therefore\quad z=a+bi=r\left(\pc{\theta}\right)$

因此,对于任何一个复数$z=a+bi$,都可以表示成
$r(\cos\theta +i\sin\theta)$
的形式(其中$r=\sqrt{a^2+b^2}$),并叫做这个复数的三角形式(也叫做复数的极坐标形式);为了方便,我们将$a+bi$叫做复数的代数形式.

复数的代数形式$z=a+bi$与三角形式$z=r(\cos\theta +i\sin\theta)$之间,由图1.8可知有下列互化的关系:
\[\begin{cases}
 a=r\cos\theta\\
b=r\sin\theta   
\end{cases},\qquad \begin{cases}
    r=\sqrt{a^2+b^2}\\ \tan\theta=b/a
\end{cases}\]
($\theta$只要取主值,且要考虑$a$,$b$的正负).

\begin{example}
 将下列复数的代数形式化为三角形式: 
\begin{multicols}{2}
\begin{enumerate}[(1)]
    \item $\sqrt{3}+i$
    \item $-\frac{1}{2}+\frac{\sqrt{3}}{2}i$
    \item $-1$
    \item $\cos\theta-i\sin\theta$
\end{enumerate}
\end{multicols}  
\end{example}

\begin{solution}
\begin{enumerate}[(1)]
    \item $r=\sqrt{3+1}=2$, $\tan\theta=\frac{1}{\sqrt{3}}=\frac{\sqrt{3}}{3}$,且$a,b$均为正值,$\theta$在I象限,因而$\arg(\sqrt{3}
    +i)=\frac{\pi}{6}$

    $\therefore\quad \sqrt{3}+i=2\left(\pc{\frac{\pi}{6}}\right)$
\item $r=\sqrt{\frac{1}{4}+\frac{3}{4}}=1$, $\tan\theta=-\sqrt{3}$,且$a<0, b>0$,$\theta$在II象限,因而$\theta=\frac{2\pi}{3}$(主值)

$\therefore\quad -\frac{1}{2}+\frac{\sqrt{3}}{2}i=\pc{\frac{2\pi}{3}}$

\item $r=1$, $\tan\theta =0$,且$a<0$,因此$\arg(-1)=\pi$

$\therefore\quad -1=\pc{\pi}$

\item $r=\sqrt{\cos^2\theta+\sin^2\theta}=1$

$\therefore\quad \cos\theta-i\sin\theta=\pcx{-\theta}=\pcx{2\pi-\theta}$

\end{enumerate}
\end{solution}


\begin{example}
    将下列复数的三角形式化为代数形式:
\begin{multicols}{2}
\begin{enumerate}[(1)]
    \item $2\left(\pc{\frac{\pi}{4}}\right)$
    \item $\sqrt{2}\left(\pc{\frac{5\pi}{6}}\right)$
    \item $\pc{101\pi}$
    \item $5\left(\pc{\frac{\pi}{2}}\right)$
\end{enumerate}    
\end{multicols}
\end{example}

\begin{solution}
\begin{enumerate}[(1)]
    \item $2\left(\pc{\frac{\pi}{4}}\right)=\sqrt{2}+\sqrt{2}i$
    \item $\sqrt{2}\left(\pc{\frac{5\pi}{6}}\right)=-\frac{\sqrt{6}}{2}+\frac{\sqrt{2}}{2}i$
    \item $\pc{101\pi}=\pc{\pi}=-1$
    \item $5\left(\pc{\frac{\pi}{2}}\right)=5i$
\end{enumerate}  
\end{solution}

\begin{rmk}
复数的三角形式$r(\pc{\theta})$中,必须满足
    $r>0$、式中保持“$+$”号两特征,至于幅角$\theta$则不一定非要取主值,除常用主值表示外,也可以在幅角中任取一个。例如,
\[\sqrt{2}-\sqrt{2}i=2\left[\pcx{-\frac{\pi}{4}}\right]\]
也是复数$\sqrt{2}-\sqrt{2}i$的三角形式.

此外,若$z=a+bi=0$,即$z=0+0i$,这时$r=0$,其幅角是任意实数,则$0=0(\pc{\theta})$.
\end{rmk}

\begin{ex}
\begin{enumerate}
    \item 判断下列给出的是不是复数的三角形式?若不是,请把它们化为三角形式:
\begin{multicols}{2}
    \begin{enumerate}[(1)]
        \item $2\left(\cos\frac{\pi}{4}-i\sin\frac{\pi}{4}\right)$
        \item $-2\left(\pc{\frac{\pi}{3}}\right)$
        \item $\frac{1}{2}\left(\pc{\frac{3\pi}{4}}\right)$
        \item $\cos\frac{7\pi}{5}+i\sin\left(-\frac{3\pi}{5}\right)$
    \end{enumerate}
\end{multicols}
    \item 将复数表示成三角形式:
\begin{multicols}{3}
    \begin{enumerate}[(1)]
        \item $2$
        \item $-3$
        \item $4i$
        \item $-2i$
        \item $1+i$
        \item $-4+3i$
        \item $-1-\sqrt{3}i$
        \item $\frac{\sqrt{3}}{2}-\frac{1}{2}i$
    \end{enumerate}
\end{multicols}
\item    写出下列复数的代数形式:
\begin{multicols}{2}
    \begin{enumerate}[(1)]
        \item $3\left(\pc{\frac{\pi}{3}}\right)$
        \item $\sqrt{2}\left(\pc{\frac{3\pi}{4}}\right)$
        \item $\frac{1}{\sqrt{3}}\left[\pcx{-\frac{\pi}{6}}\right]$
        \item $5\left(\pc{\frac{7\pi}{2}}\right)$
    \end{enumerate}
\end{multicols}

\item 想一想:“两个复数若相等,则它们的模与幅角必定相等”对吗?为什么?
\end{enumerate}
\end{ex}

\subsection{再谈复数的乘、除法和乘方运算}
通过第二节对复数四则运算的学习,我们会感到:复数的代数形式在进行加、减法运算时,步骤十分简明易学,但进行乘、除法和乘方运算时,却显得繁杂.

我们将会看到,复数的三角形式在进行乘、除、乘方甚至开方运算时,比代数形式要简捷得多.

\begin{blk}
 {定理1} 两个复数的乘积是一个复数,它的模等于两乘数的模的乘积;它的幅角等于两乘数的幅角的和,即   

设$z_1=r_1(\pc{\theta_1})$, $z_2=r_2(\pc{\theta_2})$,则
\[\begin{split}
z_1\cdot z_2 &=r_1(\pc{\theta_1})\cdot r_2(\pc{\theta_2})\\
&=r_1\cdot r_2 \left[\pcx{\theta_1+\theta_2}\right]
\end{split}\]
\end{blk}

\begin{proof}
\[\begin{split}
    \because\quad z_1\cdot z_2&=r_1\cdot r_2(\pc{\theta_1})(\pc{\theta_2})\\
&=r_1\cdot r_2 \left[\left(\cos\theta_1\cdot \cos\theta_2-\sin\theta_1\cdot \sin\theta_2\right)\right.\\
&\qquad \qquad +i\left.\left(\cos\theta_1\cdot \sin\theta_2+\sin\theta_1\cdot \cos\theta_2\right)\right]\\
&=r_1\cdot r_2 \left[\pcx{\theta_1+\theta_2}\right]
\end{split}\]

$\therefore\quad $定理成立.
\end{proof}

根据这一定理,联系复数的向量表示,我们可以给出两复数相乘的几何意义:

两复数$z_1=r_1(\pc{\theta_1})$与$z_2=r_2(\pc{\theta_2})$相乘时,可以先在复平
面上画出它们对应的向量$\VEC{OZ_1}$, $\VEC{OZ_2}$,然后将$\VEC{OZ_1}$旋转一个角$|\theta_2|$(若$\theta_2>0$时,就按逆时针方向旋转$\VEC{OZ_1}$;若$\theta_2<0$时,就按顺时针方向旋转$\VEC{OZ_1}$),再把它的模变为原来的$r_2$倍,就可得到向量$\VEC{OZ}$(图1.9所示),这一向量就对应着所求两复数的积.

\begin{figure}[htp]
    \centering
\begin{tikzpicture}[>=stealth, scale=.7]
\draw[->](-1,0)--(5,0)node[right]{$x$};
\draw[->](0,-1)--(0,7)node[right]{$y$};
\node[below left]{$O$};
\coordinate(Z1) at (1.5,1.5);
\coordinate(Z2) at (2.5,1.8);
\coordinate(Z) at (1.05,6.45);
\draw[->, very thick](0,0)--node[above]{$r_1$}(Z1)node[above]{$Z_1$};
\draw[->, very thick](0,0)--node[below right]{$r_2$}(Z2)node[right]{$Z_2$};
\draw[->, very thick](0,0)--node[above right]{$r_1\cdot r_2$}(Z)node[above]{$Z$};
\draw[->](.75,0)node[above right]{$\theta_2$}arc (0:36:.75);
\draw[->](1.75,0)node[above right]{$\theta_1$}arc (0:45:1.75);
\draw[->](2.75,0)arc (0:80:2.75)node[right]{$\theta_1+\theta_2$};
\end{tikzpicture}
    \caption{}
\end{figure}



\begin{blk}
{推论} $n$个复数的积仍是一个复数,它的模等于$n$个乘数
的模的积,它的幅角等于$n$个乘数的幅角的和。(可用数学
归纳法证明)即:
\[\begin{split}
 z_{1}\cdot z_{2}\cdots z_{n}&=r_{1}\left(\cos\theta_{1}+i\sin\theta_{1}\right)\cdot r_{2}\left(\cos\theta_{2}+   i\sin\theta_{2}\right)\\
 &\qquad \qquad \cdots r_n(\pc{\theta_n})\\
 &=r_{1}\cdot r_{2}\cdots  r_{n}[\cos(\theta_1+\theta_2+\cdots+\theta_n)\\
 &\qquad \qquad \qquad +i\sin (\theta_1+\theta_2+\cdots+\theta_n)] 
\end{split}\]
\end{blk}

这样,在计算复数的乘积时,只要先化为三角形式。再求出模之积与幅角之和,就可写出所求乘积的三角形式。

在上边的推论中,如果$z_1=z_2=\cdots=z_n=z$, 即$r_1=r_{2}= \cdots = r_{n}= r$ 且 $\theta _{1}= \theta _{2}= \cdots = \theta _{n}= \theta $. 也就是说有$n$个相同的复数相乘时,那么就得到:
\[\left[r(\pc{\theta})\right]^n=r^n(\pc{n\theta})\quad (n\in\N)\]
这就是著名的棣美佛定理.

\begin{blk}
{定理2(棣美佛定理)}复数的$n$($n\in\N$)次方幂仍是一个复数,它的模等于这个复数的模的$n$次幂,它的幅角等于这个复数的幅角的$n$倍.    
\end{blk}

\begin{example}
    计算:$\left(\pc{\frac{\pi}{6}}\right)\cdot \sqrt{2}\left(\pc{\frac{\pi}{12}}\right)$
\end{example}

\begin{solution}
由乘法定理可知:
\[\begin{split}
\text{原式}&=\sqrt{2}\left[\pcx{\frac{\pi}{6}+\frac{\pi}{12}}\right]\\
&=\sqrt{2}\left(\pc{\frac{\pi}{4}}\right)\\
&=\sqrt{2}\left(\frac{\sqrt{2}}{2}+\frac{\sqrt{2}}{2}i\right)=1+i
\end{split}\]
\end{solution}

\begin{example}
    计算$(1-i)^{10}$
\end{example}

\begin{solution}
    先将$1-i$化为三角形式:
\[1-i=\sqrt{2}\left[\pcx{-\frac{\pi}{4}}\right]\]
由棣美佛定理,得
\[\begin{split}
    (1-i)^{10}&=\left[\sqrt{2}\left(\pc{\frac{-\pi}{4}}\right)\right]^{10}\\
    &=\left(\sqrt{2}\right)^{10}\cdot \left[\pcx{-\frac{5\pi}{2}}\right]\\
    &=2^5\left(\cos\frac{\pi}{2}-i\sin\frac{\pi}{2}\right)=-32i
\end{split}\]
\end{solution}

\begin{example}
    如果将复数$z=2+3i$在复平面上所对应的向量$\VEC{OZ}$旋转$150^{\circ}$,那么,在复平面上所得新向量$\VEC{OZ_1}$将对应哪一个复数(用代数形式表示).
\end{example}

\begin{solution}
    由复数乘法的几何意义知道,一个向量(对应一个复数)旋转一个角度$\theta<0$,就相当于这个复数与模长为1,幅角主值为$\theta$的另一复数(对应于幅角为$\theta$的单位向量)作乘积.

因此,所求的复数就等于复数$3i$与复数$1\cdot(\cos150^{\circ}+i\sin 150^{\circ})$的乘积,即
\[\begin{split}
    (2+3i)\cdot (\cos150^{\circ}+i\sin 150^{\circ})
    &=(2+3i)\left(-\frac{\sqrt{3}}{2}+\frac{1}{2}i\right)\\
&=-\frac{2\sqrt{3}+3}{2}+\frac{2-3\sqrt{3}}{2}i
\end{split}\]
\end{solution}

\begin{blk}
{定理3} 两个复数的商,在除数不为零的条件下仍是一个复数,它的模等于两个复数的模的商,它的幅角等于两复数的幅角的差(被除数的幅角减去除数的幅角).即:当$z_2\ne 0$时,有
\[\begin{split}
    \frac{z_1}{z_2}&=\frac{r_1(\pc{\theta_1})}{r_2(\pc{\theta_2})}\\
    &=\frac{r_1}{r_2}\left[\pcx{\theta_1-\theta_2}\right]
\end{split}\]

\end{blk}

\begin{proof}
不妨设$z=\frac{r_1}{r_2}\left[\pcx{\theta_1-\theta_2}\right]$,则由乘法,
\[\begin{split}
    z\cdot z_2 &=\frac{r_1}{r_2}\cdot r_2\left[\pcx{\theta_1-\theta_2+\theta_2}\right]\\
    &=r_1(\pc{\theta_1})=z_1
\end{split} \]

又由除法的定义,可得
\[\frac{z_1}{z_2}=z\quad (z_2\ne 0)\]
\end{proof}

\begin{example}
计算$\left(\sqrt{3}-i\right)\div \left(\sqrt{3}+i\right)$
\end{example}

\begin{solution}
$\because\quad \sqrt{3}-i=2\left(\frac{\sqrt{3}}{2}-\frac{1}{2}i\right)=2\left[\pcx{-\frac{\pi}{6}}\right]$

$\sqrt{3}+i=2\left(\pc{\frac{\pi}{6}}\right)$

\[\begin{split}
    \therefore\quad \left(\sqrt{3}-i\right)\div \left(\sqrt{3}+i\right)&=(2\div 2)\left[\pcx{-\frac{\pi}{6}-\frac{\pi}{6}}\right]\\
&=1\cdot \left[\pcx{-\frac{\pi}{3}}\right]\\
&=\frac{1}{2}-\frac{\sqrt{3}}{2}i
\end{split}\]
\end{solution}

\begin{example}
    已知平面内并列的三个边长为1的正方形(如图1.10所示),利用复数证明:
$\angle 1+\angle 2-\angle 3=0$.
\end{example}

\begin{figure}[htp]
    \centering
\begin{tikzpicture}[>=stealth, scale=1.8]
\draw[->](-.5,0)--(4,0)node[right]{$x$};    
\draw[->](0,-.5)--(0,2)node[right]{$y$};    
\foreach \x/\y in {1/3,2/2,3/1}
{
    \node at (\x,0)[below]{$\x$};
    \draw(\x,0)--(\x,1)node[above]{$Z_{\y}$}--(0,0);
}
\draw(0,1)node[left]{1}--(3,1);
\node[below left]{$O$};

\draw(1-.25,1) arc (180:180+45:.25)node[left]{$3$};
\draw(2-.25,1) arc (180:180+26:.25)node[left]{$2$};
\draw(3-.5,1) arc (180:180+19:.5)node[left]{$1$};

\draw(.25,0)arc (0:45:.25);
\draw(.35,0)arc (0:26:.35);
\draw(.5,0)arc (0:19:.5);


\end{tikzpicture}
    \caption{}
\end{figure}

\begin{solution}
 如图建立坐标系以确定复平面,不在轴上的三个正
方形顶点设为$Z_1,Z_2,Z_3$.显然,它们分别对应着复数$3+i$, $2+i$, $1+i$; 由于平行线的内错角相等,因而$\angle 1$、$\angle 2$、$\angle 3$ 分别等于复数$3+i$, $2+i$,$1+i$的幅角的主值。

又由复数的乘、除法可知,$\angle 1+\angle 2-\angle 3$就是复数$\frac{(3+i)(2+i)}{1+i}$的幅角,而
\[\frac{(3+i)(2+i)}{1+i}=\frac{5+5i}{1+i}=5\]
它的幅角的主值是0,所以$\angle 1+\angle 2-\angle 3=0$.
\end{solution}

\begin{ex}
\begin{enumerate}
    \item 计算
\begin{enumerate}[(1)]
    \item $\sqrt{2}\left(\pc{\frac{\pi}{6}}\right)\cdot \frac{\sqrt{2}}{2}\left(\pc{\frac{\pi}{4}}\right)$
    \item $2\left(\pc{\frac{4\pi}{3}}\right)\cdot 3\left(\pc{\frac{5\pi}{6}}\right)$
    \item $\sqrt{3}(\pc{240^{\circ}})\cdot \frac{\sqrt{3}}{3}(\pc{150^{\circ}})$
    \item $4(\pc{18^{\circ}})\cdot [\pcx{-306^{\circ}}]\cdot 5(\pc{108^{\circ}})$
\end{enumerate}
\item 用棣美佛定理计算:
\begin{enumerate}[(1)]
    \item $\left[2\sqrt{2}(\pc{18^{\circ}})\right]^5$
    \item $(-1-i)^6$
    \item $\left[\sqrt{3}\left(\pc{\frac{3\pi}{4}}\right)\right]^4$
    \item $(2-i)\cdot \left(-\frac{1}{2}+\frac{\sqrt{3}}{2}i\right)^7$
\end{enumerate}
\item 计算:
\begin{enumerate}[(1)]
    \item $12\left(\pc{\frac{7\pi}{4}}\right)\div 6\left(\pc{\frac{2\pi}{3}}\right)$
    \item $2\div (\pc{45^{\circ}})$
    \item $-i\div 2\left(\pc{\frac{4\pi}{3}}\right)$
    \item $(-i+i)\cdot \sqrt{2}(\pc{120^{\circ}})\div 3(\pc{55^{\circ}})$
\end{enumerate}

\item 你能说明复数除法的几何意义吗?
\item 利用复数证明:在${\rm Rt}\triangle ABC$中,
$\angle C=90^{\circ}$, $AC=3BC$,$E$为$AC$上一点,且$CE=2EA$,则:
\[\angle CBE+\angle CBA-\frac{\pi}{4}=\frac{\pi}{2}\]
\end{enumerate}
\end{ex}

\subsection{复数的开方运算}
开方是乘方的逆运算,求一个复数$a+bi$的$n$次方根,就是求一个数$z$,使得$z^n=a+bi$. 凡满足这一等式的数$z$,就叫做复数$a+bi$的$n$次方根.

设$a+bi=r(\cos\theta+i\sin\theta)$, $z=\rho(\cos\varphi+i\sin\varphi)$,
则对于任意自然数$n$,都有
\[r(\cos\theta+i\sin\theta)=[\rho(\cos\varphi+i\sin\varphi)]^n=\rho^n(\cos n\varphi+i\sin n\varphi)\]

由于两复数相等,它们的模相等,而它们的幅角可以相差$2\pi$的整数倍,所以
\[\begin{cases}
    \rho^n=r\\
    n\varphi=\theta+2k\pi\quad (k\in\Z)
\end{cases}\]

因而可得
\[\rho=\sqrt[n]{r},\qquad \varphi=\frac{\theta+2k\pi}{n}\]

因此,复数$r(\pc{\theta})$的$n$次方根是
\[\sqrt[n]{r}\left(\pc{\frac{2k\pi+\theta}{n}}\right)\]
其中,当$k=0,1,\ldots,n-1$各值时,可以得到上式的$n$个不同值;当$k$继续取$n,n+1,\ldots$以及其它的值时,由于正弦、余弦函数都是以$2\pi$为周期,上式的值又重复出现$k$取$0,1,\ldots,n-1$时的同样结果.所以,复数的$n$次($n\in\N$)方根是$n$个复数,它们的模都等于这个复数的模的$n$次算术根,它们的幅角分别等于这个复数的幅角与$2\pi$的$0,1,\ldots,n-1$倍的和的$n$分之一. 即
\[\sqrt[n]{r(\pc{\theta})}=\sqrt[n]{r}\left(\pc{\frac{\theta+2k\pi}{n}}\right)\quad (k=0,1,\ldots,n-1)\]

\begin{rmk}
在复数范围内,$\sqrt[n]{z}$应表示$n$个复数,这$n$个复数都是$z$的$n$次方根,为了区别于在实数范围内,$\sqrt[n]{z}$仅表示算术根,我们约定:被开方数是实数时,$\sqrt[n]{z}\; (a\in\R)$仍表示算术根(单值),$\sqrt[n]{a+0i}$或$\sqrt[n]{a(\pc{0})}\; (a>0)$或$\sqrt[n]{|a|(\pc{\pi})}$都表示$n$个$n$次方根(多值). 至于被开方数是虚数时,$\sqrt[n]{a+bi}\; (b\ne 0)$或$\sqrt[n]{r(\pc{\theta})}$都表示$n$个方根.
\end{rmk}

\begin{example}
    求$i$的平方根.
\end{example}

\begin{solution}
$\because\quad i=\pc{\frac{\pi}{2}}$

$\therefore\quad \sqrt{i}=\pc{\frac{\frac{\pi}{2}+2k\pi}{2}}=\pcx{\frac{\pi}{4}+k\pi}\quad (k=0,1)$

因此,$i$的平方根是
\[\begin{split}
z_1=\pc{\frac{\pi}{4}}=\frac{\sqrt{2}}{2}+\frac{\sqrt{2}}{2}i\qquad &\text{($k=0$时)}\\
z_2=\pc{\frac{5\pi}{4}}=-\frac{\sqrt{2}}{2}-\frac{\sqrt{2}}{2}i\qquad &\text{($k=1$时)}
\end{split}\]
\end{solution}

\begin{example}
    求$-1+i$的四次方根.
\end{example}

\begin{solution}
$\because\quad -1+i=\sqrt{2}\left(\pc{\frac{3\pi}{4}}\right)$

$\therefore\quad \sqrt[4]{-1+i}=\sqrt[8]{2}\left(\pc{\frac{\frac{3\pi}{4}+2k\pi}{4}}\right)\quad (k=0,1,2,3)$

因此,$-1+i$的四次方根是:
\[\begin{split}
z_1=\sqrt[8]{2}\left(\pc{\frac{3\pi}{16}}\right)\qquad & (k=0)\\
z_2=\sqrt[8]{2}\left(\pc{\frac{11\pi}{16}}\right)\qquad & (k=1)\\
z_3=\sqrt[8]{2}\left(\pc{\frac{19\pi}{16}}\right)\qquad & (k=2)\\
z_4=\sqrt[8]{2}\left(\pc{\frac{27\pi}{16}}\right)\qquad & (k=3)\\
\end{split}\]
\end{solution}

\begin{example}
    求$-3$的平方根.
\end{example}

\begin{solution}
$\because\quad -3=3(\pc{\pi})$

$\therefore\quad \sqrt{-3+0i}=\sqrt{3}\left(\pc{\frac{\pi+2k\pi}{2}}\right)\quad (k=0,1)$

因此,$-3$的平方根有两个复数:
\[\begin{split}
z_1&=\sqrt{3}\left(\pc{\frac{\pi}{2}}\right)=\sqrt{3}i\\
z_2&=\sqrt{3}\left(\pc{\frac{3\pi}{2}}\right)=-\sqrt{3}i\\
\end{split}\]
\end{solution}

一般来说,在复数范围内,任一个负实数$-a$有两个平方根:$\pm\sqrt{a}i$.

这样一来,以前在解实系数一元二次方程$ax^2+bx+c=0$时,如果$b^2-4ac<0$,在实数范围内无解. 现在,在复数范围内,方程就有两个解:
\[x=\frac{-b\pm\sqrt{-(b^2-4ac)}i}{2a}\qquad (b^2-4ac<0)\]
显然,这两个解是一对共轭复数.

\begin{example}
    解方程$x^2-2x+6=0$.
\end{example}

\begin{solution}
$\because\quad \Delta=b^2-4ac=4-24=-20<0$

$\therefore\quad x=\frac{2\pm\sqrt{20}i}{2}=1\pm\sqrt{5}i$
\end{solution}

由以上可知,复数集$\mathbb{C}$对于开方运算是封闭的,这是复数系不同于实数系的一个重要性质.

利用复数开方的方法,我们可以解一些特殊的高次方程,例如,

形如$a_nx^n+a_0=0\; (a_n\ne 0)$的方程,我们叫做二项方程,就可以完整地解出,即,首先化成$x^n=b\quad \left(b=-\frac{a_0}{a_n}\right)$的形式,从而解出$x$的$n$个值.

\begin{example}
    解方程$x^5+243=0$.
\end{example}

\begin{solution}
$\because\quad x^5=-243=243(-1+0i)=243(\pc{\pi})$

$\therefore\quad x=\sqrt[5]{243}\left(\pc{\frac{\pi+2k\pi}{5}}\right)\quad (k=0,1,2,3,4)$

即
\[\begin{split}
x_1&=3\left(\pc{\frac{\pi}{5}}\right)\qquad\quad (k=0)\\
x_2&=3\left(\pc{\frac{3\pi}{5}}\right)\qquad (k=1)\\
x_3&=3\left(\pc{\pi}\right)\qquad\qquad (k=2)\\
x_4&=3\left(\pc{\frac{7\pi}{5}}\right)\qquad (k=3)\\
x_5&=3\left(\pc{\frac{9\pi}{5}}\right)\qquad (k=4)\\
\end{split}\]
\end{solution}

仔细观察这五个解可以发现:它们的模都相同,它们的幅角依次相差$\frac{2\pi}{5}$:因此,这个方程的五个解所对应的复平面内的五个点是均匀地分布在以原点为圆心,以3为半径的圆周上(图1.11).这就是此方程的解的几何意义。
\begin{figure}[htp]
    \centering
\begin{tikzpicture}[>=stealth]
\draw[->](-2.5,0)--(2.7,0)node[right]{$x$};
\draw[->](0,-2)--(0,2.7)node[right]{$y$};
\draw(0,0)node[below left]{$O$} circle(1.5);
\coordinate (Z1) at (36:1.5);
\coordinate (Z2) at (36*3:1.5);
\coordinate (Z3) at (36*5:1.5);
\coordinate (Z4) at (36*7:1.5);
\coordinate (Z5) at (36*9:1.5);
\draw[very thick](0,0)--(Z1)node[right]{$Z_1$};
\draw[very thick](0,0)--(Z2)node[above]{$Z_2$};
\draw[very thick](0,0)--(Z3)node[above left]{$Z_3$};
\draw[very thick](0,0)--(Z4)node[below]{$Z_4$};
\draw[very thick](0,0)--(Z5)node[right]{$Z_5$};
\node at (-1.5,0)[below left]{$-3$};
\foreach \x in {1,2,3,4,5}
{
    \draw[fill](Z\x) circle(1.5pt);
}


\end{tikzpicture}
    \caption{}
\end{figure}

一般地,二项方程$x^n=b\; (b\in\mathbb{C})$有$n$个复数根,这些根的几何意义是复平面内的$n$个点(或向量),这些点(或向量的终点)均匀地分布在以原点为圆心,以$\sqrt[n]{|b|}$为半径的圆上.

还要进一步指出:对一般的实系数$n$次方程在复数集内也是恰有$n$个根,但未必都是实数根,可能会有虚根(如例1.25).

可以证明,实系数方程的虚根总是成对地出现的,即

\begin{blk}{定理4(虚根成对定理)}
如果$z=a+bi$是实系数方程
$f(x)=a_nx^n+a_{n-1}x^{n-1}+\cdots+a_1x+a_0=0$
的一个根,那么$z=a-bi$也是方程$f(x)=0$的一个根.
\end{blk}

\begin{proof}
    由于$z=a+bi$是方程的根,所以
\[f(z)=f(a+bi)=0\]
但根据习题1 2第15题(2)已证明的结论,应有
\[f(\bar z)=f(a-bi)=f(\overline{a+bi})=\bar 0=0\]
因此,$\bar z=a-bi$也是方程$f(x)=0$的根.
\end{proof}

\begin{example}
    已知方程
$2x^4-x^3+5x^2+13x+5=0$
的一个根为1,求这个方程的其余根.
\end{example}

\begin{solution}
    这是一个实系数一元四次方程,由定理4及已知一根$1-2i$可知,这一方程必有根$1+2i$.

又由因式定理知:$(x-1+2i)(x-1-2i)=x^2-2x+5$是方程左边的因式,利用长除法可得
\[2x^4-x^3+5x^2+13x+5=(2x^2+3x+1)(x^2-2x+5)\]

所以只要解方程$2x^2+3x+1=0$的根:$x_1=-1$, $x_2=-\frac{1}{2}$,就可知:

原方程的其余三根为$1+2i,\; -1,\; -\frac{1}{2}$.
\end{solution}

\begin{ex}
\begin{enumerate}
    \item 直接说出以下各数的平方根:
\[-4,\quad -2.25,\quad -3,\quad -t\; (t\in\R^+),\quad -m^2\; (m\in\R)\]
\[a\; (a\in\R),\quad a-b\; (a,b\in\R)\]
    \item 求下列各数的方根:
\begin{enumerate}[(1)]
    \item $-i,\; \sqrt{2}-\sqrt{2}i,\; -\frac{1}{2}+\frac{\sqrt{3}}{2}i$的平方根;
    \item $27,\; 1+i$的立方根;
    \item $-i$的五次方根.
\end{enumerate}
    \item 解方程(在复数范围内):
\begin{multicols}{2}
\begin{enumerate}[(1)]
    \item $x^2+x+16=0$
    \item $x^3-1=0$
    \item $x^4+16=0$
    \item $x^5-i=0$
\end{enumerate}
\end{multicols}
    \item 已知方程$4x^4-2x^3-x-1=0$有一个根为$-\frac{i}{\sqrt{2}}$,试求这个方程的其余三个根.
\end{enumerate}
\end{ex}


\subsection{单位根}
我们把方程$x^n=1$在复数范围内的$n$个根,叫做$n$次单位根.也就是在复数范围内,1的$n$个不同的$n$次方根,都叫$n$次单位根. 即

$\because\quad x^n=1=\pc{0}$

$\therefore\quad x=1\cdot \left(\pc{\frac{2k\pi}{n}}\right)\quad (k=0,1,\ldots,n-1)$

这样可得出$n$个单位根:
\[\begin{split}
x_1&=\pc{0}=1\qquad \qquad \text{(当$k=0$时)}\\
x_2&=\pc{\frac{2\pi}{n}}\qquad \qquad \text{(当$k=1$时)}\\
x_3&=\pc{\frac{4\pi}{n}}\qquad \qquad \text{(当$k=2$时)}\\
\cdots&\cdots\cdots\\ 
x_n&=\pc{\frac{2(n-1)\pi}{n}}\qquad \text{(当$k=n-1$时)}\\
\end{split}\]

由二项方程解的几何意义可以知道:$n$个不同的单位根可用复平面内以原点为圆心的单位圆的一个内接正$n$边形顶点来表示.

\begin{example}
    求三次单位根.
\end{example}

\begin{solution}
$\because\quad x^3=1=\pc{0}$

$\therefore\quad x=\pc{\frac{2k\pi}{3}}\quad (k=0,1,2)$

因此,三个三次单位根为(如图1.12):
\[\begin{split}
    x_1&=\pc{0}=1\\
    x_2&=\pc{\frac{2\pi}{3}}=-\frac{1}{2}+\frac{\sqrt{3}}{2}i\\
    x_3&=\pc{\frac{4\pi}{3}}=-\frac{1}{2}-\frac{\sqrt{3}}{2}i\\
\end{split}\]
\end{solution}

\begin{minipage}{.45\textwidth}
\centering
\begin{tikzpicture}[>=stealth]
\draw[->](-2.5,0)--(2.7,0)node[right]{$x$};
\draw[->](0,-2)--(0,2.7)node[right]{$y$};
\draw(0,0)node[below left]{$O$} circle(1.5);
\draw[very thick](1.5,0)node[above right]{$x_1$}--(120:1.5)node[above]{$x_2$}--(240:1.5)node[below]{$x_3$}--cycle;

\end{tikzpicture}
\captionof{figure}{ }
\end{minipage}\hfill
\begin{minipage}{.45\textwidth}
\centering
\begin{tikzpicture}[>=stealth]
    \draw[->](-2.5,0)--(2.7,0)node[right]{$x$};
    \draw[->](0,-2)--(0,2.7)node[right]{$y$};
    \draw(0,0)node[below left]{$O$} circle(1.5);
\draw[very thick](1.5,0)node[above right]{$x_1$}--(0,1.5)node[above right]{$x_2$}--(-1.5,0)node[above left]{$x_3$}--(0,-1.5)node[below left]{$x_4$}--cycle;

\end{tikzpicture}
\captionof{figure}{ }
\end{minipage}

\begin{example}
    求四次单位根
\end{example}

\begin{solution}
$\because\quad x^4=1$

$\therefore\quad x=\pc{\frac{2k\pi}{4}}\quad (k=0,1,2,3)$

因此,四个四次单位根为(如图1.13):
\begin{center}
\begin{tabular}{p{.2\textwidth}p{.15\textwidth}p{.2\textwidth}p{.15\textwidth}}
    $x_1=1$& $(k=0)$ & $x_2=i$ & $(k=1)$\\
    $x_3=-1$ &$ (k=2)$& $x_4=-i$ & $(k=3)$\\
\end{tabular} 
\end{center}
\end{solution}


如果用复数的向量表示法
来看这$n$个不同的$n$次单位根的话,我们会发现单位根之间还有一定的关系.

\begin{figure}[htp]
    \centering
\begin{tikzpicture}[>=stealth]
    \draw[->](-2.5,0)--(2.7,0)node[right]{$x$};
    \draw[->](0,-2)node[below]{$(n=12)$}--(0,2.7)node[right]{$y$};
    \draw(0,0)node[below left]{$O$} circle(1.5);
\foreach \x in {0,1,2,...,11}
{
    \draw[fill](\x*30:1.5) circle(1.5pt);
}
\node at (1.5,0)[below right]{1};
\node at (0:1.5)[above right]{$x_1$};
\node at (30:1.5)[above right]{$x_2$};
\node at (60:1.5)[above right]{$x_3$};
\node at (-30:1.5)[below right]{$x_n$};


\end{tikzpicture}
    \caption{}
\end{figure}




\begin{enumerate}
\item 如图1.14所示,在复平面上$n$个$n$次单位根正好将单位圆$n$等分,其中:
$x_1=1$,单位向量$\VEC{Ox_1}$旋转$\frac{2\pi}{n}$
得:$x_2=\pc{\frac{2\pi}{n}}$
记为$\omega$;再旋转$\frac{2\pi}{n}$
得:$x_3=\pc{\frac{4\pi}{n}}=\omega^2$;再旋转$\frac{2\pi}{n}$可一样地
得:$x_4=\pc{\frac{6\pi}{n}}=\omega^3$,……

同理可得:
\[x_n=\pc{\frac{2(n-1)\pi}{n}}=\omega^{n-1}\]

因此,设$\omega=\pc{\frac{2\pi}{n}}$,则$n$个$n$次单位根又可写成:
\[1,\; \omega,\; \omega^2,\; \ldots, \omega^{n-1}\]

\item 进一步考虑这$n$个单位根的关系,可以发现:

\[\begin{split}
    \because\quad \omega^{n-1}=\pc{\frac{2(n-1)\pi}{n}}&=\pcx{\frac{-2\pi}{n}}\\
    &=\cos\frac{2\pi}{n}-i\sin\frac{2\pi}{n}=\overline{\omega}
\end{split}\]

$\therefore\quad \omega^{n-1}=\overline{\omega}=\frac{1}{\omega}=\omega^{-1}$

同理可得:\[\omega^{n-2}=\overline{\omega^2}=\omega^{-2},\ldots\]

\[\begin{cases}
\text{当$n$为偶数时}& \omega^{\tfrac{n}{2}+1}=\overline{\omega^{\tfrac{n}{2}-1}}=\omega^{-\left(\tfrac{n}{2}-1\right)}=\omega^{-\tfrac{n}{2}+1}\\
\text{当$n$为奇数时}& \omega^{\tfrac{n+1}{2}}=\overline{\omega^{\tfrac{n-1}{2}}}=\omega^{-\tfrac{n-1}{2}}
\end{cases}\]
所以,这$n$个$n$次单位根还可以写成:
\begin{itemize}
    \item 当$n$为偶数时(如例1.29):
\[1,\; \omega,\; \omega^2,\;\ldots,\;  \omega^{\tfrac{n}{2}-1},\;-1,\;  \omega^{-\left(\tfrac{n}{2}-1\right)},\; \ldots, \;  \omega^{-2},\; \omega^{-1}\]
或
\[1,\; \omega,\; \omega^2,\;\ldots,\;  \omega^{\tfrac{n}{2}-1},\;-1,\;  \overline{\omega^{\tfrac{n}{2}-1}},\; \ldots, \;  \overline{\omega^{2}},\; \overline{\omega}\]

\item 当$n$为奇数时(如例1.28):
\[1,\; \omega,\; \omega^2,\;\ldots,\;  \omega^{\tfrac{n-1}{2}},\;  \omega^{-\tfrac{n-1}{2}},\; \ldots, \;  \omega^{-2},\; \omega^{-1}\]
或
\[1,\; \omega,\; \omega^2,\;\ldots,\;  \omega^{\tfrac{n-1}{2}},\;  \overline{\omega^{\tfrac{n-1}{2}}},\; \ldots, \;  \overline{\omega^{2}},\; \overline{\omega}\]
其中,$\omega=\pc{\frac{2\pi}{n}}$.
\end{itemize}

例如,三次单位根为$1,\; \omega,\; \omega^{-1}$(或$\bar\omega$),其中
\[\omega=\pc{\frac{2\pi}{3}}=-\frac{1}{2}+\frac{\sqrt{3}}{2}i\]

又如,四次单位根为$1,\; \omega,\; -1,\; \omega^{-1}$(或$\bar\omega$),其中
\[\omega=\pc{\frac{2\pi}{4}}=i\]

\item 利用单位根,可以方便地求出任一个正实数$ca>0$且$c\ne 1$的$n$个$n$次方根.

这就是:若$a>0$且$a\ne 1$,则$a$的$n$个$n$次方根即方程$x^n=a$的解为:
\[\begin{cases}
x_1=\sqrt[n]{a}\\
x_2=\sqrt[n]{a}\cdot \omega\\
x_3=\sqrt[n]{a}\cdot \omega^2\\
\cdots\cdots\cdots\\
x_n=\sqrt[n]{a}\cdot \omega^{n-1}\\
\end{cases}\qquad \text{这里$\omega=\pc{\frac{2\pi}{n}}$是单位根之一}\]
\end{enumerate}

\begin{ex}
试求出六次和七次单位根,并分别检验它们之间的关系.
\end{ex}

\section*{习题1.3}
\begin{enumerate}
    \item 把下列复数化成三角形式表示:
\begin{multicols}{3}
\begin{enumerate}[(1)]
    \item 6
    \item $6i$
    \item $1+i$
    \item $\frac{1}{2}-\frac{\sqrt{3}}{2}i$
    \item $-\sqrt{2}-\sqrt{2}i$
\end{enumerate}
\end{multicols}

\item 把下列复数化为代数形式表示,并画出复平面上相应的向量来:
\begin{multicols}{2}
\begin{enumerate}[(1)]
    \item $3\sqrt{2}\left(\pc{\frac{\pi}{4}}\right)$
    \item $2\left(\pc{\frac{11\pi}{6}}\right)$
    \item $\pc{\pi}$
    \item $\sqrt{3}\left[\pcx{-\frac{2\pi}{3}}\right]$
    \item $\pcx{k\cdot \frac{\pi}{4}}$
    
    $(k=0,1,2,\ldots,6,7)$
\end{enumerate}
\end{multicols}

\item 计算
\begin{enumerate}[(1)]
    \item $\sqrt{3}\left(\pc{\frac{\pi}{3}}\right)\cdot 2\sqrt{3}\left(\pc{\frac{\pi}{6}}\right)$
    \item $\sqrt{10}\left(\pc{\frac{\pi}{2}}\right)\cdot \sqrt{2}\left(\pc{\frac{\pi}{4}}\right)$
\end{enumerate}

\item 下列复数是不是三角形式?如果不是,请化成三角形式表示:
\begin{multicols}{2}
\begin{enumerate}[(1)]
    \item $\cos\frac{\pi}{6}-i\sin\frac{\pi}{6}$
    \item $\sqrt{2}\left(\cos\frac{4\pi}{3}+i\sin\frac{2\pi}{3}\right)$
    \item $7\left(\sin\frac{3\pi}{4}+i\cos\frac{3\pi}{4}\right)$
    \item $3\left(\cos\frac{5\pi}{6}+i\sin\frac{-7\pi}{6}\right)$
\end{enumerate}
\end{multicols}

\item 求证:
\begin{enumerate}[(1)]
    \item $\sqrt{2}(\pc{75^{\circ}})\cdot \sqrt{2}\left(\pc{15^{\circ}}\right)=2i$
    \item $(\cos3\theta-i\sin3\theta)(\cos2\theta-i\sin2\theta)=\cos5\theta-i\sin5\theta$
\end{enumerate}

\item 如图1.15所示,菱形$OABC$的一个内角$\angle AOC=\frac{\pi}{4}$,且$A$点所对应的复数为$2+i$, 试求,$B$、$C$两点各对应的复数.

\begin{figure}[htp]
    \centering
\begin{tikzpicture}[>=stealth]
    \draw[->](-.75,0)--(3.4,0)node[right]{$x$};
    \draw[->](0,-.75)--(0,3.4)node[right]{$y$};
    \node[below left]{$O$} ;
    \coordinate (A) at (30:2);
    \coordinate (C) at (75:2);
    \coordinate (B) at (105/2:3.7);
    \draw[very thick](0,0)--(A)node[right]{$A$}--(B)node[above]{$B$}--(C)node[above]{$C$}--cycle;
\end{tikzpicture}
    \caption{}
\end{figure}


\item 已知下列各复数$z$,求$\left|\frac{1}{z}\right|$及$\arg\left(\frac{1}{z}\right)$
\begin{multicols}{2}
\begin{enumerate}[(1)]
    \item $z=2\left(\pc{\frac{\pi}{12}}\right)$
    \item $z=\cos\frac{\pi}{6}-i\sin\frac{\pi}{6}$
    \item $z=\frac{1-i}{\sqrt{2}}$
\end{enumerate}    
\end{multicols}

\item 把复数$3-\sqrt{3}i$对应的向量旋转$\frac{\pi}{3}$,所得向量相对应的复数是什么?

\item 用棣美佛定理计算:
\begin{multicols}{2}
\begin{enumerate}[(1)]
    \item $[2(\pc{15^{\circ}})]^6$
    \item $\left[\frac{1}{\sqrt{2}}(\pc{225^{\circ}})\right]^4$
    \item $(1+\sqrt{3}i)^4$
    \item $(2-2\sqrt{3}i)^4$
    \item $\frac{(\sqrt{3}+i)^5}{-1+\sqrt{3}i}$
    \item $\left(\frac{2+2i}{1-\sqrt{3}i}\right)^8$
\end{enumerate}
\end{multicols}

\item 用棣美佛定理证明:
\begin{enumerate}[(1)]
    \item $\cos2\theta=\cos^2\theta-\sin^2\theta,\qquad \sin2\theta=2\sin\theta\cdot \cos\theta$
    \item $\cos3\theta=4\cos^3\theta-3\cos\theta,\qquad \sin3\theta=2\sin^2\theta-4\sin\theta$
\end{enumerate}

\item 设$n\in\N$,$n$取何值时,$(1+\sqrt{3})^n$是一个实数?

\item 计算:
\begin{enumerate}[(1)]
    \item $10\left(\pc{\frac{2\pi}{3}}\right)\div 2\left(\pc{\frac{\pi}{6}}\right)$
    \item $-10i\div 6\left(\pc{\frac{\pi}{6}}\right)$
\end{enumerate}

\item 求证:
\begin{enumerate}[(1)]
    \item $\frac{1}{\cos\theta+i\sin\theta}=\cos\theta-i\sin\theta$
    \item $(\pc{\theta})^{-n}=\pcx{-n\theta}\quad (n\in\mathbb{N})$
    \item $(\cos\theta-i\sin\theta)^n=\cos n\theta-i\sin n \theta\quad (n\in\N)$
\end{enumerate}

\item 利用复数证明余弦定理.

\item 化简:
\begin{enumerate}[(1)]
    \item $\frac{(\pc{7\theta})(\pc{2\theta})}{(\pc{5\theta})(\pc{3\theta})}$
    \item $\frac{\cos\varphi-i\sin\varphi}{\pc{\varphi}}$
    \item $\frac{\pc{\alpha}}{\sin\alpha+i\cos\alpha}$
\end{enumerate}
\item 在复数范围内解下列方程:
\begin{multicols}{2}
\begin{enumerate}[(1)]
    \item $4x^2+9=0$
    \item $2(x^2+4)=5x$
    \item $(x-3)(x-5)=-2$
    \item $\frac{1}{x+3}-\frac{1}{x}=1$
    \item $x^4+3x^2+1=0$
    \item $\frac{x}{x^2+1}+\frac{x^2+1}{x}-\frac{5}{2}=0$
\end{enumerate}
\end{multicols}
\item 解下列方程组:
\begin{multicols}{2}
\begin{enumerate}[(1)]
    \item $\begin{cases}
a+b=2\\ab=2        
    \end{cases}$
    \item $\begin{cases}
        x^2+y^2=0\\
        xy=1
    \end{cases}$
\end{enumerate}
\end{multicols}

\item 求:
\begin{enumerate}[(1)]
    \item $8\left(\pc{\frac{\pi}{3}}\right)$的六次方根;
    \item $-2i$的五次方根.
\end{enumerate}
\item 求:六次单位根、七次单位根,并在复平面上用向量表示出来.

\item 解下列方程:
\begin{multicols}{2}
\begin{enumerate}[(1)]
    \item $x^3+1=i$
    \item $x^4+16=0$
    \item $x^{10}-32x^5+1024=0$
    \item $x^{12}+63x^6-64=0$
\end{enumerate}
\end{multicols}

\item 求证:
\begin{enumerate}[(1)]
    \item 不经计算证明$\left|\frac{a+bi}{a-bi}\right|\equiv 1$
    \item 有相同幅角的两个复数之比必为实数;
    \item 复平面上四个点$Z_1$, $Z_2$, $Z_3$, $Z_4$共圆或共直线的充分必要条件是:
\[\left.\frac{Z_3-Z_1}{Z_3-Z_2}\right/\frac{Z_4-Z_1}{Z_4-Z_2} \text{ 为实数}\]
\end{enumerate}
\end{enumerate}

\section{复数的指数形式}
我们已经学习了复数的代数形式、三角形式以及它们的运算.在科学技术,特别是在电工和无线电计算中,常要涉及复数或三角函数的变换和计算,为方便还采用复数的另一种形式——复数的指数形式,在这一节将学习复数的指数形式及其运算、应用.

\subsection{复数的指数形式及其运算}
我们首先引进一个著名的公式——欧拉公式:
\[e^{i\theta}=\pc{\theta}\]
其中$e=2.71828\cdots$是一个重要的常数,也就是自然对数的底数.

欧拉公式在以后的“复变函数论”中可以证明。这个公式表明了:模为1、幅角为$\theta$的复数$\pc{\theta}$与以$e$为底的复指数函数的关系,从这个公式出发,对任一个复数
\[z=a+bi=r(\pc{\theta})\]
就可以表示成$z=r\cdot e^{i\theta}$的形式,我们就把这一表达式叫做复数的指数形式.

例如:
\begin{itemize}
    \item 复数$\pc{\frac{\pi}{2}}$的指数形式为$e^{i\tfrac{\pi}{2}}$
    \item 复数$3\left(\pc{\frac{\pi}{4}}\right)$的指数形式为$3e^{i\tfrac{\pi}{4}}$
\end{itemize}

这样,同一个复数,就有三种不同形式的表达方法——代数形式、三角形式、指数形式.
如:
\[\begin{split}
    \frac{1}{2}+\frac{\sqrt{3}}{2}i&=\pc{\frac{\pi}{3}}=e^{i\tfrac{\pi}{3}}\\
    -2\sqrt{3}+2i&=4\left(\pc{\frac{5\pi}{6}}\right)=4e^{i\tfrac{5\pi}{6}}
\end{split}\]

复数的指数形式便利于乘、除、乘方及开方运算.因为可以证明原有在实数范围内的指数运算律它都仍然满足. 这就是:

设$z_1=r_1e^{i\theta_1}$,$z_2=r_2e^{i\theta_2}$,则有
\[\begin{split}
    z_1\cdot z_2&=\left(r_1e^{i\theta_1}\right)\cdot \left(r_2e^{i\theta_2}\right)\\
    &=r_1(\pc{\theta_1})\cdot r_2(\pc{\theta_2})\\
    &=r_1r_2[\pcx{\theta_1+\theta_2}]=r_1r_2\cdot e^{i(\theta_1+\theta_2)}
\end{split}\]
即
\[r_1e^{i\theta_1}\cdot r_2e^{i\theta_2}=r_1r_2 e^{i(\theta_1+\theta_2)}\]

同理可证:
\[\begin{split}
    \frac{z_1}{z_2}&=\frac{r_1e^{i\theta_1}}{r_2e^{i\theta_2}}=\frac{r_1}{r_2}\cdot e^{i(\theta_1-\theta_2)}\quad (z_2\ne 0)\\
z^n_1&=(r_1e^{i\theta_1})^n=r^n_1 \cdot e^{i n\theta_1}\quad (n\in\N)
\end{split} \]
$z_1$的$n$次方根:$\sqrt[n]{z_1}=\sqrt[n]{r_1e^{i\theta_1}}=\sqrt[n]{r_1}e^{i\tfrac{\theta_1+2k\pi}{n}}\quad (k=0,1,\ldots,n-1)$

\begin{example}
把复数$z=2i$化成指数形式.
\end{example}

\begin{solution}
$\because\quad z=2i=2\left(\pc{\frac{\pi}{2}}\right)$

$\therefore\quad z=2\cdot e^{i\tfrac{\pi}{2}}$
\end{solution}

\begin{example}
已知$z_1=\sqrt{2}\cdot e^{i\left(-\tfrac{\pi}{4}\right)}$,$z_2=\sqrt{5}\cdot e^{i\tfrac{2\pi}{3}}$. 试求:
\begin{multicols}{2}
\begin{enumerate}[(1)]
    \item $z_1\cdot z_2$
    \item $z_1$的平方根
\end{enumerate}
\end{multicols}
要把结果表示成复数的三角形式.
\end{example}

\begin{solution}
\begin{enumerate}[(1)]
    \item \[\begin{split}
z_1\cdot z_2&=\left(\sqrt{2} e^{i\left(-\tfrac{\pi}{4}\right)}\right)\cdot \left(\sqrt{5} e^{i\tfrac{2\pi}{3}}\right)\\
&=\sqrt{10}e^{i\tfrac{5\pi}{12}}=\sqrt{10}\left(\pc{\frac{5\pi}{12}}\right)
    \end{split}\]
\item $\sqrt{\sqrt{2}\cdot e^{i\left(-\tfrac{\pi}{4}\right)}}=\sqrt[4]{2}\cdot e^{i\tfrac{-\tfrac{\pi}{4}+2k\pi}{2}}\quad (k=0,1)$

$\therefore\quad z_1$的两个平方根为
\[\begin{split}
x_1&=\sqrt[4]{2}\cdot e^{i\left(-\tfrac{\pi}{8}\right)}=\sqrt[4]{2}\left(\pc{\frac{15\pi}{8}}\right)\\
x_2&=\sqrt[4]{2}\cdot e^{i\tfrac{7\pi}{8}}=\sqrt[4]{2}\left(\pc{\frac{7\pi}{8}}\right)
\end{split}\]
\end{enumerate}
\end{solution}



\begin{example}
试用$e^{i\theta}$与$e^{-i\theta}$表示$\cos\theta$与$\sin\theta$.    
\end{example}

\begin{solution}
由于
\begin{align}
e^{i\theta}&= \pc{\theta} \tag{1}\\
e^{-i\theta}&= \pcx{-\theta}=\cos\theta-i\sin\theta \tag{2}
\end{align}
(1)(2)联立,可解出
\[\begin{split}
    \cos\theta&=\frac{1}{2}\left(e^{i\theta}+e^{-i\theta}\right)\\
    \sin\theta&=\frac{1}{2}i\left(e^{i\theta}-e^{-i\theta}\right)\\
\end{split}\]
\end{solution}

\begin{ex}
\begin{enumerate}
    \item 将复数$-1$,$2+2i$,$\pc{15^{\circ}}$,$4i$分别表示成指数形式.
    \item 将复数$e^{-i\tfrac{\pi}{2}}$,$3e^{2i}$,$\sqrt{2}e^{i\tfrac{3\pi}{2}}$分别化为三角形式和代数形式.
    \item 用指数形式计算:
\begin{enumerate}[(1)]
    \item $8\left(\pc{\frac{7\pi}{6}}\right)\cdot 2\left(\pc{\frac{\pi}{4}}\right)$
    \item $2\div e^{i\tfrac{\pi}{4}}$
    \item $(1+\sqrt{3}i)^{10}$
    \item 求81的四次方根
\end{enumerate}
\end{enumerate}
\end{ex}

\subsection{证明三角恒等式}

在我们学习过的三角公式和三角恒等式当中,除了公式$\sin(\alpha+\beta)$与$\cos(\alpha+\beta)$是证明棣美佛定理的依据外,其余公式都可以应用复数的指数形式来加以证明.其要点是:首先将公式中的正切、余切、正割、余割诸函数都化为正弦、余弦函数表达;其次再用由复数的指数形式而导出来的公式:(例1.32)
\[\begin{split}
\sin\theta&=\frac{1}{2}i\left(e^{i\theta}-e^{-i\theta}\right)\\
    \cos\theta&=\frac{1}{2}\left(e^{i\theta}+e^{-i\theta}\right)\\
    \end{split}\]
将三角函数转化为指数形式的代数运算;从而达到证明的目的.

\begin{example}
求证:$-2\sin\alpha\cdot \sin\beta=\cos(\alpha+\beta)-\cos(\alpha-\beta)$    
\end{example}

\begin{proof}
$\because\quad \sin\alpha=\frac{1}{2i}\left(e^{i\alpha}-e^{-i\alpha}\right),\quad \sin\beta=\frac{1}{2i}\left(e^{i\beta}-e^{-i\beta}\right)$

\[\begin{split}
\therefore\quad \text{左}&=-2\cdot \frac{1}{4i^2}\left[e^{i(\alpha+\beta)}+e^{-i(\alpha+\beta)}-e^{i(\alpha-\beta)}-e^{i(\beta-\alpha)}\right]\\
&=\frac{1}{2}\left[e^{i(\alpha+\beta)}+e^{-i(\alpha+\beta)}\right]-\frac{1}{2}\left[e^{i(\alpha-\beta)}+e^{-i(\alpha-\beta)}\right]\\
&=\cos(\alpha+\beta)-\cos(\alpha-\beta)=\text{右}
\end{split}\]

$\therefore\quad $原等式成立.
\end{proof}

\begin{example}
求证:$\sin(\alpha+\beta)\cdot \cos(\alpha-\beta)=\sin\alpha\cdot \cos\alpha+\sin\beta\cdot \cos\beta$
\end{example}

\begin{proof}
\[\begin{split}
\text{左式}&=\frac{1}{2i}\left[e^{i(\alpha+\beta)}-e^{-i(\alpha+\beta)}\right]\cdot \frac{1}{2}\left[e^{i(\alpha-\beta)}+e^{-i(\alpha-\beta)}\right]\\
&=\frac{1}{4i}\left[e^{2\alpha i}+e^{2\beta i}-e^{-2\beta i}-e^{-2\alpha i}\right]
\end{split}\]
\[\begin{split}
    \text{右式}&=\frac{1}{2i}\left(e^{i\alpha}-e^{-i\alpha}\right)\cdot\frac{1}{2}\left(e^{i\alpha}+e^{-i\alpha}\right)+\frac{1}{2i}\left(e^{i\beta}-e^{-i\beta}\right)\cdot \frac{1}{2}\left(e^{i\beta}+e^{-i\beta}\right)\\
    &=\frac{1}{4i}\left[e^{2\alpha i}-e^{-2\alpha i}+e^{2\beta i}-e^{-2\beta i}\right]
\end{split}\]

$\because\quad $左式$=$右式

$\therefore\quad $原等式成立.
\end{proof}

对于具有条件$\alpha+\beta+\gamma=\pi$的三角恒等式,应用复数的指数形式加以证明更为有利.

\begin{example}
在$\triangle ABC$中,求证:
\[1+\cos 2A+\cos 2B+\cos 2C+4\cos A\cdot \cos B\cdot \cos C=0\]
\end{example}

\begin{proof}
\[\begin{split}
 \because\quad    4\cos A\cdot \cos B\cdot \cos C &=4\cdot\frac{e^{iA}+e^{-iA}}{2}\cdot\frac{e^{iB}+e^{-iB}}{2}\cdot\frac{e^{iC}+e^{-iC}}{2}\\
 &=\frac{1}{2}\left[e^{i(A+B+C)}+e^{i(B+C-A)}+e^{i(A+C-B)}\right.\\
 &\qquad   +e^{i(A+B-C)}+e^{-i(B+C-A)}+e^{-i(A+C-B)}\\
&\qquad  \left.+e^{-i(A+B-C)}+e^{-i(A+B+C)}\right]\qquad  \qquad  (*)
\end{split}\]

但由于$A+B+C=\pi$,进而有$B+C-A=\pi-2A$, 
$A+C-B=\pi-2B$, $A+B-C=\pi-2C$; 因此可得:
\[\begin{split}
    e^{i(A+B-C)}&=e^{i\pi}=-1\\
    e^{i(B+C-A)}&=e^{i(\pi-2A)}=e^{i\pi}\cdot e^{-2Ai}=-e^{-2Ai}\\
\end{split}\]
同理:$e^{i(A+C-B)}=-e^{-2Bi},\qquad e^{i(A+B-C)}=-e^{-2Ci}$
\[\begin{split}
    e^{-i(A+B+C)}=e^{-i\pi}=-1,\qquad 
    e^{-i(B+C-A)}=-e^{2Ai}\\
    e^{-i(A+C-B)}=-e^{2Bi},\qquad 
    e^{-i(A+B-C)}=-e^{2Ci}
\end{split}\]

将以上结果代入(*)式,得
\[\begin{split}
    4\cos A\cdot \cos B\cdot \cos C &=\frac{1}{2}\left[-2-e^{2Ai}-e^{2Ai}-e^{2Bi}-e^{2Bi}-e^{2Ci}-e^{2Ci}\right]\\
    &=-1-\cos2A-\cos2B-\cos2C
\end{split}\]

$\therefore\quad 1+\cos 2A+\cos 2B+\cos 2C+4\cos A  \cos B  \cos C=0$
\end{proof}

\begin{ex}
利用复数的指数形式,证明以下等式:
\begin{enumerate}
    \item $\sin2\alpha=2\sin\alpha\cdot \cos\alpha$
    \item $\cos^2\beta-\sin^2\alpha=\cos(\alpha+\beta)\cdot \cos(\alpha-\beta)$
\end{enumerate}
\end{ex}

\section*{习题1.4}
\begin{enumerate}
    \item 把下列复数化为指数形式:
\begin{multicols}{3}
\begin{enumerate}[(1)]
    \item $5$
    \item $2+2i$
    \item $-2i$
    \item $1-\sqrt{3}i$
    \item $\pc{\frac{\pi}{4}}$
    \item $\pc{3}$
    \item $1+\pc{\frac{\pi}{3}}$
\end{enumerate}
\end{multicols}
\item 设复数$a+bi=r\cdot e^{i\theta}$,写出复数$a-bi$, $-a+bi$, $-a-bi$的指数形式.
\item 把下列复数化为三角形式及代数形式:
\begin{multicols}{3}
\begin{enumerate}[(1)]
    \item $4e^{i\tfrac{\pi}{6}}$
    \item $\sqrt{3}e^{i\tfrac{2\pi}{3}}$
    \item $e^{-i\tfrac{3\pi}{2}}$
\end{enumerate}
\end{multicols}

\item 试求:复数$e^{i\tfrac{4\pi}{5}}$, $e^{i\tfrac{2\pi}{3}}$ 所对应的向量间的夹角$\alpha\; (0\le \alpha\le \pi)$.
\item 用复数的指数形式计算:
\begin{enumerate}[(1)]
    \item $\sqrt{2}\left(\pc{\frac{4\pi}{3}}\right)\cdot\frac{\sqrt{2}}{4}\left(\pc{\frac{5\pi}{6}}\right)$
    \item $\sqrt{3}(\pc{150^{\circ}})\div \sqrt{2}(\pc{225^{\circ}})$
\end{enumerate}

\item 根据欧拉公式求证:$e^{-i\theta}=(\pc{\theta})^{-1}$.

\item 用复数的指数形式计算:
\begin{multicols}{3}
\begin{enumerate}[(1)]
    \item $(1+\sqrt{3}i)^{10}$
    \item $(\sqrt{3}-i)^8$
    \item $\sqrt[4]{64+0i}$
\end{enumerate}
\end{multicols}

\item 用复数的指数形式证明恒等式:
\begin{enumerate}[(1)]
    \item $\sin\alpha-\sin\beta=2\cos\frac{\alpha+\beta}{2}\cdot \sin\frac{\alpha-\beta}{2}$
    \item $(\sin\alpha+\sin\beta)^2+(\cos\alpha+\cos\beta)^2=4\cos^2\frac{\alpha-\beta}{2}$
\end{enumerate}
\end{enumerate}

\section*{本章内容要点}

一、本章里,我们引进了虚数单位($i^2=-1$),把数的概念扩展到了复数集. 复数的分类如下:
\begin{center}
\begin{tikzpicture}
\node (A) at (0,0)[text width=2cm, align=center]{复数$a+bi$\\$(a,b\in\R)$};
\node (B1) at (2.5,1)[text width=2cm, align=center]{实数\\($b=0$时)}; 
\node (B2) at (2.5,-1)[text width=2cm, align=center]{虚数\\($b\ne 0$时)}; 
\node (C1) at (5,0)[text width=2cm, align=center]{纯虚数\\($a=0$时)}; 
\node (C2) at (5,-2)[text width=2cm, align=center]{非纯虚数\\($a\ne 0$时)}; 
\draw[decorate, decoration={brace, amplitude=5pt}](1.5,-1.3)--(1.5,1.3);
\draw[decorate, decoration={brace, amplitude=5pt}](3.8,-2.3)--(3.8,.3);


\end{tikzpicture}
\end{center}

且复数集$\supset$实数集$\supset$有理数集$\supset$整数集$\supset$自然数集,即:
\[\mathbb{C} \supset \R \supset \mathbb{Q} \supset \Z\supset\N\]

二、复数有四个基本概念:模,幅角,共轭复数,复数的相等;任一个复数有三种表示形式:代数形式$a+bi\; (a,bE\in\R)$,三角形式$r(\pc{\theta})$,其中$r=\sqrt{a^2+b^2}$, $\tan\theta =\frac{b}{a}$,指数形式$re^{i\theta}$;复数$z=a+bi$与复平
面内的点$Z(a,b)$对应,并且和从原点出发的向量$\VEC{OZ}$相对应,因此,复数集$\mathbb{C}$与复平面内的点集,与从原点出发的向量集合之间成一一对应关系,即
\[\mathbb{C}:\; \{a+bi\mid a,b\in\R,\; i^2=-1\} \longleftrightarrow \{Z(a,b)\}\longleftrightarrow \{\VEC{OZ}\}\]

据此,一些平面几何,平面解析几何以及平面向量的问题,也可以通过复数来解决.

三、复数集对于四则运算及乘方,开方运算都是封闭的;复数集保持了数系运算的“通性”,即加法、乘法的交换律、结合律以及乘法对于加法的分配律、数0与1的运算特性等;但复数集不再具有顺序性,对任意两个不全为实数的复数,就不能比较大小.

四、复数的运算法则及其几何意义:
\begin{enumerate}
    \item 加、减法运算用复数的代数形式表达比较方便,即
    \[(a+bi)\pm (c+di)=(a\pm c)+(b\pm d)i\]
    其几何意义是对应的向量相加、减,即“平行四边形”法则和“三角形”法则,特别是两复数的差的模表示复平面内两点间的距离.
    \item 乘、除法,乘方、开方的运算用复数的三角形式或指数形式表达比较方便,即
\[\begin{split}
    r_1(\pc{\theta_1})\cdot r_2(\pc{\theta_2})&=r_1\cdot r_2\left[r_1\pcx{\theta_1+\theta_2}\right]\\
\frac{r_1(\pc{\theta_1})}{r_2(\pc{\theta_2})}&=\frac{r_1}{r_2}\left[r_1\pcx{\theta_1-\theta_2}\right]
\end{split}\]
棣美佛定理:
\[[r(\pc{\theta})]^n=r^n(\pc{n\theta})\]
$n$个$n$次方根:
\[\begin{split}
\sqrt[n]{r(\pc{\theta})}&=\sqrt[n]{r}\left(\pc{\frac{\theta+2k\pi}{n}}\right)\\
&\qquad (k=0,1,2,\ldots,n-1)\quad (n\in\N)    
\end{split} \]
\end{enumerate}

复数的这些运算,都有其几何意义.特别是:复数的相乘对应着平面向量的旋转;$n$个$n$次方根对应着圆内接正$n$边形的顶点,即圆周的$n$等分点,这些结论在应用上很重要.

五、在复数集中,本章学习了在实数集中所不能解的有关问题:
\begin{enumerate}
    \item 当$b^2-4ac<0$时,一元二次方程$ax^2+bx+c=0\; (a\ne 0)$有两个共轭的虚根,即
\[x=\frac{-b\pm i\sqrt{4ac-b^2}}{2a}\]
\item 二项方程$x^n=b\; (b\in\mathbb{C})$有$n$个不同的复数根,即设$x^n=b=r(\pc{\theta})$,则
\[x=\sqrt[n]{r}\left(\pc{\frac{\theta+2k\pi}{n}}\right)\]
当$k=0,1,2,\ldots,n-1$时,可得出$n$个不同的解.

\item 特别地,方程$x^n=1$的复数解,叫做$n$次单位根,若设$\omega=\pc{\frac{2\pi}{n}}$,则当$n$为偶数时,$n$个单位根为
\[1,\; \omega,\; \omega^2,\;\ldots,\;  \omega^{\tfrac{n}{2}-1},\;-1,\;  \overline{\omega^{\tfrac{n}{2}-1}},\; \ldots, \;  \overline{\omega^{2}},\; \overline{\omega}\]
当$n$为奇数时,$n$个单位根为
\[1,\; \omega,\; \omega^2,\;\ldots,\;  \omega^{\tfrac{n-1}{2}},\;  \overline{\omega^{\tfrac{n-1}{2}}},\; \ldots, \;  \overline{\omega^{2}},\; \overline{\omega}\]

\item 实系数一元方程,若有虚根存在,则一定是成对出现,即实系数方程若有根$a+bi\;(b\ne 0)$则必有根$a-bi$.
\end{enumerate}

\section*{复习题一}
\begin{enumerate}
    \item 在复平面上任选一点$Z$(不在原点)表示复数$z$,然后用几何法在平面找出下列各复数的点:
\begin{multicols}{4}
\begin{enumerate}[(1)]
    \item $-z$
    \item $\bar z$
    \item $-\bar z$
    \item $z+\bar z$
    \item $z-\bar z$
    \item $z+|z|$
    \item $z-|z|$
    \item $iz$
    \item $-iz$
    \item $i\bar z$
    \item $-i\bar z$
\end{enumerate}
\end{multicols}

\item 已知$z=a+bi\; (a,b\in\R)$,求下列各复数的实部与虚部:
\begin{multicols}{2}
\begin{enumerate}[(1)]
    \item $z^2$
    \item $z^3$
    \item $\frac{1}{z}$
    \item $v_0 z+\frac{M}{2\pi}\cdot \frac{1}{z}\quad (v_0,M\in\R)$
\end{enumerate}
\end{multicols}

\item 求复数$z=\frac{a^2-b^2+2abi}{ab\sqrt{2}+\sqrt{a^4+b^4}i}$的模.

\item 求证:表示复数$z_0$的点关于直线${\rm Re}(z)={\rm Im}(z)$的对称点表示的是复数$\overline{i z_0}$.

\item 下列方程($t$为实参数)给出了怎样的曲线?
\begin{multicols}{2}
\begin{enumerate}[(1)]
    \item $z=t(1+i)$
    \item $z=a\cos t+ib\sin t$
    \item $z=t+\frac{i}{t}$
    \item $z=t^2+\frac{i}{t}$
\end{enumerate}
\end{multicols}

\item 已知$(x+yi)^3=a+bi$,$a,b,x,y\in\R$,求证:
$\frac{a}{x}+\frac{b}{y}=4(x^2-y^2)$

\item 若$a$,$b$是共轭复数,且$(a+b)-3abi=4-6i$, 试求$a$及$b$.
\item 若$x,y\in\R$, $z=x+y-30-xyi$且$\bar z=60i-|x+yi|$,试求$x$和$y$的值.
\item 若$z=x+yi$,求证:$|z|\le |x|+|y|\le \sqrt{2}|z|$
\item 试判断以下复数,哪些是实数?哪些是纯虚数?($z,z_1,z_2\in\mathbb{C}$)
\begin{multicols}{2}
\begin{enumerate}[(1)]
    \item $z+\bar z$
    \item $z-\bar z$
    \item $z\cdot \bar z$
    \item $z^2-\bar z^2$
    \item $\frac{\left(\frac{1}{z}+\frac{1}{\bar z}\right)(z+\bar z)}{z-\bar z}$
    \item $z_1\cdot \bar z_2-\bar z_1\cdot z_2$
    \item $z_1\cdot \bar z_2+\bar z_1\cdot z_2$
    \item $\frac{z_1\cdot \bar z_2+\bar z_1\cdot z_2}{z_1\cdot \bar z_1-1}$
    \item $\frac{z_1\cdot \bar z_2-\bar z_1\cdot z_2}{i(z_1\cdot \bar z_1+z_2\cdot \bar z_2)}$
    \item $\frac{i(z_1\cdot \bar z_2+z_2\cdot \bar z_1)}{z_1\cdot \bar z_2-\bar z_1\cdot z_2}$
\end{enumerate}
\end{multicols}

\item 设复数$z_1,z_2$分别对应于点$Z_1,Z_2$,求证:
\begin{enumerate}[(1)]
    \item $OZ_1\bot OZ_2$的充要条件是$z_1\cdot \bar z_2+\bar z_1\cdot z_2=0$
    \item $OZ_1$与$OZ_2$共线的充要条件是$z_1\cdot \bar z_2-\bar z_1\cdot z_2=0$
\end{enumerate}

\item 设$\alpha$是复常数,$k$是实常数,$z\in\mathbb{C}$,试说明下述方程在复平面上的图形是什么?
\begin{multicols}{2}
\begin{enumerate}[(1)]
    \item $\alpha\bar z+\bar\alpha\cdot z=k$;
    \item $z\bar z+z\bar\alpha+\alpha\bar z+k=0$.
\end{enumerate}
\end{multicols}
\item 求证:如果$\frac{1}{z}+z\in\R$,那么${\rm Im}(z)=0$或$|z|=1$.
\item 化简:$\left(\frac{1-i}{1+i}\right)^{4\alpha}$,其中$\alpha=\csc10^{\circ}-\sqrt{3}\sec10^{\circ}$.
\item 解方程组
\begin{multicols}{2}
\begin{enumerate}[(1)]
    \item $\begin{cases}
        x+iy=2+4i\\ ix+y=0
    \end{cases}$
    \item $\begin{cases}
        z_1+2z_2=1+i\\  3z_1+iz_2=2-3i
    \end{cases}$
    \item $\begin{cases}
x^2+y^2=6\\        
x^2+y^2=0\\        
x^2+z^2=8i\\        
    \end{cases}$
\end{enumerate}
\end{multicols}

\item 化简
\[\frac{(\cos2\theta-i\sin2\theta)(\pc{\theta})^2}{\pcx{\theta+\varphi}}\cdot \frac{(\pc{2\theta})^2(\cos2\varphi-i\sin2\varphi)}{\pcx{\theta-\varphi}}\]

\item 将下列复数化为三角形式:
\begin{multicols}{2}
\begin{enumerate}[(1)]
    \item $\frac{(\pc{5\varphi})^2}{(\cos3\varphi-i\sin3\varphi)^3}$
    \item $1-\pc{\theta},\quad (0\le\theta\le\pi)$
\end{enumerate}
\end{multicols}

\item 设$x=e^{i\alpha}$, $y=e^{i\beta}$, $z=e^{i\gamma}$,求证:
\[(x+y)(y+z)(z+x)=8xyz\cos\frac{\beta-\gamma}{2}\cdot \cos\frac{\gamma-\alpha}{2}\cdot \cos\frac{\alpha-\beta}{2}\]

\item 设$\sin A+\sin B+\sin C=\cos A+\cos B+\cos C=0$,求证:
\begin{enumerate}[(1)]
    \item $\sin3A+\sin3B+\sin3C=3\sin(A+B+C)$
\item $\cos3A+\cos3B+\cos3C=3\cos(A+B+C)$
\end{enumerate}
\item 在复平面上,$z_1=1$, $z_2=2+i$与$z_3$组成一个正三角形,求$z_3$.
\item 若$z_1,z_2,z_3,z_4$是一个正方形的四个顶点,且$z_1=1+2i$, $z_2=3-5i$, 求$z_3,z_4$.
\item 求证:$\triangle z_1z_2z_3$是等边三角形的充要条件是:
\[z_1^2+z_2^2+z_3^2=z_1\cdot z_2+z_2\cdot z_3+z_3\cdot z_1\]
\item 下列条件表示什么样图形?
\begin{enumerate}[(1)]
    \item ${\rm Im}(z^2)=2$
    \item $|z|=1$且$\arg z\in(0,\pi)$
    \item $0<a\le {\rm Im}(z)\le b$且$\frac{\pi}{4}\le \arg z\le \frac{3\pi}{4}$
\end{enumerate}
\item 你能用复数表示出图中的半圆来吗?(包括边界)

\begin{minipage}{.45\textwidth}
    \centering
\begin{tikzpicture}[>=stealth]
\draw[->](-1.5,0)--(1.5,0)node[right]{$x$};
\draw[->](0,-1)--(0,2)node[right]{$y$};
\draw[pattern=north east lines](1,0)node[below]{2} arc (0:180:1)node[below]{$-2$} --cycle;
\node[below left]{$O$};
\node at (0,1)[above right]{2};    
\end{tikzpicture}
\captionof{figure}{ }
\end{minipage}\hfill
\begin{minipage}{.45\textwidth}
    \centering
\begin{tikzpicture}[>=stealth]
    \draw[->](-1.5,0)--(1.5,0)node[right]{$x$};
\draw[->](0,-1.5)--(0,1.5)node[right]{$y$};
\draw[pattern=north east lines](0,1)node[right]{1} arc (90:180+90:1)node[right]{$-1$}--cycle; 
\node[below right]{$O$};   
\node at (-1,0)[below left]{$-1$};   
\end{tikzpicture}
\captionof{figure}{ }
\end{minipage} 

\item 设$z\in\mathbb{C}$,解方程:
\begin{multicols}{2}
\begin{enumerate}[(1)]
    \item $\frac{1}{2}(z-1)=\frac{\sqrt{3}}{2}(1+z)i$
    \item $(z+1)^8=(1+i)^8$
\end{enumerate}
\end{multicols}

\item 设$z$是虚数,解方程:
\begin{multicols}{2}
\begin{enumerate}[(1)]
    \item $z+|\bar z|=2+i$
    \item $z^2=\bar z$
\end{enumerate}
\end{multicols}

\item 求证:
\begin{enumerate}[(1)]
    \item $|z|=1\; (z\in\mathbb{C})$的充要条件是$z^{-1}=\bar z$;
    \item 共轭虚数的$n\; (n\in\N)$次幂仍是共轭虚数;
    \item 虚数的平方根仍是虚数.
\end{enumerate}

\item 在复数集中解下列方程组:
\begin{enumerate}[(1)]
    \item $\begin{cases}
        x^2+y^2-2xy+3(x+y)-2=0\\
        2(x^2+y^2)-2xy-2(x+y)+15=0
    \end{cases}$
    \item $\begin{cases}
        \frac{x-y}{1+xy}=\frac{1}{3}\\  \frac{x+y}{1-xy}=3
    \end{cases}$
\end{enumerate}    

\item 设三次单位根为$1,\omega,\omega^2$,其中
$\omega=-\frac{1}{2}+\frac{\sqrt{3}}{2}i$,求证:
\begin{enumerate}[(1)]
    \item $1+\omega+\omega^2=0$
    \item $1\cdot \omega\cdot \omega^2=1$
    \item $(1+\omega^2)^4=\omega$
    \item $\begin{vmatrix}
        1&1&\omega\\
        1&1&\omega^2\\
        \omega^2&\omega&1
    \end{vmatrix}=-3$
    \item $1^n+\omega^n+(\omega^2)^n=\begin{cases}
        3,& n=3k\\
        0,& n\ne 3k
    \end{cases}(k\in\Z)$

    \item $a^3+b^3=(a+b)(a\omega+b\omega^2)(a\omega^2+b\omega);$
    \item $a^{3}+b^{3}+c^{3}-3abc =(a+b+c)(a+b\omega+c\omega^{2})(a+b\omega^{2}+c\omega)$
\end{enumerate}

\item 利用上题中$\omega$的性质,计算:
\begin{multicols}{2}
\begin{enumerate}[(1)]
    \item $(-1+\sqrt{3}i)^{6}-(-1-\sqrt{3}i)^{6}$;
    \item $\left(\frac{\sqrt{3}+i}{2}\right)^{6}-\left(\frac{\sqrt{3}-i}{2}\right)^{6}$;
    \item $\left(-\frac{1}{2}i-\frac{\sqrt{3}}{2}\right)^{12}$.
\end{enumerate}    
\end{multicols}

\item 已知$f(x)=x^4+3x^2-30x^2+366x-340$的一个根为$3+5i$,求其余的三个根。
\item 若实系数方程$x^4+px^3+qx^2+rx+s=0$有两个根为$2+i$, $-2+i$. 求$p$、$q$、$r$、$s$的值。
\item 若实系数多项式$x^4+px^3+qx^2+rx+s$有一个根是纯虚数,求证:$r^2+p^2s=pqr$.

\item 已知实系数方程$x^2+mx+n=0$的一个根是另一个根的$i$倍,试求$m$、$n$的关系.
\item 已知$f(x)=x^4+(2+i)x^3+(3+2i)x^2-(4-3i)x-4i$, 

$g(x)=x^4-(1-i)x^3-(1+i)x^2+(4+i)x-4i$,求$f(x)$与$g(x)$的公根.
\end{enumerate}


\chapter{排列与组合~~二项式定理}
\section{排列与组合}
\subsection{加法原理和乘法原理}
假如我们要从甲地到乙地,可以乘火车、轮船或公共汽车,火车每天有两班,轮船有两班,公共汽车有六班,问从甲地到乙地有多少种不同的方法?

因为,从甲地到乙地,乘火车有两种不同的方法,乘轮船有两种不同的方法,乘公共汽车有六种不同的方法.而每种方法都可以由甲地到达乙地,因为,从甲地到乙地共有
$2+2+6=10$
种不同的方法.

一般地说,有如下原理:
\begin{blk}{加法原理}
如果完成某件事有$n$种方式,在第一种方式中有$m_1$种方法,在第二种方式中有$m_2$种方法……,在第$n$种方式中有$m_n$种方法. 那么完成这件事共有$m_1+m_2+\cdots+m_n$种不同的方法.
\end{blk}

又假如,要从甲地到丙地,必须经过乙地,现在已知由甲地到乙地有三条道路,由乙地到丙地有两条道路(图2.1),问从甲地到丙地共有多少种不同的走法?
\begin{figure}[htp]
    \centering
\begin{tikzpicture}
\node[draw, circle] (A) at (0,0){甲};    
\node[draw, circle] (B) at (4,0){乙};    
\node[draw, circle] (C) at (8,0){丙};
\draw(A)--(B);
\draw(A)to [bend left=45](B);
\draw(A)to [bend left=-45](B);
\draw(C)to [bend left=15](B);
\draw(C)to [bend left=-15](B);

\end{tikzpicture}
    \caption{}
\end{figure}

因为从甲地到乙地有3种不同的走法,到乙地后又各有两种不同的走法到丙地,因此,从甲地到丙地共有
$3\x2=6$
种不同的走法.

一般地说,有如下原理:

\begin{blk}
    {乘法原理}如果完成某种事需要分成$n$个步骤,第一步骤有$m_1$种方法,第二步骤有$m_2$种方法……,第$n$步骤有$m_n$种方法,那么完成这件事共有$m_1\cdot m_2\cdots m_n$种不同的方法.
\end{blk}

总的来说,如果完成一种事有几种不同的方法,这些方法又是互不相关的,任选一种方法都能完成这件事的,那么完成这件事的方法的总数,应该用加法计算. 如果完成一件事,必须分成若干步骤,每个步骤又有不同的方法,而且每一步骤中的一种方法完成之后,都可以开始下一步骤的工作,依次完成全部步骤,才能达到完成这一件事的目的,那么完成这件事的方法的总数应该用乘法计算.

\begin{ex}
\begin{enumerate}
    \item 完成某件工作,甲有两种不同的方法,乙有3种不同方法,丙有4种不同方法.从三人中选一人完成这件工作,共有几种不同选法?
    \item 一种车床用两个手柄联合控制转速,一个手柄有两档,另一个手柄有3档,问能控制几种不同的转速?
    \item 乘积$(a+b)\cdot (c+d+e)\cdot (m+n+p+q)$展开后共有多少项?
    \item 从甲地到乙地有一条道路,从乙地到丁地有两条道路;从甲地到丙地有三条道路,从丙地到丁地有四条道路.若从甲地到丁地必须经过乙地或丙地,问从甲地到丁地共有多少种不同的走法?
\end{enumerate}
\end{ex}


\subsection{排列}
一般地说,从$n$个不同的元素中任意选取$m\; (m\le n)$个元素,按照一定的顺序排成一列,叫做从$n$个元素中取出$m$个元素的一个排列.

例如:由$1,2,3$三个数字中选取两个数字写成两位数,那么$1,2,3$都被看作是排列的元素。而12与21显然都是由$1,2$两个元素得到的不同的排列.同样,12,21,13,31,23,32等都被认为是不同的排列.只有用相同的元素,又是按相同的顺序排列而成的排列,才叫做相同的排列,例如:123与123就是相同的排列.

\begin{example}
    由数字1,2,3,4可以组成多少个没有重复数字的两位数?
\end{example}

\begin{solution}
    要排出两位数,第一步先要排出十位上的数字(当然也可以先排个位上的数字),第二步是确定个位上的数字.

显然,第一步在十位的位置上,$1,2,3,4$都可以放,故有四种不同的方法.

第二步是放个位上的数字,由于在第一步结束时,剩下的只有3个数字,因此第二步只能有3种不同的选择.因此个位与十位上的数的排列工作都结束后,这个两位数才算排出来,所以,用乘法原理计算得:$4\x3=12$.即以$1,2,3,4$可以排出12个不同的两位数.
\end{solution}

\begin{center}
\begin{tikzpicture}
\begin{scope}
\draw(-.5,-.25) rectangle (.5,.25);
\draw(0,-.25)--(0,.25);
\node at (-0.25,0){1};
\foreach \x/\y in {-1/4,0/3,1/2}
{
    \draw(1, \x-.25) rectangle (2, \x+.25);
    \draw(1.5,\x-.25)--(1.5,\x+.25);
    \node at (1.25,\x){1};
    \node at (1.75,\x){\y};
}
\draw[decorate, decoration={brace, amplitude=5}](.9,-1.25)--(.9,1.25);
\end{scope}
\begin{scope}[xshift=3cm]
    \draw(-.5,-.25) rectangle (.5,.25);
    \draw(0,-.25)--(0,.25);
    \node at (-0.25,0){2};
    \foreach \x/\y in {-1/4,0/3,1/1}
    {
        \draw(1, \x-.25) rectangle (2, \x+.25);
        \draw(1.5,\x-.25)--(1.5,\x+.25);
        \node at (1.25,\x){2};
        \node at (1.75,\x){\y};
    }
    \draw[decorate, decoration={brace, amplitude=5}](.9,-1.25)--(.9,1.25);
    \end{scope}
    \begin{scope}[xshift=6cm]
        \draw(-.5,-.25) rectangle (.5,.25);
        \draw(0,-.25)--(0,.25);
        \node at (-0.25,0){3};
        \foreach \x/\y in {-1/4,0/2,1/1}
        {
            \draw(1, \x-.25) rectangle (2, \x+.25);
            \draw(1.5,\x-.25)--(1.5,\x+.25);
            \node at (1.25,\x){3};
            \node at (1.75,\x){\y};
        }
        \draw[decorate, decoration={brace, amplitude=5}](.9,-1.25)--(.9,1.25);
        \end{scope}
\begin{scope}[xshift=9cm]
\draw(-.5,-.25) rectangle (.5,.25);
\draw(0,-.25)--(0,.25);
\node at (-0.25,0){4};
\foreach \x/\y in {-1/3,0/2,1/1}
{
    \draw(1, \x-.25) rectangle (2, \x+.25);
    \draw(1.5,\x-.25)--(1.5,\x+.25);
    \node at (1.25,\x){4};
    \node at (1.75,\x){\y};
}
\draw[decorate, decoration={brace, amplitude=5}](.9,-1.25)--(.9,1.25);
\end{scope}
\end{tikzpicture}
\end{center}




\begin{example}
$a,b,c,d$四个元素,每次取出三个元素的不同排列有多少种,并写出所有的排列.(在同一排列里,元素不能重复出现.)
\end{example}

\begin{solution}
要把取出的三个元素排成一列,要分三步完成. 第一步先在第一个位置上放一个元素,应该有四种不同的方法;第二步要在第二个位置上放一个元素,由于第一步的工作完成后剩下的是三个元素,从三个元素中选一个放在第二个位置上应有3种不同的方法;第三步要在第三个位置上放一个元素,由于前两步工作完成后只剩下两个元素,所以只有两种不同的方法. 三步工作都完成后才完成排列工作,因此应该用乘法原理计算排列的总数,所以共有:$4\x4\x3\x2=24$种不同的排列.即有如下的24种不同的排列:
\begin{center}
\begin{tabular}{cccccc}
$abc$& $abd$& $acb$ &$acd$ &$adb$& $adc$\\
$bac$& $bad$ &$bca$& $bcd$ &$bda$& $bdc$\\
$cab$& $cad$ &$cba$& $cbd$ &$cda$& $cdb$\\
$dab$& $dac$& $dba$ &$dbc$& $dca$ &$dcb$    
\end{tabular}
\end{center}
\end{solution}

一般地说,从$n$个不同元素中,任取$m$个$(m\le n)$元素(不许重复),按照一定顺序排成一列的个数,叫做从$n$个不同元素中取出$m$个元素的排列数,用符号表示$\perm{m}{n}$.

\begin{blk}
  {定理1} 从$n$个不同元素中,任取$m$个$(m\le n)$没有重复的元素的排列数为$\perm{m}{n}=n(n-1)\cdots (n-m+1)$
\end{blk}

\begin{proof}
    在取出$m$个元素排成一列时,首先在第一个位置上可以选$n$个元素中的任何一个,故有$n$种方法. 在第二个位置上可以选余下的$n-1$个元素中的任一个.如此类推,在第$m$个位置上可以选余下的$(m-1)$个元素中的任一个. 根据乘法原理得到总的不同排列数应该是
\[\perm{m}{n}=m(n-1)\cdots(m-m+1)\]

这里的$m$是自然数,并且$m\le n$, 这个公式就叫做排列数公式.
\end{proof}

特别是当$m=n$时 有
\[\perm{n}{n}=n(n-1)\cdots3\cdot2\cdot1\]
这表示,$n$个不同的元素全部取出参加排列的排列数,叫做$n$个不同元素的全排列,自然数1到$n$的连乘积叫做$n$的阶乘,用$n!$表示,所以全排列数公式又可以写成
\[\perm{n}{n}=n!\]
有了阶乘,排列数公式又可以写成
\[\perm{m}{n}=\frac{n!}{(n-m)!}\]
这里,当$m=n$时,分母出现$(n-m)!$就变成$0!$,为了使公式仍能成立,我们特别规定$0!=1$.

例如$\perm{5}{8}=8\cdot  7\cdot 6\cdot 5\cdot 4=6720$,
$\perm{5}{5}=5!=5\cdot 4\cdot 3\cdot 2\cdot 1=120$

\begin{example}
    试证明下列等式成立.
\begin{multicols}{2}
\begin{enumerate}[(1)]
    \item $\perm{m}{n}+m\cdot \perm{m-1}{n}=\perm{m}{n+1}\quad (m\le n)$
\item $\perm{n+1}{n+1}-\perm{n}{n}=n^2\cdot \perm{n-1}{n-1}$
\end{enumerate}
\end{multicols}
\end{example}

\begin{proof}
\begin{enumerate}[(1)]
    \item \[\begin{split}
        \perm{m}{n}+m\cdot \perm{m-1}{n}&=\frac{n!}{(n-m)!}+m\cdot\frac{n!}{(n-m+1)!}\\
        &=\frac{(n-m+1)\cdot n!+m\cdot n!}{(n-m+1)!}=\frac{(n+1)\cdot n!}{(n-m+1)!}\\
        &=\frac{(n+1)!}{[(n+1)-m]!}=\perm{m}{n+1}
    \end{split}\]
所以原等式成立.

\item \[\begin{split}
    \perm{n+1}{n+1}-\perm{n}{n}&=(n+1)!-n!=(n+1)\cdot n!-n!\\
    &=n\cdot n!=n^2\cdot (n-1)!\\
    &=n^2\cdot \perm{n-1}{n-1}
\end{split}\]
所以原等式成立.
\end{enumerate}
\end{proof}

\begin{example}
    用0到9十个数字,可以写出多少个没有重复数的三位数?其中有多少个是偶数?
\end{example}

\begin{solution}
\begin{enumerate}[(1)]
    \item 第一步,因为0不能放在百位的位置上,所百位上的数有9种不同的选法,
    
第二步是从余下的九个数字中任选两个数字放在十位个位上,这有$\perm{2}{9}$种方法.所以
\[9\cdot \perm{2}{9}=9\cdot 9\cdot 8=648\]
\item 要得到三位偶数,必须个位上的数是0, 2, 6或8.但是0又不能排在百位上,这说明0怎样放在适合的置上是比其他问题要求更高的. 所以我们还是把0元素的题先解决.

第一类,0放在个位上这一选法确定后,其余百位、十位可以从余下的九个数字中任选两个数字放上去,这有$\perm{2}{9}$种方法.

第二类,个位上可放2,4,6,8中的任一个,在这类方法中,第一步在个位上有4种选法,第二步,在百位上可以放除0以外所余的八个字数的任一个,故有八种选择方法.第三步,余下的八个数字均可以放在十位上,所以
\[\perm{2}{9}+4\x8\x\perm{1}{8}=72+256=328\]
\end{enumerate}
答:有648个没有重复数字的三位数,其中有328个是偶数.

\end{solution}

关于例2.4中的第一个问题,我们还可以有如下的解法:

从0到9十个数字中任选三个数字排列总数为$\perm{3}{10}$.但是其中0在百位上的数有$\perm{2}{9}$个. 所以
\[\perm{3}{10}-\perm{2}{9}=10\cdot 9\cdot 8-9\cdot 8=648\]

\begin{ex}
\begin{enumerate}
    \item 写出从$a,b,c,d,e$中任取两个不同元素的所有排列.
    \item 计算:
\begin{multicols}{3}
\begin{enumerate}[(1)]
    \item $\perm{3}{7}$
    \item $\perm{3}{15}$
    \item $\perm{4}{4}$
    \item $\perm{4}{8}-3\cdot\perm{2}{8}$
    \item $\perm{3}{7}-5\perm{2}{7}$
    \item $\frac{\perm{7}{12}}{\perm{6}{11}}$
\end{enumerate}
\end{multicols}

\item 求证:
\begin{enumerate}[(1)]
    \item $n!=\frac{(n+1)!}{n+1}$
    \item $\perm{m}{n}=\frac{\perm{n+2}{n+2}}{\perm{2}{n+2}\cdot \perm{n-m}{n-m}}\quad (m\le n)$
\end{enumerate}

\item 某一段铁路共有8个车站,需要为这段铁路印制多少种不不同的客车票?(都是普通座位票)
\item 6个同学排成一列,有多少种不同的排法?
\item 从12人的1个学习小组中,选1个正组长和1个副组长,问有多少种不同的选法?
\item 用0至5共六个数字,可以组成多少不同的三位数?
\item 用0到9这十个数字,其中有多少个没有重复数字的二位数,其中有多少个二位奇数?
\item 某信号兵用红、黄、篮三面旗子从上到下挂在竖直的旗杆上表示信号,每次可以任挂一面、二面或三面,并且不同的顺序表示不同的信号,问一共可以表示多少种不同的信号?
\item 有五道不同的工序分配给五个工人工作,其中有一个工人不会做其中两道工序,问有多少种不同的分配法?
\item 由2, 3, 4, 5所组成的没有重复数字的四位数,
\begin{enumerate}[(1)]
\item 求所有这些四位数里各个数字的和;
\item 求所有这些四位数的和。
\end{enumerate}
\end{enumerate}
\end{ex}

\subsection{组合}
我们知道,从一个学习小组中选出正、副组长各1人,这种选法是排列问题.如果改变为从一个学习小组中选出代表2人参加学校的学代会,有多少种不同的选法?”这时的代表无所谓正、副之分,就无先后顺序的意义了,也就不成为排列的问题了.

又例如从1, 2, 3, 4, 5五个数选出两个数字相乘,问有多少个不同的乘积?那么只要所选出的数字是3和5;它们的积只有15,至于是$5\x3$还有$3\x5$,那是不要求加以区别的,也是没有排列次序的区别的.

一般地说,从$n$个不同的元素中,任取$m\; (m\le n)$个元素并成一组,叫做从$n$个不同元素中取出$m$个元素的一个组合.

这里要注意,元素$a$,$b$,$c$的组合与$a,c,b$; $c,a,b$; $b,a,c$; $b,c,a$或$c,b,a$都被看作是相同的组合了. 对于排列来说,上列的是六个不同的排列,而对于组合来说,只算是一个组合,那么组合数是怎样进行计算的呢?

从$n$个不同元素中取出$m\; (m\le n)$个元素的所有组合的个数,叫做从$n$个不同元素中取出$m$个元素的组合数,用符号$\comb{m}{n}$表示。

例如,从四个不同元素中取出3个元素进行排列,有多少种不同的排法?

我们已经知道应该有$\perm{3}{4}=24$种不同的排法,现在我们
想从另外的方法入手解决这个问题.

第一步,从$a,b,c,d$中取三个元素作一组,这样可以得到$a,b,c$; $a,b,d$; $a,c,d$; $b,c,d$四个组,这就是说$\comb{3}{4}=4$.

第二步,把每组的三个元素作全排列,应有$\perm{3}{3}$个不同排列.

根据乘法原理得$\perm{3}{4}=4\cdot \perm{3}{3}$,即$\perm{3}{4}=\comb{3}{4}\cdot \perm{3}{3}$.

又如,从五个不同的元素中取出2个元素的列排数是$\perm{2}{5}$. 如果我们从五个不同元素中取出两个分成一组,我们可以组成如下的小组:

$a,b$; $a,c$; $a,d$; $a,e$; $b,c$; $b,d$; $b,e$; $c,d$, $c,e$和$d,e$等10个组,这就是$\comb{2}{5}$;第二步把每组元素进行全排列应是$\perm{2}{2}$,这样应该有$\perm{2}{5}=\comb{2}{5}\cdot \perm{2}{2}$. 因此,一般地说,从$n$个不同元素中取出$m\; (m\le n)$个元素的排列数$\perm{m}{n}$可以等于$\comb{m}{n}\cdot \perm{m}{m}$.


\begin{blk}
  {定理2} 从$n$个不同元素中任取$m$个($m\le n$)没有重复元素组合数为$\comb{m}{n}=\frac{n!}{m!(n-m)!}$.  
\end{blk}

\begin{proof}
    由$\perm{m}{n}=\comb{m}{n}\cdot \perm{m}{m}$,所以
\[\comb{m}{n}=\frac{\perm{m}{n}}{\perm{m}{m}}=\frac{\frac{n!}{(n-m)!}}{m!}=\frac{n!}{m!(n-m)!}\]
\end{proof}
注意:$\comb{m}{n}=\frac{n(n-1)\cdots(n-m+1)}{m!}$

\begin{example}
计算$\comb{3}{8}$, $\comb{5}{8}$和$\comb{2}{9}$
\end{example}

\begin{solution}
    根据组合数公式有:
\[\begin{split}
    \comb{3}{8}&=\frac{8\x 7\x 6}{3\x 2\x 1}=56\\
    \comb{5}{8}&=\frac{8\x 7\x 6\x 5\x 4}{5\x 4\x 3\x 2\x 1}=56\\
    \comb{2}{9}&=\frac{9\x 8}{2\x 1}=36
\end{split}\]
\end{solution}

组合数有以下性质:
\begin{blk}
{性质1}$$\comb{m}{n}=\comb{n-m}{n}$$
\end{blk}

\begin{proof}
根据组合数公式:
\[\begin{split}
    \comb{m}{n}&=\frac{n!}{m!(n-m)!}\\
    \comb{n-m}{n}&=\frac{n!}{(n-m)![n-(n-m)]!}=\frac{n!}{m!(n-m)!}
\end{split}\]
所以$\comb{m}{n}=\comb{n-m}{n}$.
\end{proof}

\begin{blk}{性质2}
\[\comb{m}{n+1}=\comb{m}{n}+\comb{m-1}{n}\]
\end{blk}

\begin{proof}
根据组合数公式:
\[\begin{split}
    \comb{m}{n+1}&=\frac{(n+1)!}{m!(n-m+1)!}\\
    \comb{m}{n}+\comb{m-1}{n}&=\frac{n!}{m!(n-m)!}+\frac{n!}{(m-1)!(n-m+1)!}\\
    &=\frac{(n-m+1)\cdot n! +m\cdot n!}{m!(n-m+1)!}=\frac{n!\cdot (n-m+1+m)}{m!(n-m+1)!}\\
    &=\frac{(n+1)!}{m!(n-m+1)!}\\
\end{split}\]
所以$\comb{m}{n+1}=\comb{m}{n}+\comb{m-1}{n}$.
\end{proof}

为了使性质1在$n=m$时也能成立,我们规定$\comb{0}{n}=1$.

\begin{example}
计算$\comb{98}{100}$和$\comb{2}{10}+\comb{3}{10}+\comb{2}{11}+\comb{2}{12}$
\end{example}

\begin{solution}
根据组合数性质1
\[\comb{98}{100}=\comb{100-98}{100}=\comb{2}{100}=\frac{100\x 99}{2\x 1}=4950\]
根据组合数性质2
\[\begin{split}
    \comb{2}{10}+\comb{3}{10}+\comb{2}{11}+\comb{2}{12}&=\comb{3}{11}+\comb{2}{11}+\comb{2}{12}\\
    &=\comb{3}{12}+\comb{2}{12}=\comb{3}{13}\\
    &=\frac{13\x 12\x 11}{3\x 2\x 1}=286
\end{split}\]
\end{solution}

\begin{example}
    解不等式 $\comb{3}{n}>\comb{5}{n}$
\end{example}

\begin{solution}
\[\frac{n(n-1)(n-2)}{3\x 2\x 1}>\frac{n(n-1)(n-2)(n-3)(n-4)}{5\x 4\x 3\x 2\x 1}\]
因为$5\le n$,所以$n,\; n-1,\; n-2,\; n-3,\; n-4$均为正数,所以
\[1>\frac{(n-3)(n-4)}{5\x 4}\]
整理得:$n^2-7n-8<0$

解不等式得:$-1<n<8$,根据题意,得:$n=5,6,7$.
\end{solution}



\begin{example}
高中三个年级,每个年级有六个班。
\begin{enumerate}[(1)]
\item 每级都进行单循环(每班都与级内其他各班比赛一场)篮球预赛,问预赛需要比赛多少场?
\item 从各级预赛中选出每级的冠、亚军,再进行单循环决赛,规定在预赛中已经比赛过的队不再比赛,问决赛需要比赛多少场?
\end{enumerate}
\end{example}

\begin{solution}
\begin{enumerate}[(1)]
    \item 根据题意每个级单循环比赛的场次都是$\comb{2}{6}$,所以
\[3\comb{2}{6}=3\times 15=45\]
\item 决赛共6个班,如果单循环要比赛$\comb{2}{6}$场,但预赛中比赛过的有3场,所以
\[\comb{2}{6}-3=15-3=12\]
\end{enumerate}
答:预赛要比赛45场,决赛要比赛12场.
\end{solution}



\begin{example}
    平面有10个不同的点,其中除4个点在同一直线上外,不再有3个点共线,以每三个点为顶点可以作多少个不同的三角形?
\end{example}


\begin{solution}
    如果10点中的任三点都能连成三角形,那么有$\comb{3}{10}$个三角形. 但是如果在四个共线点中任选三点就不可能构成三角形. 所以
\[\comb{3}{10}-\comb{3}{4}=120-4=116\]
答:可以作116个不同的三角形。
\end{solution}



\begin{example}
    在产品检验时,常从产品中抽出一部分进行检查,假设100件产品中有2件次品,
    \begin{enumerate}[(1)]
        \item 从100件产品中抽出三件检查,其中有1件次品的抽法有多少种?
        \item 从100件产品中抽出4件检查,其中至少有一件次品的抽法有多少种?
    \end{enumerate}
\end{example}

\begin{solution}
\begin{enumerate}[(1)]
    \item 在100件产品中,次品有两件,从2件次品中抽出1件的抽法有$\comb{1}{2}$,从98件正品中抽出2件的方法有$\comb{2}{98}$. 所以
\[\comb{1}{2}\cdot \comb{2}{98}=2\cdot \frac{98\cdot 97}{2}=9506\]
    \item 至少有1件是次品就是包括有1件次品和2件次品两类抽法的和,所以 
\[\comb{1}{2}\cdot \comb{3}{98}+\comb{2}{2}\cdot \comb{2}{98}=2\times \frac{98\cdot 97\cdot 96}{3\cdot 2}+\frac{98\cdot 97}{2}=304192+4753=308945\]
\end{enumerate}
答:3件中有1件是次品的抽法是9506, 4件中至少有1件次品的抽法是308945。
\end{solution}

\begin{example}
    从0, 3, 5, 7, 11五个数中可以得到多少个不同的乘积?
\end{example}

\begin{solution}
    因为零乘以任何数其积都是零,所以不论零与多少个数相乘都只算一个乘积就是0;

在非零的乘积中有$\comb{2}{4}$, $\comb{3}{4}$, $\comb{4}{4}$三类,所以
\[\comb{2}{4}+\comb{3}{4}+\comb{4}{4}+1=6+4+1+1=12\]

答:可以得到12个不同的乘积.
\end{solution}

\begin{ex}
\begin{enumerate}
    \item 写出:
    \begin{enumerate}[(1)]
    \item 从4个相异元素$a$,$b$,$c$,$d$中取出2个元素的所有组合;
    \item 从5个相异元素$a$,$b$,$c$,$d$,$e$中取出3个元素的所有组合.
    \end{enumerate}

    \item 计算:
\begin{multicols}{2}
\begin{enumerate}[(1)]
    \item  $\comb{2}{7}$
    \item $\comb{48}{50}$
    \item $\comb{3}{100}-\comb{3}{90}$
    \item $\comb{1}{5}+\comb{2}{5}+\comb{3}{5}+\comb{4}{5}+\comb{5}{5}$
\end{enumerate}
\end{multicols}
    
\item   平面内有12个点,其中任意3点不在同一直线上,以每3点为顶点作一三角形,一共可以作多少个不同的三角形?
\item 圆上$n$个不同的点,
\begin{enumerate}[(1)]
\item 可以引多少条不同的弦?
\item 可以连接多少个内接三角形?
\end{enumerate}

\item 凸四边形可以连多少条对角线?凸五边形可以连多少条对角线?凸$n$边形可以连多少条对角线?
\item 某班有50名学生,其中正副班长各1人,现选5名学生参加某项活动
\begin{enumerate}[(1)]
\item 如果班长和副班长必须在内,有多少种不同的选法?
\item 如果要有一个班长(正、副都可以)在内,有多少种不同的选法?
\item 如果班长都不参加,有多少种不同的选法?
\item 如果至少要有一个班长(正、副都可以)在内,有多少种不同的选法?
\end{enumerate}
\end{enumerate}
\end{ex}

\section*{习题2.1}
\begin{enumerate}
    \item 计算:
\begin{multicols}{2}
\begin{enumerate}[(1)]
    \item $5\cdot \perm{3}{5}+4\cdot \perm{2}{4}$
    \item $\frac{\perm{5}{7}-\perm{6}{6}}{7!+6!}$
\end{enumerate}
\end{multicols}

\item 求证:
\begin{enumerate}[(1)]
    \item $\frac{(n+1)!}{k!}-\frac{n!}{(k-1)!}=\frac{(n-k+1)\cdot n!}{k!}$
    \item $\frac{(2n)!}{2^n \cdot n!}=1\cdot 3\cdot 5\cdots (2n-1)$
\end{enumerate}

\item 解方程:
\begin{multicols}{2}
\begin{enumerate}[(1)]
    \item $\perm{3}{2n}=10\perm{3}{n}$
    \item $\frac{\perm{5}{n}+\perm{4}{n}}{\perm{3}{n}}=4$
    \item $\perm{4}{2n+1}=140\perm{3}{n}$
\end{enumerate}
\end{multicols}

\item 解不等式:
\begin{multicols}{2}
\begin{enumerate}[(1)]
    \item $\perm{2}{m-2}+m>2$
    \item $\perm{x}{9}>6\perm{x-2}{9}$
\end{enumerate}
\end{multicols}

\item \begin{enumerate}[(1)]
    \item 从多少个不同的元素中取出2个元素的排列数是56?
    \item 已知从$n$个不同的元素中取出2个元素的排列数等于从$n-4$个不同元素中取2个元素的排列数的7倍,求$n$.
\end{enumerate}
\item 有5本不同的书分借给3个同学,每人1本,共有多少种不同的借法?
\item 用0, 1, 2, 3, 4, 5可以组成多少个没有重复数字的五位数?其中有多少个偶数?
\item 用数字1, 2, 3, 4, 5可以组成多少个没有重复数字的自然数?其中有多少个比13000大的自然数?
\item 7部不同型号的机床分配给7个工人使用:
\begin{enumerate}[(1)]
\item 甲必须使用2号机床;
\item 甲必须使用2号机床,乙必须使用4号机床;
\item 甲不适宜使用1号机床,乙不适宜使用7号机床;问三种情况下各有多少种不同的分配方案?
\end{enumerate}

\item 6个儿童站成一排,其中一人不站在排头,也不站在排尾,一共有多少种不同的排法?
\item 五部不同型号的半导体收音机和四部不同型号的电视机陈列成一排,任两部电视机不靠在一起,有多少种不同的陈列方法?
\item 在3000和7000间有多少个没有重复数字的5的倍数?

\item 计算:
\begin{multicols}{2}
\begin{enumerate}[(1)]
    \item $\comb{3}{15}$
    \item $\comb{98}{100}$
    \item $\comb{3}{6}+\comb{6}{8}$
    \item $\comb{n}{n+1}\cdot \comb{n-2}{n}$
\end{enumerate}
\end{multicols}

\item 求证:
\begin{enumerate}[(1)]
    \item $\comb{m}{n+1}=\comb{m-1}{n}+\comb{m}{n-1}+\comb{m-1}{n-1}$
    \item $\comb{k+1}{n}\div \comb{k}{n}=\frac{n-k}{k+1}$
    \item $\comb{0}{m}+\comb{1}{m+1}+\comb{2}{m+2}+\cdots+\comb{n-1}{m+n-1}=\comb{n-1}{m+n}$
    \item $\comb{m+1}{n}+\comb{m-1}{n}+2\comb{m}{n}=\comb{m+1}{n+2}$
\end{enumerate}

\item 解不等式:
\begin{multicols}{2}
\begin{enumerate}[(1)]
    \item $\comb{n}{4}>\comb{n}{6}$
    \item $\comb{n-4}{21}<\comb{n-2}{21}<\comb{n-1}{21}$
\end{enumerate}
\end{multicols}

\item 在200件产品中有2件次品,现在抽5件进行检查:
\begin{enumerate}[(1)]
\item 其中有一件次品的不同抽法有多少种?
\item 没有次品的不同抽法有多少种?
\item 至少有一件次品的不同抽法有多少种?
\end{enumerate}

\item 问2310能被多少个偶数整除?
\item 有8个划船运动员,其中3人能划左舷,2人能划右舷,3人能划左右舷,要选出6人分配在两边,问有多少种不同的选法?

\item  某车间有9名技术熟练的工人,其中2人能当车、钳工,有3人只能当车工,其余4人只能当钳工,现在要抽调三个车工,三个钳工成立一个新车间,问有多少种不同的抽调方法?
\item  空间有12个不同的点,其中有4点在同一平面内,此外不再有4点共面,
\begin{enumerate}[(1)]
\item  这些点可以确定多少个不同的平面?
\item  这些点可以连成多少个不同的四面体?
\end{enumerate}

\item  有8部机器分给陈、李、张三位师傅管理。
\begin{enumerate}[(1)]
\item 陈管4部,李管3部,张管1部,有几种不同分配方法?
\item 陈管4部,其他机器由1人管3部,1人管1部,有多少种不同的分配方法?
\item 1人管4部,一人管3部,一人管1部,有几种不同的分配方法?
\end{enumerate}

\item  从1, 5, 7, 9中任取3个数字,2, 4, 6, 8中任取两个数字,组成没有重复数字的五位数,一共有多少个不同的五位数?
\end{enumerate}

\section{环形排列、重复排列和组合}
\subsection{环形排列}
我们已经学握了从$n$个相异的元素中取出$m\; (m\le n)$个元素的排列的知识了,那么怎样解决环形排列的问题呢?

我们先从最简单的情况入手,就是先找出环形排列与直
线排列的相互关系,然后利用直线排列的知识解决环形排列的问题.

首先在环形排列中,下面的排列都是当作相同的排列的(图2.2).
\begin{figure}[htp]
    \centering
\begin{tikzpicture}[>=stealth]
\begin{scope}
\draw circle(.9);
\node at (90:1.2){甲};
\node at (90+180:1.2){乙};
\draw[->](90-20:1.2) arc (90-20:20:1.2);
\end{scope}
\begin{scope}[xshift=3cm]
\draw circle(.9);
\node at (0:1.2){甲};
\node at (0+180:1.2){乙};
\draw[->](0-20:1.2) arc (0-20:-70:1.2);
\end{scope}
\begin{scope}[xshift=6cm]
\draw circle(.9);
\node at (-90:1.2){甲};
\node at (-90+180:1.2){乙};
\draw[->](-90-20:1.2) arc (-90-20:-160:1.2);
\end{scope}
\begin{scope}[xshift=9cm]
\draw circle(.9);
\node at (45:1.2){甲};
\node at (45+180:1.2){乙};
\draw[->](45-20:1.2) arc (45-20:-20:1.2);
\end{scope}

\end{tikzpicture}
    \caption{}
\end{figure}


因为从甲开始,按同一方向,例如顺时针方向看过去都是乙,所以尽管位置不同,但仍然是同一样的元素的排列顺序.
\begin{figure}[htp]
    \centering
\begin{tikzpicture}[>=stealth]
\begin{scope}
\draw circle(.9);
\node at (90:1.2){甲};
\node at (90-120:1.2){乙};
\node at (90+120:1.2){丙};
\draw[->](90-20:1.2) arc (90-20:20:1.2);
\end{scope}
\begin{scope}[xshift=3.5cm]
\draw circle(.9);
\node at (0:1.2){甲};
\node at (0-120:1.2){乙};
\node at (0+120:1.2){丙};
\draw[->](0-20:1.2) arc (0-20:-70:1.2);
\end{scope}
\begin{scope}[xshift=7cm]
\draw circle(.9);
\node at (-90:1.2){甲};
\node at (-90-120:1.2){乙};
\node at (-90+120:1.2){丙};
\draw[->](-90-20:1.2) arc (-90-20:-160:1.2);
\end{scope}
  
\end{tikzpicture}
    \caption{}
\end{figure}


又例如三个不同元素甲、乙、丙如图2.3的三个环形排列也都算是相同的排列,而且与从谁开始无关,即甲$\to $乙$\to $丙$\to $甲、乙$\to $丙$\to $甲$\to $乙与丙$\to $甲$\to $乙$\to $丙,也都算作是相同的环形排列.

从上述的例子,可以得到三个不同元素的环形排列数应该是2.具体的排列是:甲—乙—丙—甲与甲—丙—乙—甲.

现在,我们再来分析一下环形排列与直线排列不同的特点. 我们已经知道下列三个环形排列都是相同的排列(图2.4)
\begin{figure}[htp]
    \centering
\begin{tikzpicture}[>=stealth]
\begin{scope}
\draw circle(.9);
\node at (90:1.2){甲};
\node at (90-120:1.2){乙};
\node at (90+120:1.2){丙};
\draw[->](-.35,1.5)--(-.35,1);
\end{scope}
\begin{scope}[xshift=3.5cm]
    \draw circle(.9);
    \node at (90:1.2){丙};
    \node at (90-120:1.2){甲};
    \node at (90+120:1.2){乙};
    \draw[->](-.35,1.5)--(-.35,1);

    \end{scope}
\begin{scope}[xshift=7cm]
    \draw circle(.9);
    \node at (90:1.2){乙};
    \node at (90-120:1.2){丙};
    \node at (90+120:1.2){甲};
    \draw[->](-.35,1.5)--(-.35,1);

    \end{scope}
\end{tikzpicture}
    \caption{}
\end{figure}

如果我们从图上所示的地方把环切断,并把环拉成直线后,我们却可以得到三个不同的直线的排列,就是:
\[\text{甲、乙、丙;丙、甲、乙;乙、丙、甲}\]

而三个不同元素的全排列数是$\perm{3}{3}$,所以三个不同元素的环形排列数是$\frac{\perm{3}{3}}{3}$

\begin{blk}
    {定理3} $m$个不同元素的环形排列数就是:
\[\frac{\perm{m}{m}}{m}=\frac{m!}{m}=(m-1)!\]
\end{blk}

\begin{proof}
    
\end{proof}
设有$m$个不同元素排成一个环形,如图(图2.5)
\begin{figure}[htp]
    \centering
\begin{tikzpicture}
\draw(0,0) circle(1);
\draw[fill](15:1)node[right]{$a_4$} circle(1pt);
\draw[fill](15+30:1)node[above right]{$a_3$} circle(1pt);
\draw[fill](15+60:1)node[above]{$a_2$} circle(1pt);
\draw[fill](15+90:1)node[above]{$a_1$} circle(1pt);
\draw[fill](15+120:1)node[above left]{$a_m$} circle(1pt);
\draw[fill](15+150:1)node[left]{$a_{m-1}$} circle(1pt);
\end{tikzpicture}
    \caption{}
\end{figure}
就是按顺时针方向的一个排列.其次,$a_1$不动,改变$a_2,a_3,\ldots,a_m$的顺时针排列顺序,就得到第二个不同的环形排列.因此,如果$a_1$不动,要使$a_2,a_3,\ldots,a_m$有不同的排
列,其不同的排列数为$(m-1)!$,故$m$个不同元素的环形排列数为$\perm{m-1}{m-1}=(m-1)!$


\begin{blk}
  {定理4} 从$n$个不同的元素中,任取$m$个($m\le n$)没有重复的元素的环形排列数$\comb{m}{n}\cdot (m-1)!$.  
\end{blk}

\begin{proof}
    从$n$个不同元素中取出$m\; (m\le n)$个元素进行环形排列,我们第一步可以先从$n$个元素中先取出$m$个组成一组,第二步是把$m$个元素排成一圈,根据乘法原理,其排列数应为:
\[\comb{m}{n}\cdot \perm{m-1}{m-1}=\comb{m}{n}(m-1)!\]
\end{proof}

\begin{example}
    从10个不同元素中取出6个元素排成一圈,有多少种不同的排法?
\end{example}

\begin{solution}
    从10个不同元素中取出6个的环形排列数为:
\[\comb{6}{10}\cdot\perm{6-1}{6-1}=\comb{4}{10}\cdot  5!=210\cdot 120=25200\]
答:有25200种不同的排法.
\end{solution}

\begin{example}
    有五个男同学与五个女同学围成圆圈跳集体舞,如果男女相间,问有几种不同的排列方法?
\end{example}

\begin{solution}
第一步,五个男同学先作环形排列,其排列数为$4!$;

第二步,五个女同学分别站在两个男同学之间的位置上作全排列.

$\therefore\quad 4! \cdot 5! =2880$

答:有2880种不同的排列方法.    
\end{solution}

\begin{ex}
\begin{enumerate}
\item 把10个人平均分成两组,每组都围坐一个圆桌,问有多少种不同的坐法?
\item 今有5颗不同颜色的珠子,用线把它们串成珠圈,间有多少种串法?
\end{enumerate}
\end{ex}

\subsection{重复排列}

前面我们所研究的都是不允许元素在排列和组合中重复出现的,假如我们允许元素在同一排列中重复出现的话,其排列数应该是多少呢?

\begin{blk}
{定理5} 从$n$个不同的元素中,取出$m$个元素排成一列(允许元素重复出现)的不同排列数为$n^m$.
\end{blk}

\begin{proof}
    选在第一个位置上的元素有$n$个不同的方法,选在第二个位置上的元素仍然有$n$个不同的方法,……,选在第$m$个位置上的元素仍然有$n$个选法,所以根据乘法原理,从$n$个不同的元素中取出$m$个元素的允许重复的排列数应为$n^m$.
\end{proof}

这种排列又称为重复排列,这里应该注意的是,由于元素可以重复选取,因此选取的次数$m$就不会再受$n$个元素的限制. 即$m$不再有$m\le n$的限制了.

\begin{example}
    电报通讯的明码是用0到9十个数字中的每4个数字的重复排列代表一个汉字,问用这种方法制成的电报通讯明码可以表示多少个不同的汉字?
\end{example}

\begin{solution}
    从10个不同数字中选取4个数字的重复排列数是:
\[10^4=10000\]
答:可以表示10000个不同的汉字.
\end{solution}

\begin{example}
    由1, 2, 3, 4四个数字能够组成多少个大于2300的四位数?
\end{example}

\begin{solution}
    这里的四位数显然是包括重复数字的四位数.

第一类:3在千位上的四位数一定大于2300,那么,百、十、个位上的数字每次都有4种选取可能,所以应有$4^3$个四位数.

第二类:4在千位上的四位数,同理也有$4^3$个四位数。

第三类:2在千位上,3在百位上的四位数,由于十位和个位上的数都可以从1, 2, 3, 4中选取,都大于2300,所以,这样的数有$4^2$个.

第四类,2在千位上,4在百位上的四位数有$4^2$个.所以
\[2\cdot 4^3+2\cdot 4^2=2\cdot 4^2(4+1)=160\]

答:这样的四位数有160个.
\end{solution}

\begin{ex}
    \begin{enumerate}
\item 有三封不同的信,投入4个信箱里,一共有多少种投信的方法?
\item 有数学、天文、诗歌三个课外小组,8个同学准备报名,若每人限报一个组,问一共可有多少种报名的方法?
    \end{enumerate}
\end{ex}

\subsection{重复组合}
从$n$个不同元素中取出$m$个元素组成一组,如果允许元素在组合中重复出现,这样的组合简称为重复组合,这种重复组合用符号$\rep{m}{n}$表示.

怎样计算重复组合的组合数呢?我们还是先从具体例子开始考虑,假如有三个不同的元素$a_1,a_2,a_3$,从其中取出两个元素组成重复组合,那么,可以得到如下的六个不同组合:
\begin{equation}
a_1a_1,\; a_1a_2,\; a_1a_3,\; a_2a_1,\; a_2a_3,\; a_3a_3 \tag{1}
\end{equation}

如果是不允许元素重复的话,要选两个元素得到不同的组合数是6的话,应是$\comb{2}{4}$,设这四个元素为$a'_1,a'_2,a'_3,a'_4$,那么得到的组合为:
\begin{equation}
a'_1a'_2,\; a'_1a'_3,\; a'_1a'_4,\; a'_2a'_3,\; a'_2a'_4,\; a'_3a'_4 \tag{2}
\end{equation}
组合(1)与组合(2)不难建立它们之间的一一对应关系.在组合(1)的每个组合的元素下标上分别加上0和1,就得到了下标与(2)完全一致的组合(3).
\begin{equation}
a_{1+0}a_{1+1},\quad a_{1+0}a_{2+1},\quad a_{1+0}a_{3+1},\quad a_{2+0}a_{2+1},\quad a_{2+0}a_{3+1},\quad a_{3+0}a_{3+1}
\tag{3}
\end{equation}

又例如从4个不同元素中取出3个元素的重复组合有如下的不同组合:
\begin{center}
\begin{tabular}{ccccc}
    $a_{1}a_{1}a_{1}$&$a_{1}a_{1}a_{2}$&$a_{1}a_1a_{3}$&$a_{1}a_{1}a_{4}$\\
    &$a_{1}a_{2}a_{2}$&$a_{1}a_{2}a_{3}$&$a_{1}a_{2}a_{4}$\\
    &$a_{1}a_{3}a_{3}$&$a_{1}a_{3}a_{4}$&$a_{1}a_{4}a_{4}$\\
    &$a_{2}a_{2}a_{2}$&$a_{2}a_{3}a_{3}$&$a_{2}a_{2}a_{4}$\\
    &$a_{2}a_{3}a_{3}$&$a_{2}a_{3}a_{4}$&$a_{2}a_{4}a_{4}$\\
 &$a_{3}a_{3}a_{3}$&$a_{3}a_{3}a_{4}$&$a_{3}a_{4}a_{4}$\\
   &$a_{3}a_{3}a_{3}$&$a_{3}a_{3}a_{4}$&$a_{3}a_{4}a_{4}$&$a_{4}a_{4}a_{4}$\\ 
\end{tabular}
\end{center}

只要我们在每组元素的下标上分别加0, 1, 2, 就得到
\begin{center}
\begin{tabular}{ccccc}
    $a_{1}a_{2}a_{3}$&$a_{1}a_{2}a_{4}$&$a_{1}a_{2}a_{5}$&$a_{1}a_{2}a_{6}$\\
 & $a_{1}a_{3}a_{4}$&$a_{1}a_{3}a_{5}$&$a_{1}a_{3}a_{6}$\\
 &$a_{1}a_{4}a_{5}$&$a_{1}a_{4}a_{6}$&$a_{1}a_{5}a_{6}$\\
 &$a_{2}a_{3}a_{4}$&$a_{2}a_{3}a_{5}$&$a_{2}a_{3}a_{6}$\\
 &$a_{2}a_{4}a_{5}$&$a_{2}a_{4}a_{6}$&$a_{2}a_{5}a_{6}$\\
 &$a_{3}a_{4}a_{5}$&$a_{3}a_{4}a_{6}$&$a_{3}a_{5}a_{6}$&$a_{4}a_{5}a_{6}$
\end{tabular}
\end{center}

这又正是从六个不同元素中取出3个元素的组合的全部。
从上面的例子说明了
\[{\rm H}_{3}^{2}= {\rm C}_{3+ ( 2- 1) }^{2}, \qquad {\rm H}_{4}^{3}={\rm C}_{4+ ( 3- 1) }^3\]

按同样的方法我们可以得出:从$n$个不同元素中取出$m$
个的重复组合数有如下的关系:
$${\rm H}_{n}^{m}= {\rm C}^m_{n+ ( m- 1) }$$

现在我们来证明这个结论.

\begin{blk}
 {定理6} 从$n$个不同元素中,任取$m$个元素 的重复组合为:
 $${\rm H}_{n}^{m}={\rm C}_{n+ ( m-1) }^{m}$$ 
\end{blk}

\begin{proof}
    为了研究问题方便,我们用$1,2,3,\ldots,n$表示这$n$个不同的元素。在其中任取$m$个允许重复的数字作为一个组合,并按大小顺序排成
$$1\le  i_2\le i_2\le\cdots\le i_m\le n$$

    又记$j_1=i_1+0,\; j_{2}=i_{2}+1,\;\ldots, j_{m}=i_{m}+(m-1)$
    
    自然,$1\le j_1<j_2<j_3<\cdots<j_m\leq n+(m-1)$,所以$\{j_1,j_2,\ldots,j_m\}$ 是$n+ m- 1$个 数,并 且 是$1, 2, \ldots , n+m-1$中的$m$个不同数的组合.
\end{proof}

下面来证明:在$n$个不同元素中任取$m$个允许重复的元
素的组合,和$n+m-1$个不同元素中任取$m$个不同元素 的组合之间有一个到上的一一对应关系。如果这点证明了,那么$n$个不同元素的$m$个允许重复元素的组合数就与$n+m-1$个不同元素的$m$个不同元素的组合数相等了.

\begin{proof}
    
为了证明这个一一对应,要先证明当允许重复的组合
$\{i_1,i_2,\ldots,i_m\}$与$\{i_1^{\prime},i_2^{\prime},\ldots,i'_m\}$不同 的时候,
(这里取$1\le i_1\le i_2\le\cdots\le i_m\le n$, $1\le i'_1\le i'_2\le\cdots\le i^{\prime}_m\le n$)那么,不同元素不允许重复的组合$\{j_1,i_2,\ldots, i_m\}$与$\{j_1^{\prime},j_2^{\prime},\ldots,j'_m\}$也不同,这里$\{j_1=i_1+0,\; j_2=i_{2}+1,\ldots,j_{m}=i_{m}+m-1,\; j_{1}^{\prime}=i_{1}^{\prime}+0,\; j_{2}^{\prime}=i_{2}^{\prime}+1,\ldots, j_m^{\prime}=i_m^{\prime}+m-1\}$.

事实上,如果后者相同,即有$j_{k}= j_{k}\; (k= 1, 2, \ldots, m)$ 即
$$i_k+k-1=i_k^{\prime}+k-1\; (k=1,2,\ldots,m)$$ 
所以$i_k= i_k^{\prime }\; ( k= 1, 2, \ldots , m)$, 也就是说$\{i_1, i_2,\ldots,i_m\}$也与$\{i_1^{\prime},i_2^{\prime},\ldots,i_m^{\prime}\}$相同了。这就证明了上述的对应为一一对应.

再证明这是到上的一一对应,即当$\{i_1,i_2,\ldots,i_m\}$取遍$n$个数$1,2,\ldots,n$的一切$m$个数的允许重复组合时,$\{j_i, j_2,\ldots,j_m\}$ 也取遍$n+m-1$个 数 $1,2,\ldots ,n+m-1$一切的 $m$ 个数的不允许重复的组合.

事实上,对任一个组合$\{j_1,j_2,\ldots,j_m\}$, 其中$1\le j_{1}<j_{2}<\cdots<j_{m}\le n+m-1$. 有$1\leq j_{1},\; 2\le j_{2},\; 3\le j_{3},\ldots, m\le j_{m}$. 故$1\le i_1=j_1-0,\; 1\le i_2=j_2-1,\; 1\le i_3=j_3-2,\ldots, 1\leq i_m=j_m-(m-1)$. 即$i_1,i_2,\ldots,i_m$都是自然数,且由$j_{k}>j_{k-1}$, 可知$j_k-j_{k-1}\ge 1$, 故有
\[i_k-i_{k-1}=[j_k-(k-1)]-[j_{k-1}-(k-2)]=j_k-j_{k-1}-1\ge 0\]
再者$i_{m}=j_{m}-(m-1)\le (n+m-1)-(m-1)=n$. 总之有$1\le i_1\le i_2\le \cdots \le i_m\le n$. 所以,$\{i_1,i_2,\ldots,i_m\}$ 是$n$个数$1,2,\ldots,n$中取$m$个数的允许重复的组合,这证明了对应是到上的一一对应.

\end{proof}


\begin{example}
  从3, 5, 7, 9, 11中取出3个数(可以重复选取)相乘,问有多少个不同的乘积?  
\end{example}

\begin{solution}
    因为是乘积,与元素取出的先后顺序无关,同时是可以重复选取的,即是重复组合问题. 所以
\[{\rm H}^3_5=\comb{3}{5+(3-1)}=\comb{3}{7}=35\]
答:有35个不同的乘积.
\end{solution}

\begin{example}
    从三个不同的质数$a_1,a_2,a_3$中取出4个数相乘(可以重复选取),问有多少个不同的乘积?
\end{example}

\begin{solution}
    同例2.16是重复组合问题.所以
    \[{\rm H}^4_3=\comb{4}{3+(4-1)}=\comb{4}{6}=\comb{2}{6}=15\]
答:有15个不同的乘积。
\end{solution}

\begin{example}
    用七种不同的颜色给一个四面体的四个面涂上颜色,规定每个面上只能涂一种颜色,如果四个面的颜色容许相同,问有几种不同的着色方法?
\end{example}

\begin{solution}
    因为只研究对四个面所着的颜色,对那个面先着色并无先后顺序之分,故是组合问题.又因为对四个面所着的色是允许相同的,所以又是重复组合问题. 所以
    \[{\rm H}^4_7=\comb{4}{7+(4-1)}=\comb{4}{10}=210\]
    答:共有210种不同的着色方法。
\end{solution}


\begin{ex}
\begin{enumerate}
\item 展开$(a+b+c+d)^3$,合并同类项后,一共有多少项?
\item 若把六种不同的溶液注满四个烧杯,但不使它们混合,问一共有多少种方法?
\end{enumerate}
\end{ex}

\section*{习题2.2}
\begin{enumerate}
\item 从8个不同的元素中取出5个元素排成圆圈,问有几种不同的排列方法?
\item 两个教师和五个学生代表围坐在圆桌周围开座谈会:
\begin{enumerate}[(1)]
\item 如果不加条件限制,有多少种不同的坐法?
\item 如果两个教师靠在一起,有几种坐法?
\item 如果两个教师不靠在一起,有几种坐法?
\item 学生代表甲坐在教师的中间,有几种坐法?
\end{enumerate}
\item 从三个不同的元素中取出4个元素的重复组合有多少种不同的方法?
\item 用七种颜色涂在正方体的六个面上,如果每个面只能涂一种颜色,如果六个面上的颜色允许相同,问共有几种不同的着色方法?
\item 从2, 3, 5, 7, 11, 13中至少取出两个,至多可以出五个数相乘(允许因数重复选取),问有多少个不同的乘积?
\item 展开$(a+b+c+d+e+f)^4$,再合并同类项,一共可以得到多少项?
\end{enumerate}

\section{二项式定理}

\subsection{和号$\Sigma$}

我们经常要计算下面各种和
$$1+ 2+ 3+ \cdots + n;\quad 1^{2}+ 2^{2}+ 3^{2}+ \cdots + n^{2},\quad \frac{1\times2}{2}+\frac{2\times3}{2^{2}}+\cdots+\frac{n(n+1)}{2^{n}}$$
等等。 一 般 给 出 一 个 有 顺 序 的 数 列 $a_1, a_2, \ldots , a_n$, 要计算
$a_{1}+a_{2}+\cdots+a_{n}$,
为了方便起见引进和符号$\Sigma$,于是可记
$$1+2+3+\cdots+n=\sum_{k=1}^{n}k$$
\[1^{2}+ 2^{2}+ 3^{2}+ \cdots + n^{2}=\sum^n_{k=1}k^2\]
\[\frac{1\times2}{2}+\frac{2\times3}{2^{2}}+\cdots+\frac{n(n+1)}{2^{n}}=\sum^n_{k=1}\frac{k(k+1)}{2^k}\]

一般可记
$a_1+a_2+\cdots+a_n=\sum_{k=1}^{n}a_k$
符号
$$\sum_{k=1}^{n}a_k$$

表示$k$取$1,2,\ldots,n-1,n$时,得到的$a_k$全部加起来.
例如:
$$\sum_{k=1}^{n}\sin kx=\sin x+\sin2x+\sin3x+\cdots+\sin nx$$
$$\sum _{k= 0}^{n}a_{k} x^{n- k}= a_{0} x^{n}+ a_{1} x^{n- 1}+ \cdots + a_{n- 1} x+ a_{n}$$
又例如,对给定的函数$f_1(x),f_2(x),\ldots,f_n(x)$
有
\[\begin{split}
\sum^n_{k=0}f_k(x)&=f_1(x)+f_2(x)+\cdots +f_n(x)
\\
\sum^n_{k=0}\comb{k}{n} &=\comb{0}{n}+\comb{1}{n}+\comb{2}{n}+\cdots +\comb{n}{n}
\end{split}\]
等等.

关于求和符号还有两点说明,首先,凡是指示指标从1到$n$,且仅仅指示从1到$n$.所以可以用其它的字母来代替$k$而和号仍然是表示了同一个意思. 例如
\[\sum^n_{k=1}a_k=a_1+a_2+\cdots+a_n,\qquad \sum^n_{\ell=1}a_{\ell}=a_1+a_2+\cdots+a_n\]

一般$\sum\limits^n_{*=1}a_*=a_1+a_2+\cdots+a_n$
这里$*$可以用任何一种指标来表示,其次,求和也不限于从1到$n$,也可以从0到$n+2$等等. 例如
\[\begin{split}
    \sum^{n+2}_{k=0}a_k&=a_0+a_1+\cdots+a_{n+1}+a_{n+2}\\
    \sum^{n+1}_{k=3}a_k&=a_3+a_4+\cdots+a_{n+1}
\end{split}\]

\begin{ex}
\begin{enumerate}
    \item 把下列各式用和号表示:
\begin{enumerate}[(1)]
\item $1+3+5+7\cdots+(2n-1)$ 
\item $1\cdot2+2\cdot3+3\cdot4+\cdots+n(n+1)$ 
\item $\cos x+\cos2x+\cos3x+\cdots+\cos nx$
\item $1^{2}+2^{2}+3^{2}+\cdots+(n+2)^{2}$
\end{enumerate}

\item 把下列和号式改写为一般的和式
\begin{multicols}{2}
\begin{enumerate}[(1)]
\item $\sum\limits _{k= 1}^{n}\frac 1{k( 2k+ 1) }$ 
\item $\sum\limits _{k= 1}^{n+ 2}\frac {k^{2}}{k! }$
\item $\sum\limits^n_{k=2}\frac{\sin kx}{k}$
\item $\sum\limits^n_{k=2}\frac{(-1)^k}{k^2+k-2}$
\end{enumerate}
\end{multicols}
\end{enumerate}
\end{ex}

\subsection{二项式定理}
下面我们用数学归纳法来证明二项式定理。我们已经知道:
\[\begin{split}
(x+y)^1&=x+y\\
(x+y)^2&=x^2+2xy+y^2\\
(x+y)^3&=x^3+3x^2y+3xy^2+y^3    
\end{split}\]

现在考虑二项式$(x+y)^n$的展开式,这里$n$是任意自然数,首先研究$(x+y)^4$
\[(x+y)^4=(x+y)(x+y)(x+y)(x+y)\]
等号右边的积应有下面形式的各项:
\[x^4,\; x^3y,\; x^2y^2,\; xy^3,\; y^4\]
显然$x^4$, $y^4$的系数是1,其他三项的系数,例如$x^2y^2$的系数是从四个括弧中两个里取$y$,从其他两个里取$x$作出的积的个
数,从四个括弧中任取两个$y$的取法 有$\comb{2}{4}=\frac{4\cdot 3}{1\cdot 2}=6$种,因
此$xy^2$的系数是6,同样,可以求得$x^3y,xy^3$的系数分别
是$\comb{1}{4}=4$,因为规定$\comb{0}{n}=1$,又知$\comb{4}{4}=1$,所以
\[(x+y)^4=\comb{0}{4}x^4+\comb{1}{4}x^3y+\comb{2}{4}x^2y^2+\comb{3}{4}xy^3+\comb{4}{4}y^4\]

一般地,可以归纳出以下的定理:
\begin{blk}
    {定理} 对于两个不定元$x$,$y$的多项式$(x+y)^n$,有下列展开式:
\[\begin{split}
    (x+y)^n&=\sum^n_{k=0}\comb{k}{n}x^{n-k}y^k\\
    &=\comb{0}{n}x^n+\comb{1}{n}x^{n-1}y+\comb{2}{n}x^{n-2}y^2+\cdots+\comb{n}{n}y^n
\end{split}\]
\end{blk}

\begin{proof}
    用数学归纳法来证明定理.
\begin{enumerate}[(1)]
    \item 当$n=1$时,$(x+y)^1=x+y$
\[\sum^n_{k=0}\comb{k}{1}x^{1-k}y^k=\comb{0}{1}x^1y^0+\comb{1}{1}x^{0}y^1=x+y\]
故$n=1$时原命题成立.
\item 假设$n=\ell$时有公式 $(x+y)^{\ell}=\sum\limits^\ell_{k=0}\comb{k}{\ell}x^{\ell-k}y^k$

现在来计算$(x+y)^{\ell+1}$,令
\[\begin{split}
    (x+y)^{\ell+1}&=(x+y)\cdot (x+y)^{\ell}=(x+y)\cdot \sum^\ell_{k=0}\comb{k}{\ell}x^{\ell-k}y^k\\
&=\sum^\ell_{k=0}\comb{k}{\ell}x^{\ell-k+1}y^k +\sum^\ell_{k=0}\comb{k}{\ell}x^{\ell-k}y^{k+1}\\
&=\comb{0}{\ell}x^{\ell+1}+\sum^\ell_{k=1}\comb{k}{\ell}x^{1+\ell-k}y^k+\sum^{\ell-1}_{k=0}\comb{k}{\ell}x^{\ell-k}y^{k+1}+\comb{\ell}{\ell}y^{\ell+1}\\
&=\comb{0}{\ell+1}x^{\ell+1}+\sum^\ell_{k=1}\comb{k}{\ell}x^{1+\ell-k}y^k+\sum^{\ell}_{k=1}\comb{k-1}{\ell}x^{\ell+1-k}y^{k}+\comb{\ell+1}{\ell+1}y^{\ell+1}\\
&=\comb{0}{\ell+1}x^{\ell+1}+\sum^\ell_{k=1}\left(\comb{k}{\ell}+\comb{k-1}{\ell}\right)x^{\ell+1-k}y^k+\comb{\ell+1}{\ell+1}y^{\ell+1}\\
&=\comb{0}{\ell+1}x^{\ell+1}+\sum^\ell_{k=1}\comb{k}{\ell+1}x^{\ell+1-k}y^k +\comb{\ell+1}{\ell+1}y^{\ell+1}\\
&=\sum^{\ell+1}_{k=0}\comb{k}{\ell+1}x^{\ell+1-k}y^k 
\end{split} \]
所以$n=\ell+1$时命题成立.
\end{enumerate}
根据(1)和(2),对于一切自然数$n$都有:$(x+y)^n=\sum\limits^n_{k=0}\comb{k}{n}x^{n-k}y^k$成立.
\end{proof}

这个定理叫做二项式定理,右边的多项式$\sum\limits^n_{k=0}\comb{k}{n}x^{n-k}y^k$
叫做$(x+y)^n$的二项展开式. 式中的$\comb{k}{n}x^{n-k}y^k$叫做二项展开式的通项,用${\rm T}_{k+1}$表示,即通项为展开式的第$k+1$项.
\[{\rm T}_{k+1}=\comb{k}{n}x^{n-k}y^k\]

由二项式定理,取$y=1$,有
\[(x+1)^n=\sum^n_{k=0}\comb{k}{n}x^{n-k}\]
取$y=-1$,有$$(x-1)^n=\sum\limits^n_{k=0}(-1)^k\comb{k}{n}x^{n-k}$$

在二项式定理中,如果遇到$n$是较小的正整数时展开式的系数,也可以直接用下表计算,
\begin{center}
\begin{tikzpicture}
\foreach \x in {0,1,2,...,6}
{
    \node at (0,-\x){$(x+y)^{\x}$};
}
\foreach \x/\y in {3/1, 4/6, 5/15, 6/20, 7/15, 8/6, 9/1}
{
    \node (A\x) at (\x,-6){\y};
}
\foreach \x/\y in {3/1, 4/5, 5/10, 6/10, 7/5, 8/1}
{
    \node (B\x) at (\x+.5,-5){\y};
}
\foreach \x/\y in {3/1, 4/4, 5/6, 6/4, 7/1}
{
    \node (C\x) at (\x+1,-4){\y};
}
\foreach \x/\y in {3/1, 4/3, 5/3, 6/1}
{
    \node (D\x) at (\x+1.5,-3){\y};
}
\foreach \x/\y in {3/1, 4/2, 5/1}
{
    \node (E\x) at (\x+2,-2){\y};
}
\foreach \x/\y in {3/1, 4/1}
{
    \node (F\x) at (\x+2.5,-1){\y};
}
\node (G) at (6,0){1};

\draw(F3)--(E4)--(F4);
\draw(E3)--(D4)--(E4)--(D5)--(E5);
\draw(D3)--(C4)--(D4)--(C5)--(D5)--(C6)--(D6);
\draw(C3)--(B4)--(C4)--(B5)--(C5)--(B6)--(C6)--(B7)--(C7);
\draw(B3)--(A4)--(B4)--(A5)--(B5)--(A6)--(B6)--(A7)--(B7)--(A8)--(B8);

\draw[dotted](1,0)--(G);
\draw[dotted](1,-1)--(F3);
\draw[dotted](1,-2)--(E3);
\draw[dotted](1,-3)--(D3);
\draw[dotted](1,-4)--(C3);
\draw[dotted](1,-5)--(B3);
\draw[dotted](1,-6)--(A3);


\end{tikzpicture}
\end{center}
表中每行两端都是1,而且除1以外的每一个数都等于它肩上两个数的和,这个表叫做杨辉三角,它首先载于我国宋朝数学家杨辉1261年所著的《详解九章算法》一书.

\begin{example}
展开$\left(1{+\frac{1}{x}}\right)^5$
\end{example}


\begin{solution}
\[\begin{split}
    \left(1{+\frac{1}{x}}\right)^5 &=1+5\left(\frac{1}{x}\right)+10\left(\frac{1}{x}\right)^2+10\left(\frac{1}{x}\right)^3+5\left(\frac{1}{x}\right)^4+\left(\frac{1}{x}\right)^5\\
    &=1+\frac{5}{x}+\frac{10}{x^2}+\frac{10}{x^3}+\frac{5}{x^4}+\frac{1}{x^5}
\end{split}\]
\end{solution}

\begin{example}
    展开$\left(2\sqrt{x}-\frac{1}{\sqrt{x}}\right)^6$
\end{example}

\begin{solution}
\[\begin{split}
    \left(2\sqrt{x}-\frac{1}{\sqrt{x}}\right)^6&=\left(\frac{2x-1}{\sqrt{x}}\right)^6 =\frac{1}{x^3}(2x-1)^6\\
&=\frac{1}{x^3}\left[(2x)^6-\comb{1}{6}(2x)^5+\comb{2}{6}(2x)^4-\comb{3}{6}(2x)^3+\comb{4}{6}(2x)^2\right.\\
&\qquad \qquad \left.-\comb{5}{6}(2x)+\comb{6}{6}\right]\\
&=\frac{1}{x^3}\left[64x^6-6\cdot 32x^5+15\cdot 16x^4-20\cdot 8 x^3+15\cdot 4x^2\right.\\
&\qquad \qquad \left.-6\cdot 2x+1\right]\\
&=64x^3-192x^2+240x-160+\frac{60}{x}-\frac{12}{x^2}+\frac{1}{x^3}
\end{split}\]
\end{solution}

\begin{example}
    求$(x+a)^{12}$的展开式中的第四项和倒数第四项。
\end{example}

\begin{solution}
    展开式的通项公式$\mathrm{T}_{k+1}=\mathrm{C}_n^k x^{n-k}y^k$。所以
$$\mathrm{T}^{4}=\mathrm{C}_{12}^{3} x^{12-3} a^{3}=\frac{12\times11\times 10}{3\times2}x^{9} a^{3}=220x^{9} a^{3}$$
又展开式共$12+1=13$, 故倒数第四项是
$$\mathrm{T}_{10}=\mathrm{C}_{12}^{9} x^{12-9} a^{9}=\mathrm{C}_{12}^{3}x^{3} a^{9}=220x^{9} a^{3}$$
\end{solution}

\begin{example}
  求$(x-\frac{1}{x})^{9}$展开式中$x^3$的系数,有设有$x^{0}$项,如
果有,试求出该项来。  
\end{example}

\begin{solution}
    展开式的通项是:
$$T_{k+1} = ( -1 )^{k} C_{9}^{k} x^{9-k} (\frac{1}{x})^{k}=( -1)^{k} C_{9}^{k} x^{9-k-k}$$
依题意有,$9-2k=3$, 所以$k=3$. 因此$x^3$的项系数是
$$( -1 )^{3}{\rm C}_{9}^{3}=- \frac{9\times8\times7}{3\times2}=- 84$$

又若$9-2k=0$, 即$2k=9$, 方程没有整数解,故展开式
里没有$x^0$项.
\end{solution}

\begin{example}
   在$\left(a+\frac1a\right)^{2n}$的展开式里,已知第四项和第六项
系数相等,求展开式里不含有$a$的项. 
\end{example}

\begin{solution}
因为
\[\begin{split}
    {\rm T}_4&=\comb{3}{2n}a^{2n-3}\frac{1}{a^2}=\comb{3}{2n}a^{2n-4}\\
    {\rm T}_6&=\comb{5}{2n}a^{2n-5}\frac{1}{a^5}=\comb{5}{2n}a^{2n-10}
\end{split}\]
依题意得$\comb{3}{2n}=\comb{5}{2n}$,即$\comb{2n-3}{2n}=\comb{5}{2n}$,所以
\[2n-3=5,\quad 2n=8,\quad n=4\]
故所给的二项式是$\left(a+\frac{1}{a}\right)^8$

因为${\rm T}_{k+1}=\comb{k}{8}a^{8-k}\frac{1}{a^k}=\comb{k}{8}a^{8-2k}$,
若${\rm T}_{k+1}$为不含$a$项,则$8-2k=0$,所以$k=4$.
\[{\rm T}_5=\comb{4}{8}\cdot a^0=\frac{8\x 7\x 6\x 5}{4\x 3\x 2}=70\]
\end{solution}

\begin{ex}
\begin{enumerate}
    \item 展开$(x^{3}-2x)^{7}$
    \item 求$(a+\sqrt{b})^{12}$的展开式里的第4项和第9项
    \item \begin{enumerate}[(1)]
        \item 求$(x+a)^{12}$的展开式里的中间一项;
        \item 求$\left(\frac x2+\frac2x\right)^7$的展开式里的中间两项.
    \end{enumerate}  
    \item 在$(x+a)^{10}$的展开式里,求
\begin{multicols}{2}
    \begin{enumerate}[(1)]
        \item 含有$a^{8}$的项;
        \item 含有$x^8$的项. 
    \end{enumerate}    
\end{multicols}

    \item 在$\left(x^{2}+\frac{1}{x^3}\right)^{15}$的展开式里,有没有
\begin{multicols}{2}
    \begin{enumerate}
        \item 不含$x$的项?
        \item 含有$x^{8}$的项?
    \end{enumerate}
\end{multicols}
如果有,把这项求出来.
    
\item 求$\left(3x+\frac{1}{\sqrt[3]{x}}\right)^{8}$的展开式里的常数项. 
\item 已知$\left(\sqrt[8]{a^{2}}+\frac1a\right)^{n}$的展开式里的第3项含有$a^2$,    
    求$n$.
\end{enumerate}
\end{ex}

\subsection{二项式的性质}

我们研究二项式$(x+y)^n$的展开式中各项系数的性质。二项展开式各项系数是:
\[\comb{0}{n},\;\comb{1}{n},\;\comb{2}{n},\;\ldots,\; \comb{n-1}{n},\;\comb{n}{n}\]

展开式各项系数有下列性质
\begin{enumerate}
\item 在二项展开式中,与首末两端“等距离”的两项数相等. 我们已知$\comb{m}{n}=\comb{n-m}{n}$,所以:
\[\comb{0}{n}=\comb{n}{n},\; \comb{1}{n}=\comb{n-1}{n},\; \comb{2}{n}=\comb{n-2}{n},\; \ldots,\; \comb{k}{n}=\comb{n-k}{n},\ldots\]
\item 如果二项式的幂指数是偶数,中间一项的系数大,如果二项式的幂指数是奇数,中间两项系数相同并且大.
\end{enumerate}

由于展开式各项系数顺次是
\[\comb{0}{n}=1,\quad \comb{1}{n}=n,\quad \comb{2}{n}=\frac{n(n-1)}{1\cdot 2},\ldots\]
\[\comb{k}{n}=\frac{n(n-1)(n-2)\cdots(n-k+1)}{1\cdot 2\cdot 3\cdots k},\ldots,\quad \comb{n}{n}=1\]
其中,后一个系数的分子是前一个系数的分子乘以逐次减小1的数,(如$n,\; n-1,\; n-2,\; \ldots$), 分母是乘以逐次增大1的数(如$1,2,3,\ldots$)得到的,因而,各项系数从开始起总是逐渐增大,又因为与首末两端“等距离”的两项系数相等,所以系数增大到某一项时就逐渐减小,而最大项必在中间.

当$n$是偶数时,$n+1$是奇数,展开式共有$n+1$项,所以
展开式有中间一项,即第($\frac{n}{2}+1$)项,并且系数最大。

当$n$是奇数时,$n+1$是偶数,展开式共有$n+1$项,所以有中间两项,即第($\frac{n+1}2$)和第($\frac{n+1}2+1$)两项,系数相同并且最大.

\begin{example}
    求$(x+y)^{10}$和$(x+y)^{11}$的最大系数的项.
\end{example}

\begin{solution}
    $(x+y)^{10}$最大系数项的项数是
$\frac{10}{2}+1=6$,所以
$$\mathrm{T}_{6}=\mathrm{C}_{10}^{5} x^{10-5}y^{5}=252x^{5}y^{5}$$

$(x+y)^{11}$最大系数的项 数是$\frac{11+1}{2}=6$或$6+1=7$, 所以
\[\begin{split}
    \mathrm{T}_{6}&=\mathrm{C}_{11}^{5}x^{11-5}y^{5}=462x^{6}y^{5}\\
    \mathrm{T}_{7}&=\mathrm{C}_{11}^{6}x^{11-6}y^{6}=462x^{5}y^{6}
\end{split}\]
\end{solution}

\begin{example}
    在$(5x+2y)^{29}$展开式中第几项的系数最大,并
求它的系数。
\end{example}

\begin{solution}
    设$T_{k+1}$系数最大,则它必须满足
\[\begin{cases}
\frac{{\rm T}_{k+1}\text{系数}}{{\rm T}_{k}\text{系数}}=\frac{\comb{k}{29}5^{29-k}2^k}{\comb{k-1}{29}5^{29-k+1}2^{k-1}}=\frac{2(29-k+1)}{5k}\ge 1\\
\frac{{\rm T}_{k+1}\text{系数}}{{\rm T}_{k+2}\text{系数}}=\frac{\comb{k}{29}5^{29-k}2^k}{\comb{k+1}{29}5^{29-k-1}2^{k+1}}=\frac{5(k+1)}{2(29-k)}\ge 1\\
\end{cases}\]
即
$\begin{cases}
    7k\le 60\\
    \frac{53}{7}\le k\le 29
\end{cases}$

解得$7\frac{4}{7}\le k\le 8\frac{4}{7}$. 因为$k$为正整数,故$k=8$,即系数最大项系数是
\[{\rm T}_{8+1}=\comb{8}{29}5^{29-8}\cdot 2^8=\comb{8}{29}5^{21}\cdot 2^8\]
\end{solution}

\begin{example}
    求证:$\sum\limits^n_{k=0}\comb{k}{n}=\comb{0}{n}+\comb{1}{n}+\comb{2}{n}+\cdots+\comb{n-1}{n}+\comb{n}{n}=2^n$
\end{example}

\begin{proof}
在二项式公式中取$x=y=1$,有:
\[(1+1)^n=\sum^n_{k=0}\comb{k}{n}1^{n-k}1^k=\sum^n_{k=0}\comb{k}{n}\]
所以$\sum\limits^n_{k=0}\comb{k}{n}=2^n$
\end{proof}

\begin{example}
证明在$(x+y)^n$展开式中,奇数项系数的和等于
偶数项系数的和.
\end{example}

\begin{proof}
    在二项式公式中取$x=1$, $y=-1$,有
    \[(1-1)^n=\sum^n_{k=0}(-1)^k\comb{k}{n}1^{n-k}1^k=\sum^n_{k=0}(-1)^k\comb{k}{n}\]
即
\[0=\left(\comb{0}{n}+\comb{2}{n}+\cdots\right)-\left(\comb{1}{n}+\comb{3}{n}+\cdots\right)\]
所以
\[\comb{0}{n}+\comb{2}{n}+\cdots=\comb{1}{n}+\comb{3}{n}+\cdots\]
\end{proof}

\begin{example}
求证:$\sum\limits^n_{k=1}2k\comb{k}{n}=n\cdot 2^n$
\end{example}

\begin{proof}
因为
\[\begin{split}
    k\comb{k}{n}&=k\cdot \frac{n(n-1)\cdots (n-k+1)}{k!}\\
    &=n\cdot \frac{(n-1)(n-2)\cdots (n-k+1)}{(k-1)!}=n\comb{k-1}{n-1}
\end{split}\]
所以
\[\sum^n_{k=1}2k\comb{k}{n}=2\sum^n_{k=1}k\comb{k}{n}=2n\sum^{n-1}_{k=0}\comb{k}{n-1}=2n\cdot 2^{n-1}=n\cdot 2^n\]
\end{proof}


\begin{example}
    计算$(1.009)^{20}$和$(0.991)^{5}$的近似值(精确到0.001)
\end{example}

\begin{solution}
\[\begin{split}
    (1.009)^{20}&=(1+0.009)^{20}\\
&=1+20\x0.009+190\x(0.009)^2+1140\x(0.009)^3+\cdots\\
&=1+0.18+0.01539+0.00083106+\cdots
\end{split}\]
可以看出,要使所求的近似值的误差不超过0.001,需取展开式中前三项的和,就是
\[\begin{split}
(1..009)^{20}&\approx 1+0.18+0.01539\approx 1.195\\
(0.991)^5&=(1-0.009)^5=1-5\x0.009+10\x(0.009)^2+\cdots    
\end{split}\]
根据精确度要求,第三项以后各项都可以删去,所以,
\[(0.991)^5\approx 1-0.045=0.955\]
\end{solution}

\begin{ex}
\begin{enumerate}
    \item 求二项展开式里系数最大的项,(系数只需用组合数符号表示,不必算出):
\begin{multicols}{2}
\begin{enumerate}[(1)]
    \item $(a+b)^{10}$
    \item $(1+x)^{17}$
\end{enumerate}
\end{multicols}
    \item  应用二项式定理展开$(x^2+2x+2)^5$
    \item  $(x+y)^{2n}$的展开式里如果第5项的系数与第13项的系数相等,求展开式里系数最大的项(系数用组合符号表示,不必算出).
    \item  求$(1-x)^{13}$的奇次项系数的和.
    \item 证明:$\comb{0}{n}+\comb{2}{n}+\comb{4}{n}+\cdots+\comb{n}{n}=2^{n-1}$($n$为偶数)
    \item 计算$(3.002)^0$和$(3.998)^5$误差不超过0.001的近似值.
\end{enumerate}
\end{ex}

\section*{习题2.3}
\begin{enumerate}
    \item 用二项式定理展开。
\begin{multicols}{2}
\begin{enumerate}[(1)]
    \item $\left( \frac {\sqrt {x}}2- \frac 2{\sqrt {x}}\right) ^{7}$
    \item $\left( \frac {x^{2}}2+ \frac {3}{x}\right) ^{8}$
\end{enumerate}    
\end{multicols}
\item 化简:
\begin{enumerate}[(1)]
    \item $(1+\sqrt{x})^{5}+(1-\sqrt{x})^{5}$
    \item $\left( 2x^{\tfrac 12}+ 3x^{- \tfrac 12}\right) ^{4}- \left( 2x^{\tfrac 12}- 3x^{\tfrac 12}\right) ^{4}$
\end{enumerate}
\item \begin{enumerate}[(1)]
    \item 求$(1-2x)^{15}$的展开式中前四项;
    \item 求$(2a^{3}-2b^{2})^{10}$的展开式中第11项;
    \item 求$\left(\frac{\sqrt{x}}{3}+\frac{3}{\sqrt{x}}\right)^{12}$的展开式的中间一项;
\item 求$(x\sqrt{y}-y\sqrt{x})^{15}$的展开式的中间两项.
\end{enumerate}
\item 求下列各式的展开式中指定各项的系数。
\begin{enumerate}[(1)]
    \item $\left( 1- \frac 1{2x}\right)^{10}$含$\frac1{x^5}$的项;
    \item $\left(2x^{3}-\frac{1}{2x^{2}}\right)^{10}$的常数项.
\end{enumerate}
 
\item \begin{enumerate}[(1)]
    \item $\left({\sqrt {x}} + \frac 1{\sqrt [3]{x^{2}}}\right) ^{n}$的展开式里第5项系数与
第3项系数的比是7:2, 求展开式里含有$x$的项.
\item 已知$(1+x)^n$的展开式里第四项与第八项的系数相等,求这两项的系数.
\end{enumerate}

\item \begin{enumerate}[(1)]
    \item 求$\left(\frac{x}{y}-\frac{y}{x}\right)^{13}$展开式里系数最大的项.
    \item 写出$\left(x^2-\frac{1}{\sqrt{x}}\right)^{10}$的展开式里系数绝对值最大的项.
    \item 求$(1+x+x^2)(1-x)^5$的展开式里的中间项.
\end{enumerate}

\item 用二项式定理证明:
\begin{enumerate}[(1)]
    \item $(n+1)^{n-1}$能被$n^2$整除;
    \item $99^{10}-1$能被1000整除.
\end{enumerate}

\item 证明:
\begin{enumerate}[(1)]
    \item $\left(x-\frac{1}{x}\right)^{2n}$的展开式中常数项是$(-2)^n\frac{1\cdot 3\cdot 5\cdots (2n-1)}{n!}$
    \item $(1+x)^{2n}$的展开式的中间一项是$\frac{1\cdot 3\cdot 5\cdots (2n-1)}{n!}(2x)^n$
\end{enumerate}

\item 求证:
\begin{enumerate}[(1)]
    \item $\sum\limits^n_{k=0}\comb{n}{n+k}=\comb{n+1}{n+m+1}$
    \item $2^n-\comb{1}{n}2^{n-1}+\comb{2}{n}2^{n-2}+\cdots+(-1)^{n-1}\comb{n-1}{n}\cdot 2+(-1)^n=1$
    \item $\sum\limits^n_{k=0}(2k+1)\comb{k}{n}=(n+1)2^n$
    \item $1+2\comb{1}{n}+\comb{2}{n}+2\comb{3}{n}+\comb{4}{n}+2\comb{5}{n}+\cdots +2\comb{n-1}{n}+\comb{n}{n}=3\cdot 2^{n-1}$(其中$n$是偶数)
\end{enumerate}
\item 求$(1+x_1)(1+x_2)(1+x_3)\cdots(1+x_n)$的展开式里各项系数之和.
\end{enumerate}


\section*{本章内容要点}
一、本章主要内容是排列与组合,二项式定理;重点是不许重复的排列与组合,二项式定理及其性质.

二、两个基本原理——加法原理与乘法原理,是推导排列与组合数公式的基础,也是直接解决某些计数问题的方法. 应用中,要注意区别两个原理:

加法原理与分类有关,即如果完成一件事有$n$类相互独立的办法,那么求完成这件事情的方法种数,就要用加法原理;

乘法原理与分步有关,即如果完成一件事需要分$n$步完成,缺一不可,而每一步又各有若干方法,那么,求完成这件事的方法种数,就要用乘法原理。

三、排列与组合的区别是看研究的问题是否与元素选取的顺序有关,与顺序有关就属于排列;与顺序无关就属于组合.

排列又分为不许重复选取的全排列与选排列、可重复的排列、环形排列;

组合又分为不许重复的组合与可重复的组合。

四、排列与组合的计数公式有:
\begin{enumerate}
    \item 从$n$个不同元素中,任取$m$个没有重复的元素:

排列数公式
\begin{itemize}
    \item 选排列:$\perm{m}{n}=n(n-1)\cdots(n-m+1)=\frac{n!}{(n-m)!}\quad (m<n)$
    \item 全排列:$\perm{n}{n}=n!$
\end{itemize}

组合数公式:$\comb{m}{n}=\frac{\perm{m}{n}}{\perm{m}{m}}\quad (m\le n)$

组合数的性质:$\comb{m}{n}=\comb{n-m}{n},\qquad \comb{m}{n+1}=\comb{m}{n}+\comb{m-1}{n}\quad (m\le n)$

环形排列数公式:$\comb{m}{n}\cdot \perm{m-1}{m-1}=(m-1)! \comb{m}{n}\quad (m\le n)$

\item 可重复的排列数与组合数公式:
\begin{itemize}
    \item 重复排列数$n^m$
    \item 重复组合数${\rm H}^m_n=\comb{m}{n+m-1}$
\end{itemize}
\end{enumerate}

五、二项式定理指出了二项代数利的幂的展开公式。
\[(a+ b) ^{n}=\mathrm{C}_{n}^{0}a^{n}+{\rm C}_n^{1}a^{n- 1}b+ \cdots + \mathrm{C}_{n}^{r}a^{n- r}b^{r}+ \cdots + \mathrm{C} _{n}^{n}b^{n}\]
其中,展开式的通项(第$r+1$项)为:
$$\mathrm{T}_{r+1}=\mathrm{C}_{n}^{r}a^{n-r}b^{r}$$

二项式的展开式中,系数、幂指数、项数等都有很好的
性质,其中主要性质有:
\begin{enumerate}
    \item 
展开式中,与首末两端“等距”的两项系数相等,
二项式$n$次幂的展开式共有$n+1$项.
\item 
当 幂 指 数 $n$为 偶 数 时 , 中 间 一 项 的 系 数${\rm C}_n^{\tfrac{n}{2}}$最大(第$\frac n2+1$项); 当$n$为奇数时,中间有两项系数最大,即第$\frac{n+1}2$项的系数$C_{n}^{\tfrac{n-1}2}$与第$\frac{n+1}2+1$项的系数${\rm C}_{n}^{\tfrac{n+1}2}$相等且最大.
\end{enumerate}

\section*{复习题二}
\begin{enumerate}
    \item 解不等式:
\begin{multicols}{2}
\begin{enumerate}[(1)]
    \item $2<\frac{\perm{m+1}{m+1}}{\perm{m-1}{m-1}}\le 42$
    \item $x\perm{3}{8}<\perm{3}{x}$
\end{enumerate}
\end{multicols}

\item 求证:
\begin{enumerate}[(1)]
    \item $1!+2\cdot 2!+3\cdot 3!+\cdots+n\cdot n!=(n+1)!-1$
    
    (提示:$n\cdot n!=(n+1)!-n!$)

\item $\comb{m}{n-1}+\comb{m}{n-2}+\comb{m}{n-3}+\cdots+\comb{m}{m+1}+\comb{m}{m}=\comb{m+1}{n}$
\end{enumerate}

\item \begin{enumerate}[(1)]
    \item 已知$\frac{1}{\comb{m}{5}}-\frac{1}{\comb{m}{6}}=\frac{1}{10\cdot\comb{m}{1}}$,求$\comb{m}{8}$
    \item 已知$\frac{\comb{m-1}{n}}{2}=\frac{\comb{m}{n}}{3}=\frac{\comb{m+1}{n}}{4}$,求$n$与$m$.
\end{enumerate}

\item 用数字0, 1, 2, 3, 4, 5组成没有重复数字的数:
\begin{enumerate}[(1)]
\item 能够组成多少个六位奇数?
\item 能够组成多少个大于201345的自然数? 
\end{enumerate}

\item 8个不同的元素排成一行:
\begin{enumerate}[(1)]
 \item 其中某2个元素要排在一起,有多少种排法?
\item 其中某2个元素不排在一起,有多少种排法?
\item 其中某4个元素要排在一起,另外4个元素也要排在一起,有多少种排法?  
\end{enumerate}

\item 书架上有4本不同的数学书,5本不同的物理书,3本不同的化学书,全部竖起排成一排,如果不使同类的书分开,一共有多少种排法?
\item 从1到9这九个数字每次取出5个数字:
\begin{enumerate}[(1)]
\item 可以组成多少个能被25整除的五位数?
\item 可以组成多少个小于98000的五位数?
\item 可以组成多少个五位偶数?
\item 可以组成多少个奇数位是奇数,偶数位是偶数的五位数?   
\end{enumerate}

\item \begin{enumerate}[(1)]
    \item 一个集合由8个不同的元素组成,这个集合中3个元素的子集有几个?
    \item 一个集合由5个不同的元素组成,其中含1个,2个,3个,4个元素的子集共有几个?
\end{enumerate}

\item \begin{enumerate}[(1)]
\item 平面内有$n$条直线,其中没有两条互相平行,也没有三条相交于一点,一共有多少个交点?
\item 空间有$n$个平面,其中没有两个互相平行,也没有三个相交于一直线,一共有多少条交线?
\item 平面有两组平行线,一组$m$条,另一组$n$条,这两组平行线相交,可以构成多少个平行四边形?
\item 空间有三组平行平面,第一组有$m$个,第二组有$n$个,第三组有$\ell$个,不同组的平面都相交,可以构成多少个平行六面体?
\end{enumerate}

\item 一个班有48名学生,分成六排,每排六人
\begin{enumerate}[(1)]
\item 问:有多少种不同坐法?
\item 若有2人必须坐在前排,3人必须在最后一排,有多少种不同的坐法?
\end{enumerate}

\item 由0至9这10个数字组成的没有重复数字的九位数字有多少个?其中能被3整除的有多少个?
\item 有$n+1$件不同的奖品,全部赠给$n$个先进学生代表.如果每人至少得到一件,有多少种不同的赠送方法?
\item \begin{enumerate}[(1)]
    \item 求$\left(9x-\frac{1}{\sqrt[3]{x}}\right)^{10}$展开式的常数项;
    \item 求$(1+x+x^2)(1-x)^{10}$展开式中$x^4$的系数;
    \item 求$(1-x)^{11}+x(1-2x)^{10}-x^2(1-3x)^9$展开式中$x^5$的系数;
    \item 求$(1+2x-3x^2)^6$展开式中$x^5$的系数;
    \item 求$(x^{-1}-1+x)^5$展开式里不含$x$的项.
\end{enumerate}
\item \begin{enumerate}[(1)]
    \item 用二项式定理证明$55^{55}+9$能被8整除;
    \item 用二项式定理求$89^{10}$除以88的余数.
\end{enumerate}

\item 证明:$(1+x)^{2n}$展开式中$x^n$的系数等于$(1+x)^{2n-1}$展开式中$x^n$的系数的2倍.
\item 已知$\left(\sqrt{x}+\frac{1}{\sqrt[3]{x}}\right)$展开式的系数之和比$(a+b)^{2n}$展开式的系数之和小240,求:
\begin{enumerate}[(1)]
    \item $\left(\sqrt{x}+\frac{1}{\sqrt[3]{x}}\right)$展开式的第三项:
    \item $(a+b)^{2n}$展开式的中间项.
\end{enumerate}

\item 已知$(1+x)^n$展开式里连续三项系数的比是3:8:14,求展开式里系数最大的项.
\item 已知$\left(\sqrt[3]{x^{-2}}+x^2\right)^{2n}$的展开式的系数之和比$\left(\sqrt[3]{x^{-2}}+x^2\right)^{n}$的展开式的系数之和大992,求$\left(\sqrt[3]{x^{-2}}-x^2\right)^{n}$的展开式里含$x^{\tfrac{4}{3}}$项的系数.

\item 证明:
\begin{enumerate}[(1)]
    \item $\sum\limits^n_{k=0}\left(\comb{k}{n}\right)^2=\frac{(2n)!}{n!\cdot n!}$
    \item $\comb{0}{n}\comb{k}{m}+\comb{1}{n}\comb{k-1}{m}+\cdots+\comb{k}{n}\comb{0}{m}=\comb{k}{n+m}$
    \item $\left(\comb{0}{n}-\comb{2}{n}+\comb{3}{n}+\cdots\right)^2+ \left(\comb{1}{n}-\comb{3}{n}+\comb{5}{n}+\cdots\right)^2=\comb{0}{n}+\comb{1}{n}+\comb{2}{n}+\cdots+\comb{n}{n}$
\end{enumerate}

\item 在$(a-3b)^{35}$的展开式中第几项的系数的绝对值最大?
\end{enumerate}


\chapter{概率论初步}

\section{随机事件与概率}

\subsection{随机事件及其概率}
在一定的条件下,可能发生也可能不发生的事件称为随机事件.

\begin{example}
    掷一枚硬币,“正面向上”这一事件可能发生,也可能不发生,因此,它是一随机事件。
\end{example}

\begin{example}
    对准靶发射一枪,“中靶”这一事件可能发生,也可能不发生,因此,它是一随机事件.
\end{example}

\begin{example}
    将1,2,3这三个数字随便排列,就得一个三位数,“这个三位数大于300”可能发生,也可能不发生,因此,它是一随机事件.
\end{example}


在现实世界中,有许多事件在一定的条件下,可能发生也可能不发生,这些都是随机事件. 通常,我们用大写的英文字母$A,B,C,\ldots$表示随机事件,概率就是反映随机事件出现的可能性大小的一个量;在同一条件下,有些事件发生的可能性比另一些更大些,它们相应的概率也就比另一些的大一些,所以随机事件的概率是客观存在的,问题是如何求出随机事件相应的概率. 为了讨论方便起见,习惯上把“必然事件”与“不可能事件”也看作随机事件。“必然事件”就是在给定条件下必然发生的事件,通常用字母$U$表示. “不可能事件”就是在给定条件下必然不发生的事件,通常用字母$V$表示.在例3.1中,“正面向上或反面向上”就是一个必然事件,“正面和反面都向上”就是一个不可能事件;在例3.3中,“三位数小于400”是一个必然事件,“三位数小于110”就是一个不可能事件.

为了讨论反映随机事件的可能性的大小的量——概率。我们进一步考察“投掷一枚硬币”、“从50只三级管中任抽一只检查”这样一类随机现象。例如,投掷一枚硬币,共有“出现正面”和“出现反面”两种试验结果,通常总认为硬币是“均匀的”,所以出现这两种结果的可能性是相同的。类似地,从50只三级管中任抽一只,共有50种不同的结果,而且只要检查员事先不带主观偏见,抽到任何一只的可能性都相同. 今后,我们称每一试验结果为基本事件,在具体问题中,弄清楚所有基本事件是十分重要的事,首先要注意到不同的基本事件是不能同时发生的;其次要注意到事件总是由若干个基本事件组成. 某一事件所包含的基本事件发生了,就说该事件发生了.

从这两个例子可以看到,这一类随机现象有两个共同的特征:
\begin{enumerate}[(1)]
    \item 有限性:上面每次试验只可能有有限种不同的结果,即有限个基本事件;
    \item 等可能性:由于其自然对称性,出现每一基本事件的可能性是相同的.
\end{enumerate}

这是一类最简单但却是常见的随机现象,我们称它为古典概率模型(简称古典概型),如何来描述这类事件的可能性的大小呢?例如我们问:“50只三级管中有2只次品,从
中任抽一只,抽得次品的可能性有多大?”人们常常用“次品率”来描述这种可能性,这时,次品数(2)和全产品数(50)之比为$2/50=4\%$.我们就说这批三级管的次品是4\%.对于上面的分母和分子,我们也可以这样来看:50是抽一只检查(看作一次试验)时所有可能的基本事件数,2是“抽一只检查得次品”这一事件所包含的基本事件数.

因此对于古典概型,如果试验的基本事件总数为$n$,随机事件$A$所包含的基本事件数为$m$,我们就用数$m/n$来描述每次试验中事件$A$出现的可能性的大小,称它为事件$A$的概率,记作$\Pr(A)$. 即我们定义
\[\Pr(A)=\frac{m}{n}\]

于是对“投掷一枚硬币出现正面”这一随机事件中,由于试验的基本事件总数为2,而“出现正面”只包含其中一个基本事件,所以它的概率等于1/2.对于这样定义的概率显然有

\begin{blk}
    {性质1} \[\Pr(U)=1,\qquad \Pr(V)=0\]
\end{blk}

因为必然事件$U$包含了所有基本事件,所以$\Pr(U)=\frac{n}{n}=1$,而不可能事件$V$不含任何基本事件,所以$\Pr(V)=\frac{0}{n}=0$.

\begin{blk}
 { 性质2}\[0\le \Pr(A)\le 1\]  
\end{blk}

因为事件$A$所含基本事件数$m$满足不等式$0\le m\le n$,所以$0\le \Pr(A)\le 1$.

要算出基本事件总数及A所包含的基本事件数,“排列”、“组合”常常是十分有用的工具.

\begin{example}
    $1,2,3,5,6$六个数字中任取两数,试计算它们都是偶数的概率.
\end{example}

\begin{solution}
    六个数字中任取两个,共有$\comb{2}{6}=15$种不同的取法,这就是全体基本事件,即$n=15$. 而事件$A$:“任取两数都是偶数”则要从$2,4,6$三个偶数中取两个,也就是说事件$A$包含$\comb{2}{3}=3$个基本事件,因此它的概率
\[Pr(A)=\frac{\comb{2}{3}}{\comb{2}{6}}=\frac{3}{15}=\frac{1}{5}\]
\end{solution}


\begin{example}
    在一批零件中有$n$个一等品,$m$个三等品,逐个进行检查. 若已查明前$k\;(k<n)$个均为一等品,求第$k+1$次检查时仍得一等品的概率.
\end{example}

\begin{solution}
    由于已查明前$k$个均为一等品,所求第$k+1$次检查基本事件总数为$n+m-k$,“得一等品”这一事件包含的基本事件数为$n-k$. 因此所求的概率
    \[P=\frac{n-k}{n+m-k}\]
\end{solution}


\begin{example}
    一部四卷文集,按任志次序放到书架上,问各卷自左至右或自右至左的卷号恰为1, 2, 3, 4的顺序的概率是多少?
\end{example}

\begin{solution}
    一部四卷文集,在书架上可有$\perm{4}{4}=4!=24$种不同的排列方法,其中自左至右或自右至左卷号恰为1, 2, 3, 4的顺序的共有2种.因此所求的概率
    \[P=\frac{2}{24}=\frac{1}{12}\]
\end{solution}

\begin{example}
    某停车场有12个位置列成一行.求有8个位置停了车而空着的四个位置连在一起的概率.
\end{example}

\begin{solution}
12个位置中占去8个,共有$\comb{8}{12}$种方法,把四个接连的空位看作一个位置,发生四个接连的位置空着的情形相当于这个假想的位置插入8个停车位置中间或者它们的两端,共有9种不同方法,因此所求的概率为
\[P=\frac{9}{\comb{8}{12}}=\frac{1}{55}\]
\end{solution}

\begin{example}
    鱼池中共有鱼$N$条,从中捕得$t$条,加了标志后立即放回池中,经过一段时间后,再从池中抽出$n$条鱼,问其中有$s$条标志鱼的概率是多少?
\end{example}

\begin{solution}
    从$N$条鱼中捕得$n$条,共有$\comb{n}{N}$种可能结果,它就是基本事件总数. 在捕得的$n$条鱼中恰有$s$条标志鱼,它们是从$t$条标志鱼中捕得的,这种捕法共有$\comb{s}{t}$种. 而对于这样的每一种捕法,其余的$n-s$条未加标志的鱼是从$N-t$条鱼中捕得的,共有$\comb{n-s}{N-t}$种捕法. 因此“捕得$n$条鱼,其中$s$条是标志鱼”的捕法共有$\comb{s}{t}\comb{n-s}{N-t}$种,所求概率
\[P_s(N)=\frac{\comb{s}{t}\comb{n-s}{N-t}}{\comb{n}{N}}\]
\end{solution}

\begin{example}
    某小组有成员3人,每人在某星期7天中参加劳动一天,如果劳动日期可随机安排,求3人在不同的3天参加劳动的概率.
\end{example}

\begin{solution}
    对于某一组员(甲)来说,他可以安排在7天中的任何一天参加劳动,即有7种安排方法,而对于甲的任何一种安排另一组员(乙)又可有7种安排法,因此甲乙两人共有$7^2$种不同的安排法。类似地,对于$7^2$种的任何一种安排,第三名组员(丙)又可有7种安排法.因此所有可能的安排法——总的基本事件数等于$7^3$,而3人安排在7天中的不同的3天有${\rm A}^3_7$种不同的方法,所以要求的概率为
\[P=\frac{{\rm A}^3_7}{7^3}=\frac{30}{49}\]
\end{solution}

上述例3.9颇具典型性,$N$个人在$n$天中的任意安排法相当于把$N$个球随机地放入$n$个盒子里去的方法,共有$N^n$种. 这一类问题称为盒子模型.

\begin{ex}
\begin{enumerate}
    \item 指出下列事物是必然事件,不可能事件,还是随机事件.
\begin{enumerate}[(1)]
\item 如果$a$,$b$都是实数,那么$a+b=b+a$;
\item 从一副扑克牌中任意抽取两张,得到“黑桃”;
\item 某电话总机在一分钟内接到20次呼唤.
\end{enumerate}
\item 某班有50位同学而其中女同学占15名.今碰到这个班的同学,正好碰到一个男同学的概率是多少?正好碰到一个女同学的概率是多少?这两个概率之间有什么关系?为什么?
\item 盒中有100个铁钉,其中有90个是合格的,而10个是坏的,从中任意抽取10个,问其中没有一个为坏的概率是多少?
\item 设袋中有8个球,其中5个白球3个红球,从中任意抽取4个,问恰好抽到3个白球的概率是多少?
\item 两袋分别盛着写有0, 1, 2, 3, 4, 5六个数字的六张卡片,从每袋中各取一张,求所得两数之和等于6的概率.现在有人给出下述两种不同解答:

解1:两数之和共有$0,1,2,\ldots,10$十一种不同结果,因此,所求的概率为$1/11$.

解2:从每袋中各任取一张卡片,共有$6^2$种取法,其中和数为6的情形有5种;
\[(1,5);\quad (2,4);\quad (3,3);\quad (4,2);\quad (5,1)\]
因此所求的概率为$5/36$.

试问哪一种解法正确?为什么?
\item 号码锁上有六个拨盘,每个拨盘上有0—9共10个数字,当这6个拨盘上的数字组成某一个6位数时(第一位可以是0),锁才能打开。如果不知道锁的号码,一次就把锁打开的概率是多少?
\item 设有50张考签,分别加以标号$1,2,\ldots,50$,一学生任意抽一张进行考试,求抽到前10号考签的概率?
\item 一批产品共有100件,其中有次品5件,从中任取10件,求所取10件中至多有2件是次品的概率。
\item 袋中装有$n$个白球和$m$个黑球,从中任取$a+b$个,求所取的球恰含$a$个白球和$b$个黑球的概率?
\end{enumerate}
\end{ex}

\subsection{等可能性事件的概率}
在古典概型里,我们曾规定了$\Pr(A)=m/n$,这是在等可能性的大前提下得出的,这种类型在我们计算事件发生的
概率时是常常碰到的,也是较为重要的,为着重视这种类型,我们再深入研究这类问题.

\begin{example}
    一个均匀材料造的正方体玩具,各个面上分别标以数字$1,2,3,4,5,6$,这种玩具叫骰子,
\begin{enumerate}[(1)]
\item 将骰子抛掷一次,朝上的一面出现奇数的概率是多少?
\item 抛掷2次,朝上的一面的数字之和为7的概率是多少?
\end{enumerate}
\end{example}

\begin{solution}
\begin{enumerate}[(1)]
\item 朝上的一面可为$1,2,3,4,5,6$等六种之一,且每种可能性相等,朝上一面出现奇数,可为$1,3,5$等三种之一,且每种可能性相等,设所问的事件概率为$\Pr(A)$,则$\Pr(A)=\frac{3}{6}=\frac{1}{2}$
\item 抛掷二次,每次朝上一面可有六种情况,两次则共有$6\x6=36$种情况,朝上一面的数字之和为7,则有第一次是1,第二次是6,第一次是6,第二次是1等两种情况。
2与5;3与4,又各有两种,总共有6种,故概率$P=\frac{6}{36}=\frac{1}{6}$.
\end{enumerate}
\end{solution}

\begin{example}
    一次掷三枚骰子,得3点,4点,5点,10点的概率各是多少?
\end{example}

\begin{solution}
\begin{enumerate}[(1)]
    \item 得3点的概率,掷三颗骰子共有216种情况,而得3点只有三颗都为1时的一种,故掷得3点的概率为$\frac{1}{216}$;
    \item 掷三颗骰子共有216种情况,而得4点有$2,1,1$; $1,2,1$; $1,1,2$等三种情况,故掷得4点的概率为$\frac{3}{216}=\frac{1}{72}$;
\item 掷三颗骰子共有216种情况,而得5点有$1,1,3$; $1,3,1$; $3,1,1$; $2,2,1$; $2,1,2$; $1,2,2$,即$\comb{1}{3}+\comb{2}{3}=6$种,故掷得5点的概率为$\frac{6}{216}=\frac{1}{36}$;
\item 掷三颗骰子共有216种情况,而得10点有$1,3,6$; $1,4,5$; $2,2,6$; $2,3,5$; $2,4,4$; $3,3,4$等六种
情况.而$1,3,6$;$1,4,5$;$2,3,5$各有$\perm{3}{3}$种排列.$2,2,6$;$2,4,4$;$3,3,4$有$\comb{1}{3}$种组合. 共有$3\perm{3}{3}+3\comb{1}{3}=27$种. 10点的概率为$\frac{27}{216}=\frac{1}{8}$.
\end{enumerate}  

读者试将掷三颗骰子得3点至18点各点数的概率分别求看看各个概率之间有什么关系,各个概率之和又是什么?
\end{solution}

\begin{example}
    袋中装有$a$个白球和$b$个黑球,从中任意地取出$k+1$($k+1\le a+b$)个球,每次取一球,取后不放回,求最后取出的一球是白球的概率?
\end{example}

\begin{solution}
    从$a+b$个球中无放回地接连取出$k+1$个球,一共有
$(a+b)(a+b-1)\cdots(a+b-k-1+1)={\rm A}^{k+1}_{a+b}$种取法。

$A$表“取出的$k+1$个球中最后一球是白球”的事件。最后取出的白球可以是$a$个白球中的任何一个,有$a$种;而其余$k$个可以是余下的$a+b-1$中的任意$k$个,有${\rm P}^k_{a+b-1}$种取法,因此事件$A$共含$a\cdot {\rm P}^k_{a+b-1}$个不同的基本事件,故所求概率为
\[\Pr(A)=\frac{a{\rm P}^{k}_{a+b-1}}{{\rm P}^{k+1}_{a+b}}=\frac{a}{a+b}\]

从上式可以看出,$\Pr(A)$的值与$k$无关,它说明了无论在哪个序号上,取出白球的概率总是相同的。
\end{solution}

\begin{example}
    有$n$个可辨的球,随机地放入$N\; (N\ge n)$个盒中,每一个盒可以放任意多个球,试求:
\begin{enumerate}
\item 指定的$n$个盒中各有一球的概率?
\item 某$n$个盒中各恰有一球的概率?
\item 某指定的一个盒中恰有$m$($m\le n$)个球的概率?
\end{enumerate}
\end{example}

\begin{solution}
    设球以同样的可能性落入$N$个盒中的每一个,$N$个盒中落一个球一共有$N$种方法,$n$个球可以有$N^n$种方法,这就是总的基本事件数.
\begin{enumerate}
\item 以$A$表“指定的$n$个盒中各有一球”的事件.今固定某$n$个盒,第一个球可以落入这$n$个盒中的任何一个,有$n$种方法,第二个球可落在余下的$n-1$个盒中的任何一个,有$n-1$种方法,……,第$n$个球落在最后的一个盒,只能有一种方法,因此事件$A$共含有$n!$个不同的基本事件,故所求概率为
\[\Pr(A)= \frac{n!}{N^n} \]
\item 以$B$表“某$n$个盒中各恰有一球”的事件.因为$n$个盒可从$N$个盒中任意选取,共有$\comb{n}{N}$种选法. 选出这$n$个盒后再按问题1知事件$B$共含$\comb{n}{N}n!$个不同的基本事件数,故所求概率为
\[\Pr(B)=\frac{\comb{n}{N}n!}{N^n}\]
\item 
以$C$表“某指定的一个盒中恰有$m$个球”的事件,因为$m$个球可从$n$个球中任意选出,共有$\comb{m}{n}$种选法,其余$m$个球可以任意落入其余的$N-1$个盒中,共有$(N-1)^{n-m}$种落法,因此事件$C$共含$\comb{m}{n}(N-1)^{n-m}$个不同的基本事件,故所求概率为
\[\Pr(C)=\frac{\comb{m}{n}(N-1)^{n-m}}{N^n}\]
\end{enumerate}
\end{solution}

\begin{ex}
\begin{enumerate}
\item 有人说,先后抛掷两枚硬币,只有“两枚都是正面”,“两枚都是反面”,“一枚正面,一枚反面”这3种结果,因此,“两枚都出现正面”这一事件的概率是1/3,这种说法错在哪里?
\item 有100张已编号的卡片(从1号到100号),从中任取1张,计算:
\begin{enumerate}[(1)]
    \item 卡片号是偶数的概率;    \item 卡片号是3的倍数的概率.
\end{enumerate}
\item 掷骰子一枚
\begin{enumerate}[(1)]
\item 掷一次,出现奇数的概率是多少?出现偶数的概率是多少?
\item 抛掷2次,得数字之和为8的概率是多少?
\end{enumerate}
\item 在7张数的卡片中,有4张正数卡片和3张负数卡片.从中任取2张作乘法练习,其积为正数的概率是多少?其积为负数的概率是多少?
\item 某种产品90件,其中甲等品40件,乙等品30件,丙等品20件,在运送这些产品的路上损坏了3件,如果每件产品被
损坏的可能性相同,计算这三等产品中恰好各损坏一件的概率.
\item 在80件产品中,有50件一等品,20件二等品,10件三等品.从中任取3件,计算
\begin{enumerate}[(1)]
\item 三件都是一等品的概率;    \item 2件是一等品,1件是二等品的概率;    \item 一等品,二等品三等品各有一件的概率.
\end{enumerate}

\item 六本不同的书,三本封面是红色,三本封面是兰色,将它们任意排列在书架的同一层上,问六本书封面颜色相间的概率是多少?
\item 在一副扑克牌(52张)中,有“黑桃,红心,梅花,方块”这四种花色的牌各13张,从中任取4张,这4张牌的花色相同的概率是多少?这四张牌的花色各不相同的概率是多少?
\item 有五根细木棍,它们的长度分别为1, 3, 5, 7, 9厘米,从中任取三根,它们能搭成一个三角形的概率是多少?
\end{enumerate}
\end{ex}

\subsection{互斥事件与加法定理}
在10个乒乓球中,有7个一等品,2个二等品,1个三等品,我们把从中任取一个,取出一等品叫做事件$A$,取得二等品叫做事件$B$,取出三等品叫做事件$C$. 我们看到,如果取出的乒乓球是一等品,即事件$A$发生,那么事件$B$就不发生. 如果取出的是二等品,那么事件$B$发生,那么事件$A$就不发生,也就是说,事件$A$、$B$不可能同时发生. 这种不可能同时发生的两个事件叫做互斥事件. 同理,事件$B$,$C$是互斥事件,事件$A$,$C$是互斥事件. 换句话说,事件$A$,$B$,$C$中,任何两个都是互斥事件。这时我们说事件$A$,$B$,$C$彼此互斥,一般地,如果事件$A_1,A_2,\ldots,A_n$中任何两个都是互斥事件,那么就说事件$A_1,A_2,\ldots,A_n$彼此互斥.

在上面的问题里,因为是任取一个,共有10种等可能的取法,其中得到一等品,二等品,三等品的取法分别有7种,
2种,1种,因此,$\Pr(A)=\frac{7}{10}$, $\Pr(B)=\frac{2}{10}$, $\Pr(C)=\frac{1}{10}$.

现在问:“任取一个乒乓球,取出一等品或二等品”这一事件的概率是多少?这一事件,我们记作“$A+B$”. 因为不论取出一等品还是二等品,都表示这个事件,发生而得到一等品或二等品的取法共有$7+2$种,所以取出一等品或二等品的概率:
\[\Pr(A+B)=\frac{7+2}{10}\]

由$\frac{7+2}{10}=\frac{7}{10}+\frac{2}{10}$,我们看到
\[\Pr(A+B)=\Pr(A)+\Pr(B)\]
它告诉我们:如果事件$A$,$B$互斥,那么事件“$A+B$”发生(即$A,B$中有一个发生)的概率,等于事件$A$,$B$分别发生的概率的和.

一般地,如果事件$A_1,A_2,\ldots,A_n$彼此互斥,那么事件“$A_1+A_2+\cdots+A_n$”发生(即$A_1,A_2,\ldots,A_n$中有一个事件发生,也叫和事件)的概率,等于这$n$个事件分别发生的概率的和,即
\[\Pr(A_1+A_2+\cdots+A_n)=\Pr(A_1)+\Pr(A_2)+\cdots+\Pr(A_n)\]
这叫做互斥事件的加法定理,也就是说互斥事件的和的概率等于各事件概率之和。

\begin{example}
    某地区的年降水量,在10—150毫米范围内的概率是0.12,在15-200毫米范围内的概率是0.25,在20-250毫米范围内的概率是0.16,在25-300毫米范围内的概率是0.14.计算年降水量在10-200毫米范围内的概率与在15-300毫米范围内的概率.
\end{example}

\begin{solution}
    我们把这个地区的年降水量在10-150毫米,15-200毫米,20-250毫米,25-300毫米范围内分别叫做事件$A$,$B$,$C$,$D$. 很明显,这四个事件彼此互斥,根据公式,年降水量在10-200毫米范围内的概率是
\[\Pr(A+B)=\Pr(A)+\Pr(B)=0.12+0.25=0.37\]
年降水量在15-300毫米范围内的概率是
\[\Pr(B+C+D)=\Pr(B)+\Pr(C)+\Pr(D)=0.25+0.16+0.14=0.55\]
\end{solution}


\begin{example}
    在20件产品中,有15件一级品,5件二级品,从中任取3件,其中至少有1件为二级品的概率是多少?
\end{example}

\begin{solution}
    我们把从20件产品中任取3件,其中恰有1件二级品,叫做事件$A_1$;恰有2件二级品,叫做事件$A_2$;3件全是二级品,叫做事件$A_3$,这样,事件$A_1$,$A_2$,$A_3$的概率分别是
\[\begin{split}
    \Pr(A_1)&=\frac{\comb{1}{5}\cdot \comb{2}{15}}{\comb{3}{20}}=\frac{105}{228}\\
    \Pr(A_2)&=\frac{\comb{2}{5}\cdot \comb{1}{15}}{\comb{3}{20}}=\frac{30}{228}\\
    \Pr(A_3)&=\frac{\comb{3}{5}}{\comb{3}{20}}=\frac{2}{228}\\
\end{split}\]

很明显,事件$A_1$,$A_2$,$A_3$彼此互斥,根据公式,3件产品中至少有1件为二级品的概率是
\[\Pr(A_1+A_2+A_3)=\Pr(A_1)+\Pr(A_2)+\Pr(A_3)=\frac{105}{228}+\frac{30}{228}+\frac{2}{228}=\frac{137}{228}\]

从20件产品中任取3件,或者都是一级品,或者不都是一级品(即其中至少有一件是一级品),这两个互斥事件必有一个发生,这种其中必有一个发生的两个互斥事件,叫做对立事件. 一个事件$A$的对立事件通常记作$\overline{A}$,根据对立事件的意义,$A+\overline{A}$是一个必然事件,它的概率等于1.又由于$A$与$\overline{A}$互斥,我们得到
\[\Pr(A)+\Pr(\overline{A})=\Pr(A+\overline{A})=1\]
这就是说,两个对立事件的概率的和等于1.

从上面公式还可得到$\Pr(\overline{A})=1-\Pr(A)$. 
运用这个公式计算事件的概率,有时比较简单.如例3.15还可以这样来解:

从20件产品中任取3件,3件全是一级品(记作事件$A$)的概率:
\[\Pr(A)=\frac{\comb{3}{15}}{\comb{3}{20}}=\frac{91}{228}\]
由于“任取3件,至少有一件为二级品”是事件$A$的对立事件$\overline{A}$,因此,
\[\Pr(\overline{A})=1-\Pr(A)=1-\frac{91}{228}=\frac{137}{228}\]

读者可想想,什么时候用事件的对立事件的概率来计算就比较简单.
\end{solution}

\begin{example}
一个匣子中有红球5个,白球2个,从匣中随便取出两个球,而球的颜色恰好一样的概率是多少?
\end{example}

\begin{solution}
\textbf{解1:} 令$A=$两球颜色一样,$B_1=$两球颜色均为红色,$B_2=$两球颜色均为白色,可见$B_1$和$B_2$互斥,并且$A=B_1+B_2$,因此
\[\Pr(A)=\Pr(B_1)+\Pr(B_2)\]

今从7个球中随便取出两个球,共有$\comb{2}{7}=\frac{7\x 6}{2}=21$种不同的取法,即简单事件总数为21.$B_1$为两球都为红色,即此两球由红球中取得,所以共有$\comb{2}{5}=\frac{5\x4}{2}=10$个简单事件,
因此$\Pr(B_1)=\frac{\comb{2}{5}}{\comb{2}{7}}=\frac{10}{21}$

$B_2$为两球均为白色,即此两球由白球中取得,所以共有
$\comb{2}{2}=1$个简单事件,于是$\Pr(B_2)=\frac{\comb{2}{2}}{\comb{2}{7}}=\frac{1}{21}$,
因此
\[\Pr(A)=\Pr(B_1)+\Pr(B_2)=\frac{10}{21}+\frac{1}{21}=\frac{11}{21}\]

\textbf{解2:}令$A=$“两球颜色一样”,则$\overline{A}=$“两球颜色不一样”,所以
\[\Pr(\overline{A})=\frac{\comb{1}{5}\cdot \comb{1}{2}}{\comb{2}{7}}=\frac{5\x 2}{21}=\frac{10}{21}\]
因此
\[\Pr(A)=1-\Pr(\overline{A})=1-\frac{10}{21}=\frac{11}{21}\]
\end{solution}

一般地说,如果匣中有$m$个红球,$n$个白球,如果从匣中随便抽取两个球,两球颜色相同的概率$P$可以用完全一样的方法,求得其值为
\[P=\frac{\comb{2}{m}}{\comb{2}{m+n}}+\frac{\comb{2}{n}}{\comb{2}{m+n}}=\frac{m(m-1)+n(n-1)}{(m+n)(m+n-1)}=1-\frac{\comb{1}{m}\cdot \comb{1}{n}}{\comb{2}{m+n}}\]

更一般地,如果匣中有$k$种球,第$i$种球共有$n_i$个,从匣中随便取出两个球,这两个球的颜色是一样的概率是多少呢?这也可用类似的方法求出,读者自己去算一下.

\begin{example}
    从52张扑克牌中随便抽取三张.问:
\begin{itemize}
\item $A=$三张花色相同,
\item $B=$三张是“顺子”(即点数相连),
\item $C=$三张是同花顺子(即花色相同,且点数相连)
\end{itemize}
各自出现的概率是多少?
\end{example}

\begin{solution}
从52张牌中随机抽取三张牌,全部基本事件的总个数是
\[\comb{3}{52}=\frac{52\x 51\x 50}{1\x 2\x 3}=52\x 17\x 25\]
注意到花色有四种,令
\[\begin{split}
    A_1=\text{三张都是梅花},&\qquad 
A_2=\text{三张都是方块}\\
A_3=\text{三张都是红心},&\qquad 
A_4=\text{三张都是黑桃}
\end{split}\]
则$A_1,A_2,A_3,A_4$是互斥事件,且$A=A_1+A_2+A_3+A_4$. 又
\[\begin{split}
    \Pr(A_1)&=\Pr(A_2)=\Pr(A_3)=\Pr(A_4)\\
    &=\frac{\comb{3}{13}}{\comb{3}{52}}=\frac{13\x 12\x 11}{52\x 51\x 50}\\
    &=\frac{11}{17\x 50}=\frac{11}{850}
\end{split} \]
因此,$\Pr(A)=4\Pr(A_1)=\frac{22}{425}$.

三张是顺子,从顺子的点数相连这一要求来看,不计花色,共有“$2,3,4$”,“$3,4,5$”,……“$J,Q,K$”,“$Q,K,A$”这十一种不同的情况,依次用$B_1,B_2,\ldots,B_{11}$来表示它们,于是$B_1,B_2,\ldots,B_{11}$互斥,而且是等概率的,于是
\[\Pr(B)=\Pr(B_1)+\cdots +\Pr(B_{11})=11\Pr(B_1)\]

现在来求$\Pr(B_1)$的值.出现“$2,3,4$”这样的顺子,共有$(\comb{1}{4})^3$这么多种,这是因为这时不考虑花色,同点数的牌每样又有四张,所以出现“$2,3,4$”必须在“2”的四张中取出一张,“3”的四张中取出一张,“4”的四张中取一张,因此,共有$(\comb{1}{4})^3=4^3$这么多种,于是
\[\begin{split}
    \Pr(B_1)&=\frac{4^3}{\comb{3}{52}}=\frac{4^3}{52\x 17\x 25}=\frac{16}{13\x 17\x 25}\\
    \Pr(B)&=\frac{11\x 16}{13\x 17\x 25}=\frac{176}{5525}=\frac{8}{23}\Pr(A)
\end{split}\]
显然$\Pr(B)<\Pr(A)$.

再来计算事件$C$的概率,很明显,从上面的讨论中可以看出来,“$2,3,4$”同花的只可能有四种不同的情形.因此
\[\Pr(C)=\frac{11\x 4}{\comb{3}{52}}=\frac{11}{13\x17\x25} =\frac{1}{6}\Pr(B)\]
显然$\Pr(C)<\Pr(B)$

\end{solution}

\begin{example}
   掷四枚硬币,用“1”表示正面向上,用“0”表示反面向上,于是记$x_i$为第$i$枚硬币的结果,每掷一次,$(x_1,x_2,x_3,x_4)$就表示各枚硬币正、反面的出现情况,例如,
\begin{center}
   (1, 1, 0, 0)就是(正,正,反,反)\\
   (1, 0, 0, 1)就是(正,反,反,正)    
\end{center} 

   求下列各事件的概率:
\[\begin{split}
   A&=\text{恰有两个正面向上}\\
   B&=\text{至少有两个正面向上}\\
   C&=\text{第一枚、第二枚硬币的结果正好相反,第三枚、第四枚硬币的结果正好相反}    
\end{split}\]
\end{example}

\begin{solution}
用$x_1,x_2,x_3,x_4$来表示,就有
\[\begin{split}
   A&=\{x_1+x_2+x_3+x_4=2\}\\
   B&=\{2\le x_1+x_2+x_3+x_4\le 4\}\\
   C&=\{x_1=1-x_2,\; x_3=1-x_4\}    
\end{split}\]

   全部简单事件的个数是一个有重复的排列数,它是$2^4=16$.

\begin{enumerate}[(1)]
    \item   $x_1+x_2+x_3+x_4=2$表示四个数中只有两个是“1",其余两个一定是“0”.而且只要选定了“1”的位置,则“0”的位置也就随之而定了,从四个位置中任选两个,则共有$\comb{2}{4}$种,即6种,因此
   \[\Pr(A)=\frac{6}{16}=\frac{3}{8}\]
   \item $x_1+x_2+x_3+x_4\ge 2$,这一情形可以分成下面三种等式的情形:
\[\begin{split}
    x_1+x_2+x_3+x_4=2&\qquad (\text{有$\comb{2}{4}=6$种情形})\\
    x_1+x_2+x_3+x_4=3&\qquad (\text{有$\comb{3}{4}=4$种情形})\\
    x_1+x_2+x_3+x_4=4&\qquad (\text{有$\comb{4}{4}=1$种情形})\\
\end{split}\]
因此,
\[\Pr(B)=\frac{6+4+1}{16}=\frac{11}{16}\]
\item 此时$x_1=1-x_2$, $x_3=1-x_4$,于是只要看$x_1,x_3$联合起来有多少种不同的情形. 从重复排列的计算公式可以知道$x_1$,$x_3$联合起来共有$2^2=4$种不同情况. 因此,可得
\[\Pr(C)=\frac{4}{16}=\frac{1}{4}\]

\end{enumerate}
   
\end{solution}

\begin{ex}
\begin{enumerate}
    \item 判别下列每对事件是不是互斥事件,如果是,再判别它们是不是对立事件.
从一堆产品中任取2件,其中:
\begin{enumerate}[(1)]
\item 恰有1件次品和恰有2件次品;
\item 有次品和全是次品;
\item 有正品和有次品;
\item 有次品和全是正品.
\end{enumerate}

\item 从一批乒乓球产品中任取一个,如果其重量小于2.45克的概率是0.22,重量不小于2.50克的概率是0.20,那么重量在2.45—2.50克范围内概率是多少?
\item 某射手在一次射击中射中10环,9环,8环的概率分别为0.24, 0.28, 0.19.计算这个射手在一次射击中:
\begin{enumerate}[(1)]
    \item 射中10环或9环的概率;
    \item 不够8环的概率.
\end{enumerate}

\item 一个匣子内有9张票,其号数分别为$1,2,\ldots,9$.从中任取2张,其号数至少有1个为奇数的概率是多少?
\item 在50件产品中有45件合格品,5件次品,从中任取3件,计算其中有次品的概率.
\item 从52张中随便抽取五张.问
\begin{itemize}
\item $A=$五张花色相同;
\item $B=$五张是“顺子”;
\item $C=$五张有花顺子;
\item $D=$五张中有四张点数相同(J作11点,Q作12点,K作13点).
\end{itemize}
$A$,$B$,$C$,$D$各自出现的概率是多少?
\item 掷$n$枚硬币,($n$为偶数)
\begin{itemize}
\item $A=$恰有两个正面向上,
\item $B=$至少有两个正面向上,
\item $C=$第一枚、第二枚硬币的结果正好相反,第三枚、第
四枚的结果正好相反,……第$n-1$枚与第$n$枚的结果正
好相反.
\end{itemize}
求$A$,$B$,$C$各事件的概率.
\item 同时掷三颗骰子,得3点至10点的概率是多少?得11点至18点的概率是多少?
\item 掷三颗骰子,得三颗的点数相同的概率是多少?
\item 从52张扑克中任取五张,问得到三张点数相同,其它两张点数也相同的概率.
\end{enumerate}
\end{ex}

\subsection{事件与集合的对应}
我们现在讨论的一些问题,都离不开基本事件. 所以我们可以把基本事件看作衡量随机事件的“最小单位”,它好象是几何中的“点”,集合中的“元素”.而一切随机事件都是由基本事件组成的,这样,事件之间的各种运算,就和集合之间的各种运算完全对应起来,这样既使我们掌握了事件的运算,还使我们进一步理解了集合运算的意义.

在讨论集合运算时,集合中的一个元素称为一个点,用$a$来表示,则$a$和$\{a\}$是不一样的. 前者是一个点,后者是由$a$这一个点组成的单点集,它是一个集合.

当我们用$E_1,\ldots,E_n$表示基本事件时,它相当于集合中的点,随机事件$A$总是由某些基本事件合起来组成的.如
果把一个基本事件$E_i$也看成是随机事件,那就是指由这个$E_i$组成的集合$\{E_i\}$,这个随机事件$\{E_i\}$相当于一个点组成的单点集. 这一点一定要加以注意,不要一开始就发生混淆.

下面将事件间的运算和事件间的关系的各种定义和集合的运算,以及集合的关系对照着列成一张表,以更进一步看清它们之间的联系,也是为了可以用集合语言来理解概率论.

\begin{longtable}{p{.45\textwidth}p{.45\textwidth}}
\hline
事件& 集合\\
\hline
1. 基本事件  &1. 元素(点)\\
$E_1,E_2,\ldots, E_n$ &$a_1,a_2,\ldots,a_n$\\
\hline
2. 必然事件 & 2. 全集$A$\\
$U=\{ E_1,E_2,\ldots, E_n\}$ & $A=\{a_1,a_2,\ldots,a_n\}$\\
\hline
3. 不可能事件$V$  &  3. 空集$\emptyset$\\
$V$中不含任何$E_i$& $\emptyset$中不含任何$a_i$\\
\hline
4. 随机事件$F$ &  4. 子集合$S$\\
$F=\{E_{i_1},E_{i_2},\ldots,E_{i_m}\}$&$S=\{a_{i_1},a_{i_2},\ldots,a_{i_m}\}$\\
$1\le i_1< i_2<\cdots <{i_m}\le n$&$1\le i_1<i_2<\cdots<i_m\le n$\\
\hline
5. 事件$F_1$包含$F_2$写成$F_1\supset F_2$或$F_2\subset F_1$  &  5. 集合$S_1$包含$S_2$,写成$S_1\supset S_2$或$S_2\subset S_1$\\
即$F_2$中的基本事件均为$F_1$中的基本事件。或$F_2$发生时,$F_1$一定发生。 &即$S_2$中的点均为$S_1$中的点\\
\hline
6. $F_1=\{E_{i1},E_{i2},\ldots,E_{ik}\}$,&   6. $S_1=\{a_{i1},a_{i2},\ldots,a_{ik}\}$,\\
$F_2=\{E_{j1},E_{j2},\ldots,E_{jk}\}$,& $S_2=\{a_{j1},a_{j2},\ldots,a_{jk}\}$,\\
\small $F_1+F_2=\{E_{i1},\ldots,E_{ik}, E_{j1},\ldots,E_{jk}\}$  &$S_1+S_2=\{a_{i1},\ldots,a_{ik},a_{j1},\ldots,a_{jk}\}$\\
则$F_1+F_2$称为$F_1$与$F_2$之和,即$F_1, F_2$中至少有一个事件发生。这个运算为加法运算.
& 这是集合的加法,即把$S_1$和$S_2$内的点合并在一起构成的新集合。\\
\hline
7. 由事件$F_1$和$F_2$中的公共基本事件$F_1$与$F_2$之积,写成$F_1F_2$. 即“$F_1,F_2$这两个事件都发生”这一事件。这个运算称为乘法运算。&7. 集合$S_1$与$S_2$的公共元素组成的集合,称为$S_1$和$S_2$之积,写成$S_1S_2$。即$S_1$与$S_2$的公共部分。这是集合的乘法运算。\\
\hline
8. $F_1F_2=V$,则称$F_1,F_2$是互斥的,即$F_1$和$F_2$中没有公共的基本事件,即$F_1$与$F_2$同时发生这一事件是不可能的。&
8. $S_1S_2=\emptyset$,则称$S_1,S_2$是互不相交的,即$S_1$与$S_2$没有公共的元素
\\
\hline
9. 逆事件(对立事件)&9. 余集(补集)\\
如果$F=\{E_{i1},\ldots,E_{ik}\}$,则不在$F$中的那些基本事件全体组成的随机事件称为$F$的逆事件,用$\overline{F}$表示。$\overline{F}$就是“$F$不发生”这一事件。
& 
如果$S=\{a_{i1},\ldots,a_{ik}\}$,则由不在$S$中的那些点的全体组成的集,称为$S$的余集,用$\overline{S}$表示。也即在余集$A$中,但不在$S$中的那些点组成的集合。\\
\hline
10. 设$F=\{E_{i1},\ldots,E_{ik}\}$,则由加法的定义可知$F=\{E_{i1}\}+\cdots +\{E_{ik}\}$ & 10. 设$S=\{a_{i1},\ldots,a_{ik}\}$,则由加法的定义可知$A=\{a_{i1}\}+\cdots+\{a_{ik}\}$\\
\hline
\end{longtable}

由这张对照表,可以由集合的关系式得到事件之间的关
系式,符号的含义和上表相同
\begin{enumerate}
\item $V\subset F\subset U$ (相当于$\emptyset\subset S\subset A$) 

不可能事件$\subset$随机事件$\subset$必然事件
\item 若$F_1\subset F_2$, $F_2\subset F_3$, 则$F_1\subset F_3$
(相当于$S_1\subset S_2,\; S_2\subset S_3$时,必有$S_1\subset S_3$)
\item $F\overline {F}= V$, $F+ \overline {F}= U$ (相当于$S\overline{S} = \emptyset$, $S+ \overline{S} = A$) 

注意:此即$F$与$\overline F$之和必然发生,$F$与$\overline{F}$ 之积是不可能事
件。$\overline{F}$是$F$的逆事件,$F$是$\overline{F}$的逆事件,即彼此互为逆事件。
\item 当$F_1\subset F_2$时,$F_1$中所含的简单事件均为$F_2$中的简单事件,由古典概率公式得
$\Pr\left(F_{1}\right)\le  \Pr\left(F_{2}\right)$
\item 由于$F\overline{F}= V$, $F+ \overline {F}= U$, $F$与$\overline{F}$互斥,从加法定理
得到
$$1=\Pr(U)=\Pr(F)+\Pr(\overline{F})$$
因此,
$\Pr(\overline{F})=1-\Pr\left(F\right)$
\end{enumerate}

\section*{习题3.1}
\begin{enumerate}
\item 设有$n$个房间,分给$n$个不同的人,每个人都以同等可能进入每一房间,而且每间房里的人数没有限制,试求不出现空房的概率?
\item 将$n$个可辨的球投入$N$个盒中,每个球都以同等可能落入每一个盒内,求指定某盒是空的概率.
\item 一批产品共$n$件,其中有$k$件废品,求任意抽出的$m$件产品中恰有$\ell$件废品的概率?
\item 某人有$n$把钥匙,其中只有一把能开他的门,他逐把地取钥匙试开,这试开的手续可能需要$1,2,\ldots,n$次试验,试证明在第$i$次,第$j$次打开锁的概率是一样.
\item 袋中装有10个白球5个黑球,第一次从袋中取出一球(不放回),发现是白球,求第二次从袋中取得白球的概率?并比较两次取得白球概率的大小. 如果第一次取得黑球,求第二次取得白球的概率?
\item 在52张扑克牌中任意取13张,问其中至少有一张是10点的概率是多少?
\item 一盒中有5个白球,7个黑球.
\begin{enumerate}[(1)]
    \item 从中任取一球,问取得是白球和取得是黑球的概率各为多少?
    \item 从中任取两球,问取得都是白球的概率、取得一白一黑的概率各为多少?
\end{enumerate}

\item 箱子所装球的总重量为$N$公斤,而每个球为$n$公斤,其中有$m$个是黑球. 求从箱子中任取一球恰巧是黑球的概率.
\item 一表面为红色的正方体被分割成1000个同样大小的小正方体.试求从中任取一小正方体其两面涂有红色的概率.
\item 从写有$a$,$b$,$c$,$d$,$e$的五张卡片中任取两张,求这两张卡片的字母恰好是按字母顺序相邻排着的概率.
\item 在相应地写有2, 4, 6, 7, 8, 11, 12和13数字的八张卡片中任意取两张,求由所取得的两个数字构成的分数为可约的概率.
\item 从1, 2, 3, 4, 5诸数中任取三数,排成三位数,问如此所得三位数是偶数的概率是多少?
\item 求任意取一整数$N$的:(1)平方;(2)四次方;(3)乘任意一整数后,其尾数为1的概率。
(提示:只需讨论个位数.)
\item 在书架中任意放着10本书,求某给定的三本书放在一起的概率.
\item 4个女孩,3个男孩排成一排,求男女相间的概率.
\item 一个匣子里有90只好的、10只次的螺丝钉.如果从中任意取用10只,恰都是好的螺丝钉的概率是多少?
\item 在一个坛子中放有$n_1$个白球,$n_2$个黑球,逐一全部取出,问第一个和最后一个球都为白球的概率是多少?

\item 求出下述两事件的概率:
\begin{enumerate}[(1)]
    \item 10个人的生日在10个不同的月份;
\item 6个人的生日恰巧在两个月中。(提示:“恰巧在两个月中”不包括“集中在某一个月”的情形. 可先考虑6个人的生日在某指定的两个月里有几种可能.再考虑生日集中在任意两个月里有几种可能.)
\end{enumerate}

\item 电话号码由五个数字组成.每个数字可以是$0,1,2,\ldots,9$中的任意一个数,求电话号码是由完全不相同的数字组成的概率.
\item 从一付扑克(52张)中任取四张,求四张牌的花色各不相同的概率.
\item 为了减少比赛场次,把20个球队分成两组(每组10队)进行比赛,求最强的两队被分在不同组内的概率.
\item 掷三枚骰子,得5点以上(包括5点)的概率是多少!
\item 某人的口袋装有1枚五分的硬币,2枚贰分的硬币以及3枚壹分的硬币,他从中任取3枚,取出的总钱数不少于伍分的概率是多少?
\item 把10个运动队先均分成两组预赛,求最强两队被分在:
\begin{multicols}{2}
\begin{enumerate}[(1)]
\item 不同组内;    \item 同一组内的概率.
\end{enumerate}
\end{multicols}

\item 一座楼房有四个大门,两人住在此楼内,问恰好住在同一大门内的概率是多少?
\item 设在一批有1000件一等品,50件二等品的产品中进行质量检查,如果在该批产品中任意抽查10件,结果全是一等品,问在未被抽查的产品中再任意地连续取两件,至少
有一件是一等品的概率是多少?
\item 用火车运载两个工厂生产的同类产品,其中甲厂30件,乙厂20件,有消息证实,在路途中有两件产品损坏,求损坏的是不同厂的产品的概率.
\end{enumerate}

\section{条件概率与独立事件}
\subsection{条件概率}
若一袋内有7个白球,5个红球,设事件$A=$“取出一个白球”,事件$B=$“取出两个白球”,显然
\[\Pr(A)=\frac{\comb{1}{7}}{\comb{1}{12}}=\frac{7}{12},\qquad \Pr(B)=\frac{\comb{2}{7}}{\comb{2}{12}}=\frac{7}{22}\]

如果我们取出一个白球后,不再放回,那么这时再取出两个白球(注意,实际上是取出了三个白球,但取法是先一个,再两个地取,而不是一起取出)的概率
\[\Pr(B')=\frac{\comb{2}{6}}{\comb{2}{11}}=\frac{3}{11}\]
同样的,如果我们取出两个白球后,不再放回,那么这时再取出一个白球的概率
\[\Pr(A')=\frac{\comb{1}{5}}{\comb{1}{10}}=\frac{1}{2}\]

这个例子告诉我们,我们所计算的某个事件出现的概率是根据另一事件的出现这一事实来确定的.

我们把“出现事件$B$的条件下,出现事件$A$”的概率称为事件$A$关于事件$B$的条件概率,记作$P(A|B)$. 对于上面例子,则
\[\Pr(A')=\Pr(A|B)=\frac{1}{2},\qquad \Pr(B')=\Pr(B|A)=\frac{3}{11}\]

另一方面,考虑事件$AB=$“取出三个白球”,则
\[\begin{split}
    \Pr(AB)=\frac{\comb{3}{7}}{\comb{3}{12}}=\frac{7}{44}&=\frac{7}{22}\x\frac{1}{2}=\Pr(B)\Pr(A|B)\\
    &=\frac{7}{12}\x\frac{3}{11}=\Pr(A)\Pr(A|B)\\
\end{split} \]

就是说$A$与$B$事件同时出现的概率,可以通过$B$出现的概率$\Pr(B)$与$A$关于$B$的条件概率$\Pr(A|B)$来表示.

一般地,对任两个随机事件$A$与$B$,若$\Pr(B)>0$,则有:
\begin{equation}
  \Pr(AB)=\Pr(A|B)\cdot \Pr(B)  \qquad \Pr(AB)=\Pr(B|A)\cdot \Pr(A)\tag{1}
\end{equation}
即
\begin{equation}
    \Pr(A|B)=\frac{\Pr(AB)}{\Pr(B)}\qquad \Pr(B|A)=\frac{\Pr(AB)}{\Pr(A)}\tag{2}
\end{equation}

\begin{proof}
    设基本事件的总数为$e$,其中事件$A$所包含的基本事件有$m$个,事件$B$所包含的基本事件有$n$个,因为我们并没有假定事件$A$和$B$互斥,因此一般都存在着既属于事件$A$,
也属于事件$B$的基本事件,(即事件$A\cdot B$所包含的基本事件)设这样的事件共有$r$个(如图)于是:
\[\Pr(B)=\frac{n}{e},\qquad \Pr(AB)=\frac{r}{e}\]

\begin{center}
    \begin{tikzpicture}
\draw(0,0) rectangle (5,3.5);
\node at (.5,3.25){$e$};
\draw(2,2)node{$A$} circle(1);
\draw(3.5,2)node{$B$} circle(1);
\draw(1.75,1.75)--(1.5,.6)node[below]{$m$};
\draw(3.75,1.75)--(4,.6)node[below]{$n$};
\draw(2.8,1.75)--(2.7,.6)node[below]{$r$};

    \end{tikzpicture}
\end{center}

再看条件概率$\Pr(A|B)$,因为这是在事件$B$发生的条
件下考虑问题的,这时总的基本事件数就是事件$B$所包含的个数$n$,在$B$发生的条件下$A$再发生所包含的基本事件,就必然是属于事件$AB$的$r$个基本事件.

故$\Pr(A|B)=\frac{r}{n}$

这样就有:
\[\Pr(A|B)\cdot\Pr(B)=\frac{r}{n}\cdot \frac{n}{e}=\frac{r}{e}=\Pr(AB)\]
同样可以证明$\Pr(B|A)\cdot\Pr(A)=\Pr(AB)$.
\end{proof}

\begin{example}
    某人提出一个问题,甲先答,答对的概率是0.4;如果答错,由乙答,答对的概率是0.5,求问题由乙解出的概率.
\end{example}

\begin{solution}
“问题由乙解出”相当于“甲答错”($A$)与“乙答对”($B$)两事件一起发生。(即事件$A$,$B$同时发生).

由题意:“甲答错”的概率:$\Pr(A)=1-0.4=0.6$;而“甲答错的条件下乙答对”的概率$\Pr(B|A)=0.5$,
所以“问题由乙解出”的概率:
\[\Pr(AB)=\Pr(B|A)\cdot \Pr(A)=0.5\x0.6=0.3\]
\end{solution}

\begin{example}
    一只袋中有2只白球和3只红球,从袋中取出一只球,然后在第一只不放回的前提下取出第二只球,那么所取出两只球都是红球的概率是多少?
\end{example}

\begin{solution}
    “取出的两只球都是红球”必定是“第一次取得的是红球”($A$),且“第二次取得的是红球”($B$)两事件一起发生(即事件$AB$发生)

    “第一次取置红球”的概率:$\Pr(A)=\frac{3}{5}$,而“第一次取得红球的条件下,第二次取得红球”的概率$\Pr(B|A)=\frac{2}{4}=\frac{1}{2}$,所以“两个球都是红球”的概率:
\[\Pr(AB)=\Pr(B|A)\cdot \Pr(A)=\frac{1}{2}\cdot \frac{3}{5}=\frac{3}{10}\]
\end{solution}

\begin{example}
    某篮球队在己方半场抢得篮板球的概率为0.75而抢得篮板球且反攻投中得分的概率为0.6,问该队已抢得篮板球在手,问这球反攻投中得分的概率有多大?
\end{example}

\begin{solution}
    设事件$A=$“抢得篮板球”;事件$B=$“反攻投中得分”
按题意:$\Pr(A)=0.75$,“抢得篮板球且反攻投中得分”的概率(即积事件$A\cdot B$的概率):$\Pr(AB)=0.6$

所以“抢得篮板球在手,这球反攻投中得分”的概率。
\[\Pr(B|A)=\frac{\Pr(AB)}{\Pr(A)}=\frac{0.6}{0.75}=0.8\]
\end{solution}

\begin{ex}
\begin{enumerate}
    \item 计算在一分钟的区间内进入某邮局的人数的概率如下表:
\begin{center}
\begin{tabular}{cccccc}
\hline
    1分钟内进入的人数&0人&1人&2人&3人&大于3人\\
\hline
概率&0.05&0.15&0.22&0.22&0.36\\
\hline
\end{tabular}
\end{center}    
考虑下述事件:$A=$至少有一个人到达,$B=$至少有二个人到达,$C=$3个顾客到达.

求$\Pr(C|A)$和$\Pr(C|B)$
\item 一个绿骰子和一个红骰子被转动,令$x$表示在红骰子面上出现点子的数目,$y$表示绿骰子面上出现点子的数目.

令:
\begin{multicols}{2}
    $A=$“$x+y>7$的事件”;
    \\
    $B=$“$x+y<6$的事件”;
    \\
    $C=$“$x+y=7$的事件”;\\$D=$“$y=4$的事件”;\\
$E=$“$y=2$的事件”;\\$F=$“$x$或$y$是5”;\\$G=$“$x$和$y$都是5”;\\$H=$“$x+y=10$”
\end{multicols}
求:
\begin{multicols}{3}
 \begin{enumerate}[(1)]
    \item $\Pr(A|D)$
    \item $\Pr(B|E)$
    \item $\Pr(C|D)$
    \item $\Pr(G|F)$
    \item $\Pr(G|H)$
    \item $\Pr(F|H)$
\end{enumerate}   
\end{multicols}
\end{enumerate}
\end{ex}

\subsection{独立事件与乘法定理}
先看下面的问题:“餐桌上摆着10把外形一样的餐刀,其中7把是不锈钢的;3把是镀铬的,记事件$A=$“甲从中任取一把恰为镀铬的”;事件$B=$“乙从中任取一把恰为镀铬的”.

显然,当只有甲单独取用时:$\Pr(A)=\frac{3}{10}$;当只有乙单独取用时:$\Pr(B)=\frac{3}{10}$.

下面再考虑乙先取用的两种情况:
\begin{enumerate}[(1)]
\item 当乙先取用且取得的是镀铬的,用后归还,这相当于返回抽样.

由于这时甲仍是从10把餐刀中取用,其中镀铬的也仍是3把,所以
“乙取得的是镀铬的条件下(用后归还),甲取得的是镀铬的”这一事件的概率:
\[\Pr(A|B)=\frac{3}{10}=\Pr(A)\]
根据同样的分析:$\Pr(B|A)=\frac{3}{10}=\Pr(B)$
\item 当乙先取用且取得是镀铬的,尚未归还甲就取用,这相当于不返回抽样.

这时由于事件$B$的发生,只剩下9把餐刀,其中镀铬的剩下2把,所以
“乙取得的是镀铬的条件下(用后未归还),甲取得的是镀铬的”这一事件的概率:
\[\Pr(A|B)=\frac{2}{9}\ne \Pr(A)\]
根据同样的分析:$\Pr(B|A)=\frac{2}{9}\ne \Pr(B)$
\end{enumerate}

总结上面的分析得出:当乙(甲)的取用不影响甲(乙)的取用,则有下面两等式成立
\[\Pr(A|B)=\Pr(A),\qquad \Pr(B|A)=\Pr(B)\]

当乙(甲)的取用影响到甲(乙)的取用,则:
\[\Pr(A|B)\ne \Pr(A),\qquad \Pr(B|A)\ne \Pr(B)\]

我们定义:如果对于事件$A$和$B$来说,有
\[\Pr(A|B)= \Pr(A),\qquad \Pr(B|A)= \Pr(B)\]
同时成立,那么称事件$A$和事件$B$为相互独立事件,或称事件$A$和$B$相互独立,否则,就是相关的.

对本例,事实上用后归还的话,不管“乙取用为镀铬的”这一事件发生与否,对“甲取用为镀铬的”概率是没有影响的,反之也一样。

因此,通俗地说,事件$A$与$B$相互独立,就是:事件$A$发生与否对事件$B$发生的概率没有影响,同样,事件$B$发生与否对事件$A$发生的概率也没有影响,例如:同时抛掷两枚硬币,记$A=$“第一枚正面向上”,$B=$“第二枚正面向上”;因第一枚是否正面向上对第二枚正面向上的概率没有影响,反之也一样,所以事件$A$与$B$相互独立.

容易看出:当事件$A$和$B$相互独立时,则事件$A$和$\overline{B}$;$\overline{A}$和$B$;$A$和$B$都相互独立的。(读者自己证明)

当事件$A$和$B$相互独立时,上节公式(1)
\[\Pr(AB)=\Pr(A|B)\cdot \Pr(B),\qquad \Pr(AB)=\Pr(B|A)\cdot \Pr(A)\]
变为
\begin{equation}
    \Pr(AB)=\Pr(A)\cdot \Pr(B) \tag{3}
\end{equation}

这就是说:两个相互独立的事件同时发生的概率等于每个事件发生的概率的乘积.

公式(1)和(3)通称为概率的乘法定理。

一般地,若$A_1,A_2,\ldots,A_n$为相互独立事件,则有:
\begin{equation}
    \Pr(A_1\cdot A_2\cdots A_n)=\Pr(A_1)\cdot \Pr(A_2)\cdots \Pr(A_n)\tag{4}
\end{equation}
独立性的概念在概率中是十分重要的,它有广泛的应用. 在实际问题中判断事件是否独立是很重要的.

\begin{example}
    甲乙两人进行一次射击,如果两人射中目标的概率甲为0.6,乙为0.7,计算:
\begin{enumerate}[(1)]
\item 两人都击中目标的概率;
\item 其中恰有一人击中的概率;
\item 至少有一人击中目标的概率。
\end{enumerate}
\end{example}

\begin{solution}
\begin{enumerate}[(1)]
    \item “两人都击中目标”($C$),就是“甲击中目标”($A$),且“乙击中目标”($B$),故$C=A\cdot B$. 事件$A$和$B$显然相互独立,所以
  \[  \Pr(C)=\Pr(A\cdot B)=\Pr(A)\cdot \Pr(B)=0.6\x0.7=0.42\]
    \item “其中恰有一人击中”($D$),就是“甲击中,乙未击中”($A\cdot \overline{B}$)或“甲未击中,乙击中”($\overline{A}\cdot B$),故$D=A\cdot \overline{B}=\overline{A}\cdot B$,显然事件$A\cdot \overline{B}$与$\overline{A}\cdot B$在两人各射击一次时是不可能同时发生的,即事件$A\cdot \overline{B}$与$\overline{A}\cdot B$互斥,所以
\[\begin{split}
    \Pr(D)&=\Pr(A\cdot \overline{B}+\overline{A}\cdot B)=\Pr(A\cdot \overline{B})+\Pr(\overline{A}\cdot B)\\
    &=0.6\x(1-0.7)+(1-0.6)\x 0.7=0.18+0.28=0.46
\end{split}\]
    \item “至少有一人击中”($E$),就是“甲击中,乙未击中”($A\cdot \overline{B}$)或“甲,击中,乙击中”(${A}\cdot B$),或“甲未中乙击中”($\overline{A}\cdot B$),故$E=A\cdot \overline{B}+A\cdot B+\overline{A}\cdot B$,所以
\[\begin{split}
    \Pr(E) &= \Pr(A\cdot \overline{B}+A\cdot B+\overline{A}\cdot B)\\
    &=\Pr(A\cdot \overline{B})+\Pr(A\cdot B)+\Pr(\overline{A}\cdot B)\\
    &=0.6\x0.3+0.6\x0.7+0.4\x0.7\\
&=0.18+0.42+0.28=0.88
\end{split}\]

此小题还可采用下面解法:
“没有一个击中”($\overline{E}$),就是“甲未击中”($\overline{A}$)且“乙未击中”($\overline{B}$),故$\overline{E}=\overline{A}\cdot \overline{B}$. 所以:
$$\Pr(\overline{E})=\Pr(\overline{A}\cdot \overline{B})=\Pr(\overline{A})\cdot\Pr(\overline{B})=0.4\x0.3=0.12$$
因此:$\Pr(E)=1-\Pr(\overline{E})=1-0.12=0.88$
\end{enumerate}
\end{solution}

\begin{example}
   假设每枚地-空导弹击中飞机的概率均为0.8,如要有99\%的把握击中来犯敌机,问需几枚导弹同时发射? 
\end{example}


\begin{solution}
设需$n$枚地空-导弹同时发射,

记事件$A=$“敌机被击中”,则“敌机未被击中”($\overline{A}$),就是“第1枚未击中”($\overline{A}_1$),且“第2枚未击中”($\overline{A}_2$),……,“第$n$枚未击中”($\overline{A}_n$)同时发生。

故$\overline{A}_1\cdot \overline{A}_2\cdots \overline{A}_n$,而$\overline{A}_1, \overline{A}_2,\ldots,\overline{A}_n$是彼此独立的,所以
\[\Pr(\overline{A})=\Pr(\overline{A}_1)\cdot \Pr(\overline{A}_2)\cdots\Pr(\overline{A}_n)=(1-0.8)^n=0.2^n\]
要求击中概率达99\%,即$\Pr(A)=0.99$,故
\[\Pr(\overline{A})=1-\Pr(A)=0.01\]
所以$n$应满足:$0.2^n=0.01$. 取对数
\[\begin{split}
    n\lg0.2&=\lg0.01\\
n&=\frac{\lg0.01}{\lg0.2}=\frac{-2}{-0.699}\approx 3
\end{split} \]
答:3枚导弹同时发射能以99\%的把握击中来犯敌机.    
\end{solution}

\begin{example}
    一个元件能正常工作的概率$r$称为该元件的可靠性,由元件组成的系统能正常工作的概率称为系统的可靠性。今设所用元件的可靠性都为$r\; (0<r<1)$,且每个元件能否正常工作是相互独立的。试求下面两系统的可靠性,并

\begin{enumerate}[(1)]
    \item 附加通路系统,它由两条$n\; (n\ge 2)$个元件串联成的通路并联而成,如下图:
\begin{center}
    \begin{circuitikz}[european]
\draw(0,0)--(.5,0)--(.5,.5)to [R](2.5,.5) to [R] (4.5,.5)--(5,.5);
\draw(6,.5) to [R](7.5,.5);
\draw(.5,0)--(.5,-.5)to [R](2.5,-.5) to [R](4.5,-.5)--(5,-.5);
\draw[dashed](4.5,-.5)--(6,-.5);
\draw[dashed](4.5,.5)--(6,.5);

\draw(6,-.5) to [R](7.5,-.5);
\draw(7.5,-.5)--(7.5,.5);
\draw(7.5,0)--(8.25,0);


\foreach \x/\y in {3/1,7/2,13.5/$n$}
{
    \node at (\x/2,.5){\y};
    \node at (\x/2,-.5){\y};
}

    \end{circuitikz}
\end{center}
    \item 附加元件系统,它由$n$对并联元件串联而成,如下图:
\begin{center}
    \begin{circuitikz}[european]
\draw(0,0)--(.5,0)--(.5,.5) to [R] (2,.5)--(2,-.5) to [R](.5,-.5)--(.5,0);
\draw(2.5,0)--(3,0)--(3,.5) to [R] (4.5,.5)--(4.5,-.5) to [R](3,-.5)--(3,0);
\draw(2,0)--(2.5,0);
\draw(4.5,0)--(5,0);
\draw(8.5,0)--(8,0);
\draw[dashed](6,0)--(5,0);
\draw(6,0)--(6.5,0)--(6.5,.5) to [R] (8,.5)--(8,-.5) to [R](6.5,-.5)--(6.5,0);
\foreach \x/\y in {2.5/1,7.5/2,14.5/$n$}
{
    \node at (\x/2,.5){\y};
    \node at (\x/2,-.5){\y};
}
    \end{circuitikz}
\end{center}
\end{enumerate}
\end{example}

\begin{solution}
    系统(1)有两条通路. 对每条通路来说,必须当每个元件都正常工作时才正常工作,故其可靠性为:$R_c=r^n$

    所以每条通路发生故障的概率为:$1-r^n$.

    系统发生故障,必定是两通路同时发生故障. 故系统发生故障的概率为:$(1-r^n)^2$.

    因此系统(1)的可靠性为:$$R_s=1-(1-r^n)^2=r^n(2-r^n)$$
    所以$R_s>R_c$,即系统在附加了一条通道后,比一条通路增加了可靠性。

    现在考虑系统(2). 从上述讨论易得每对并联元件的可靠性为:$$R'=1-(1-r)^2=r(2-r)$$
    因系统是由各对并联元件串联而成,故其可靠性为:
\[    R_{s'}=(R')^n=r^n(2-r)^n\]
    显然$R_{s'}>R_c$,即用附加元件的方法也能增加系统的可靠性。

两种系统所用的元件一样,其优劣就看可靠性的大小了. 为比较$R_s$与$R_{s'}$的大小,令
$Q_n=\frac{R_{s'}}{R_s}\; (n\ge 2)$,则
\[Q_n=\frac{(2-r)^n}{2-r^n}\]
当$n=2$时:
\[(2-r)^2=2-r^2+2(1-r)^2>2-r^2\]
所以$Q_2>1$

假设当$n=k$时,$Q_k>1$,即$(2-r)^k>2-r^k$,则
\[\begin{split}
 (2-r)^{k+1}>(2-r^k)\cdot (2-r)&=4-2r^k-2r+r^{k+1}\\
&=2-r^{k+1}+2(1-r)(1-r^k)>2-rk+1   
\end{split}\]
即当$n=k+1$时也有$Q_{k+1}>1$

所以对于任意大于等于2的自然数$n$,$Q_n>1$总成立,即$R_{s'}>R_s$,所以系统(2)较系统(1)可靠.
\end{solution}

\begin{ex}
\begin{enumerate}
    \item 一张牌随机地从一付扑克牌中抽取,令:    $A=$“这张牌是梅花”;    $B=$“这张牌是个$K$”;    $C=$“这张牌是高级的”(即$J$、$Q$、$K$、$A$)
    
    问:$A,B$;$A,C$和$B,C$的哪一对是独立事件?
    
    \item 掷三个硬币,令:    $A=$“不多于一个正面”;    $B=$“每面至少有一个”。
    
    问:$A$和$B$是独立的吗?
\end{enumerate}
\end{ex}

\section*{习题3.2}
\begin{enumerate}
    \item 一个盒子中装有5只好灯泡,一个调皮的孩子又放进三只坏灯泡,假如八只灯泡里先取出1只(不放回),再取出第2只,求两只都是好灯泡的概率.
    \item 在$1,2,\ldots,100$中随机选出一个数,求满足下列条件的概率:
\begin{enumerate}[(1)]
\item 已知选出的数是偶数的条件下,能被5整除;
\item 已知它是被4整除的条件下它被10整除.
\end{enumerate}

\item 一个口袋内装有2个白球和2个黑球,把“从中任意摸出一个球,得到白球”叫做事件A,把“从剩下的3个球中任意摸出一个球,得到白球”叫做事件B.事件A与事件B是相互独立的吗?再求在先摸出白球后,再摸出 白球的概率,及先摸出黑球后,再摸出白球的概率,并以求出结果验证你上面的结论.
\item 一学校的学生按照下表分类.
\begin{center}
\begin{tabular}{c|cc}
    \hline
   & 戴眼镜&    不戴眼镜\\
   \hline
    男孩&    120&    280\\
    女孩&    180& 420   \\
    \hline
\end{tabular}
\end{center}
令:事件$A=$“学生是男孩”,$B=$“戴眼镜的学生” 

问:
\begin{multicols}{2}
  \begin{enumerate}[(1)]
    \item $A$和$B$
    \item $A$和$\overline{B}$
    \item $\overline{A}$和$\overline{B}$是相互独立事件吗?
\end{enumerate}  
\end{multicols}

\item 生产一种零件,甲车间的合格率是90\%,乙车间的合格率是97\%,从它们生产的零件中各抽取一件,都抽到合格品的概率是多少?
\item 有一问题,在半小时内,甲能解决它的概率是$\frac{1}{2}$,乙能解决它的概率是$\frac{1}{3}$,如果两人都试图独立地在半小时内解决它,计算:
\begin{multicols}{2}
\begin{enumerate}[(1)]
\item 两人都未解决的概率;    \item 问题得到解
决的概率.
\end{enumerate}
\end{multicols}

\item 将一个硬币连掷5次,5次都出现正面的概率是多少?
\item 制造一种零件,甲机床的废品率是0.04,乙机床的废品率为0.05,从它们制造的产品中各任抽一件,其中恰有一件废品的概率是多少?
\item 电子设备的某一部件由9个元件组成,如果其中任何一个元件坏了,这个部件就不能工作,假定每个元件能使用到3000小时的概率是0.99,计算这个部件能工作3000小时的概率.
\item 甲、乙两人解出习题的概率分别为$\frac{2}{3}$和$\frac{4}{5}$,求下面的概率.
\begin{multicols}{2}
\begin{enumerate}[(1)]
    \item 只有1人解出习题:
    \item 至少1人解出习题.
\end{enumerate}
\end{multicols}


\end{enumerate}


\section{独立重复试验}
某射手射击一次,击中目标的概率是0.9,他射击4次,恰好击中3次的概率是多少?

如果我们把射击一次看作一次试验,那么射击4次就相当于连续依次做4次试验,这四次试验有下面的特点:
\begin{enumerate}[(1)]
    \item 这4次试验的结果彼此无关,是相互独立的。
    \item 每一个试验的结果只有两个:“击中目标”($A$)或“击不中目标($\overline{A}$),且每次试验中“击中目标”的概率$\Pr(A)$都是0.9,从而“击不中目标”概率$\Pr(\overline{A})$都是0.1
\end{enumerate}

将上面的例子的特点,推广到一般情形,我们把满足上面两个条件的试验,叫独立重复试验,简称独立试验. 这一类的概率问题称为独立试验概型. 若在试验中,我们只关心出现的事件$A$或$\overline{A}$,这种独立试验概型又称为贝努里(J. Bernoulli)概型. 上面问题(及后面将要讨论的)就是属于贝努里概型.

我们继续讨论其解法:

令${A}_i=$“第$i$次击中”,$i=1,2,3,4$,则$\overline{A}_i=$“第$i$次没有击中”,射击4次击中3次,共有下面几种情况:
\[\overline{A}_1A_2A_3A_4;\quad A_1\overline{A}_2A_3A_4;\quad A_1A_2\overline{A}_3A_4;\quad A_1A_2A_3\overline{A}_4\]
上述每一种情况,都可以看成是在4个位置上取出3个写上$A$,另一个写上$\overline{A}$,所以这些情况的种数等于从4个元素中,取出3个元素的组合数$\comb{3}{4}$,即4种.

由于4次射击,其结果是相互独立的,根据乘法定理,前3次击中,第4次未击中的概率为:
\[\Pr(A_1A_2A_3\overline{A}_4)=\Pr(A_1)\cdot \Pr(A_2)\cdot \Pr(A_3)\cdot \Pr(\overline{A}_4)=0.9^3\x(1-0.9)^{4-3}\]
同理:
$$\Pr(A_1A_2\overline{A}_3A_4)=\Pr(A_1\overline{A}_2A_3A_4)=\Pr(\overline{A}_1A_2A_3A_4)=0.9^3\x(1-0.9)^{4-3}$$
这就是说,上面射击4次,击中3次的4种情况的概率都是$0.9^3\x(1-0.9)^{4-3}$.因为这4种情况彼此互斥,根据加法定理,射击4次,击中3次的概率
\[\begin{split}
    P&=\Pr(A_1A_2A_3\overline{A}_4)+\Pr(A_1A_2\overline{A}_3A_4)+\Pr(A_1\overline{A}_2A_3A_4)+\Pr(\overline{A}_1A_2A_3A_4)\\
    &=\comb{3}{4}\x 0.9^3\x(1-0.9)^{4-3}\\
    &=4\x 0.9^3\x0.1\approx 0.29
\end{split}\]

如果射击4次改为$n$次,击中3次改为$k$次,也可得出类似的结果.

一般地,如果在一次试验中事件$A$发生的概率是$P$,那么在$n$次独立试验中事件$A$恰好发生$k$次的概率:
\[P_n(k)=\comb{k}{n}\cdot P^k(1-P)^{n-k}\]

在应用这个公式的时候,首先要分析实际问题是否满足独立重复试验的两个条件.


\begin{example}
    某气象站天气预报的准确率为80\%,计算:
\begin{enumerate}[(1)]
\item 5次预报恰有4次准确的概率;
\item 5次预报至少有4次准确的概率;
\item 5次预报准确的不多于4次的概率;
\item 5次预报准确的不少于4次的概率.
\end{enumerate}

\end{example}

\begin{solution}
\begin{enumerate}[(1)]
    \item 设$A=$“预报一次,结果准确,”预报5次相当于作5次独立试验,知$\Pr(A)=0.8$,所以5次预报恰有4次准确的概率:
\[P_5(4)=\comb{4}{5}\x0.8^4\x(1-0.8)^{5-4}=5\x0.8^4\x0.2\approx 0.41\]
\item 5次预报都准确的概率:
\[P_5(5)=\comb{5}{5}\x0.8^5\x(1-0.8)^{5-5}=0.85^5\approx 0.33\]
因此5次预报至少有4次准确的概率:
\[P=P_5(4)+P_5(5)=0.41+0.33=0.74\]
\item 5次预报准确的不多于4次的概率:
\[\begin{split}
    P&=P_{5}(0)+P_{5}(1)+P_{5}(2)+P_{5}(3)+P_{5}(4)\\
&={\rm C}_{5}^{0}\times 0.8^{0}\times(1-0.8)^{5-0}+{\rm C}_{5}^{1}\times0.8^{1}\times(1-0.8)^{5-1}+{\rm C}_{5}^{2}\times0.8^{2}\times(1-0.8)^{5-2}\\
&\qquad +{\rm C}_{5}^{3}\times0.8^{3}\times(1-0.8)^{5-3}+{\rm C}_{5}^{4}\times 0.8^{4}\times (1-0.8)^{5- 4}\\
&=0.2^{5}+5\times0.8\times0.2^{4}+10\times0.8^{2}\times0.2^{3}+10\times0.8^{3}\times0.2^{2}+5\times0.8^{4}\times0.2\\
&=0.00032+0.0064+0.0512+0.2048+0.4096\approx0.67
\end{split} \]
或
\[P=1-P_{5}(5)\approx 1-0.33=0.67\]

\item 5次预报准确的小于4次的概率:
\[\begin{split}
    P&=P_{5}(0)+P_{5}(1)+P_{5}(2)+P_{5}(3)\\
    &=0.00032+0.0064+0.0512+0.2048\approx 0.26
\end{split}\]
或
\[P=1-P_{5}(4)-P_{5}(5)=1-0.4096-0.33\approx 0.26\]
\end{enumerate}
\end{solution}

\begin{example}
    设有$\ell$升经过紫外线消毒的自来水,其中含有$n$个大肠菌. 今从其中任取一升水检验,问在取出的一升水中含有$k\; (k=0,1,\ldots,n)$个大肠菌的概率是多少?
\end{example}

\begin{solution}
  对于每一个大肠菌来说只有两种可能结果:落入或不落入取出的一升水中,且可以认为每个大肠菌落入与否是彼此独立的. 这样$n$个大肠菌落入所取出的一升水中可以看作做了$n$次独立试验,每次试验结果是某一大肠菌落入该升水($A$)或不落入该升水($\overline{A}$). 而每个大肠菌落入该升水的概率为$P=\Pr(A)=\frac{1}{\ell}$,故在取出的一升水中有$k$个大肠菌
的概率为:
\[P_n(k)=\comb{k}{n}\cdot P^k\cdot (1-P)^{n-k}=\comb{k}{n}\cdot \left(\frac{1}{\ell}\right)^k\cdot \left(1-\frac{1}{\ell}\right)^{n-k}  \]
\end{solution}



\section*{习题3.3}
\begin{enumerate}
    \item 生产一种零件,出现次品的概率是0.04,生产这种零件4件,其中恰有1件次品、恰有2件次品、至多有一件次品的概率各是多少?
    \item 某工厂生产中出现次品的概率为0.01,进行重复抽样检查,共取3个样品,求其中次品数等于0, 1, 2, 3的概率.
    \item 设一射手每次中靶的概率为3/5,求四次射击中:
\begin{multicols}{2}
\begin{enumerate}[(1)]
\item 击中一次,    \item 第三次击中,
\item 击中两次,    \item 第二、三两次击中;
\item 至少击中一次的概率。
\end{enumerate}    
\end{multicols}
    \item 有甲乙丙三批罐头,每批100个,其中各有一个是不合格的. 从三批罐头中各抽出一个,计算:
\begin{enumerate}[(1)]
  \item 3个中恰有一个不合格的概率;
\item 3个中至少有一个不合格的概率. 
\end{enumerate}
    \item 某工厂生产过程中出现次品的概率为0.05,每一百个产品为一批,检查产品质量时,在每批中任取一半来检查,如果发现次品不多于1个,则可以认为这批产品是合格的,求一批产品被认为是合格的概率.
\end{enumerate}

\section*{本章内容要点}
一、本章通过大量实例,初步介绍了随机事件的概率及其计算公式.

二、在一定条件下,可能发生也可能不发生的事件,称为随机事件.反映随机事件出现的可能性的大小的一个量,叫做概率.

三、具有有限性、等可能性的特征的随机现象,我们称为古典概型. 如果试验的基本事件总数为$n$,随机事件$A$所包含的事件数为$m$,那么,$A$的概率就定义为$\frac{m}{n}$,记作
\[\Pr(A)=\frac{m}{n}\]
特别地,必然事件$U$的概率为$\Pr(U)=1$; 不可能事件$V$的概率为$\Pr(V)=0$,显然,$0\le \Pr(A)\le 1$.

四、不可能同时发生的两个事件叫 做互斥事件,当$A$,$B$互斥时,
\begin{equation}
    \Pr(A+B)=\Pr(A)+\Pr(B)\tag{*}
\end{equation} 

五、在事件$B$出现的条件下,事件$A$的出现的概率,我们称为$A$关于$B$的条件概率,记作$\Pr(A|B)$. 且有公式
\begin{equation}
   \Pr(A|B)=\frac{\Pr(AB)}{\Pr(B)}\tag{**} 
\end{equation}

六、对于事件$A$,$B$来说,如果有
\[\Pr(A|B)=\Pr(A),\qquad \Pr(B|A)=\Pr(B)\]
同时成立,那么就称事件$A$,$B$为相互独立事件,两个相互
独立的事件同时发生的概率为:
\begin{equation}
    \Pr(A\cdot B)=\Pr(A)\cdot \Pr(B) \tag{***}
\end{equation}

应该注意区别互斥事件、独立事件这两个概念,注意运用公式(*), (***)的前提条件.

七、其中必有一个发生的两个互斥事件,叫做对立事件,当$A$,$B$是对立事件时,有
\[\Pr(A)=1-\Pr(B)\]

八、具有重复试验的结果是相互独立的,而且每次试验的结果只有$A$与$\overline{A}$的特征的随机现象,我们称为重复独立试验概型(又称贝努里概型). 如果事件$A$在一次试验中发生的概率是$P$,那么,它在$n$次独立重复试验中恰好发生$k$次的概率是
\[P_n(k)=\comb{k}{n}P^k(1-P)^{n-k}\]


\section*{复习题三}
\begin{enumerate}
    \item 某种集成电路用到2000小时还能正常工作的概率是94\%,使用到3000小时仍能正常工作的概率是78\%,问已经工作了2000小时的集成电路,能继续工作到3000小时的概率是多少?
    \item 若$M$件产品中包含$m$件废品,今在其中任取两件,求:
\begin{enumerate}[(1)]
\item 已知取出两件中有一件为废品的条件下,另一件也是废品概率;
\item 已知两件中有一件不是废品的条件下,另一件是废品的概率.
\end{enumerate}
    \item 一只瓶含有6个黑球,4个红球和2个绿球,两个球是逐个而不再放回地随机取出,求两个球都是:(1)黑的,(2)红的.(3)绿的,(4)相同的颜色的概率.
    \item 甲袋装有白球3个,黑球5个;乙袋中装有白球4个,黑球6个,现在从甲袋中随意地取出一个球放入乙袋,充分掺混后再从乙袋随意地取出一个球放入甲袋,求这时甲袋中的白球数成下面的情况的概率:
\begin{multicols}{2}
\begin{enumerate}[(1)]
\item 增加,    \item 不变.
\end{enumerate}
\end{multicols}

    \item 在$1,2,\ldots,100$中任取一数,求它既能被2整除又能被5整除的概率,又它能被2整除或能被5整除的概率是多少?
\item 假设总人口中男孩和女孩的比例是1:1,在所有恰有三个孩子的家庭中,求:
\begin{enumerate}[(1)]
\item 一个家庭有两个女孩和一个男孩的概率;
\item 已知最大的孩子是女孩的条件下,一个家庭有两个女孩和一个男孩的概率.
\end{enumerate}

\item 在一个恰有三个孩子的家庭里,若已知:已有(1)一个女孩,(2)一个男孩,求恰有两个女孩的概率是多少?
\item 甲、乙、丙三儿童,甲有红球$a$只,白球$b$只;乙全是红球;丙红、白球都有,三人各拿出1球,其中甲是随机地取出的,丙拿红球或白球的概率都是$\frac{1}{2}$,然后让甲从取出的三个球任意取回一个,求甲的球成下列情况的概率:
\begin{multicols}{3}
\begin{enumerate}[(1)]
\item 红球增加;
\item 红球个数不变;
\item 红球减少.
\end{enumerate}
\end{multicols}

\item 下图是某市街道地图的一部分,一人计划从角落$A$处散步到$B$处,这里往北要走过7条街,往东要过4条街,求:
\begin{enumerate}[(1)]
    \item 从$A$到$B$有几种不同的走法?
    \item 他通过$G$或者$H$的概率是多少?
    \item 他通过标有$G$的拐角处的概率是多少?
    \item 已知他通过$G$或$H$. 求他经过拐角$J$的条件概率.
\end{enumerate}

\begin{center}
\begin{tikzpicture}[xscale=.6, yscale=.4]
\foreach \y in {0,1,2,...,9}
{
    \draw[gray](-.5,\y)--(6.5,\y);
}    
\foreach \x in {0,1,2,...,6}
{
    \draw[gray](\x,-.5)--(\x,9.5);
}    
\tkzDefPoints{1/1/A, 3/3/H, 2/6/G, 4/6/J, 5/8/B}
\tkzDrawPoints(A,H,G,J,B)
\tkzLabelPoints[above left](A,G)
\tkzLabelPoints[above right](B,J,H)


\end{tikzpicture}
\end{center}



\item 轰炸机要完成它的使命,必须驾驶员找到了轰炸目标,投弹员又投中了目标,设驾驶员甲和乙找到目标的概率分别为0.9和0.8;又设投弹员丙和丁在驾驶员找到了目标的条件下投中目标的概率分别为0.7和0.6.现在配备两架轰炸机的人员,问甲乙与丙丁应该怎样搭配才能使完成使命有较大的概率?又这个概率比起另一种配合大多少?(只要有一架飞机投中目标即完成使命).
\item 一个工人看管四台机床,如果在一小时内这些机床不需要人去照顾的概率,第一台是0.79,第二台是0.79,第三台是0.80,第四台是0.81,假设各台机床是否需要照顾相互之间没有影响,计算在这个小时内,这四台机床都不需要照顾的概率.
\item 有甲、乙、丙三批罐头,每批分别有100、120、
150个,其中各有3个是不合格的,从三批罐头中各抽出一个,计算:
\begin{enumerate}[(1)]
\item 3个恰有一个合格的概率;
\item 3个中至少有一个不合格的概率。
\end{enumerate}

\item 甲、乙、丙三步枪手射中某目标的概率各为0.7、0.8及0.85;现向该目标各发一弹,问射中3, 2, 1, 0弹的概率各为多少?
\item 两部车床独立地工作,每部车床不需要工人照顾的概率分别为0.9和0.85,求
\begin{enumerate}[(1)]
\item 都不需要照顾的概率;
\item 恰好一台需要照顾的概率;
\item 同时需要照顾的概率.
\end{enumerate}

\item 有一乘客在电车汽车联合站候车,任何一种车都能使他到达目的地. 如果在五分钟内电车到站的概率是$\frac{1}{2}$,汽车到站的概率是$\frac{1}{3}$. 问这个乘客在五分钟内能够登上任何一种车的概率是多少?
\item 甲袋子内有$m$个白球,$n$个黑球,乙袋子内有$n$个白球,$m$个黑球. 从两个袋子内各任意摸出一个,得到一个白球一个黑球的概率是多少?
\item 一箱螺钉共十万只,次品率为0.004,问事件“从中任意抽验一只,是次品;然后又抽验两只,这两只中至少有一只是次品”的概率有多少?
\item 街道交叉口上交通的畅通程度可用任意给定的一秒钟内有一部车子通过的概率P来描述;假定在不同时刻车子的经过是互不影响的,如果一个步行者只能在三秒钟没有车子通过的情形下才能跨过街道,求他恰巧等待0, 1, 2, 3秒钟的概率.
\item 某仪表有$m$个同样的电子管,其中任一个电子管损坏,这个仪表就不能工作,如果在某段时间内每个电子管损坏的概率是$P$,计算在这段时间内,这个仪表不能工作的概率。
\item 在下图所示的开关电路中,开关$A$、$B$、$C$开或关的概率均独立地为$\frac{1}{2}$,求灯点亮的概率.
\begin{center}
\begin{circuitikz}
\draw(.5,1.5) to [normal open switch, l=$A$] (2.5,1.5) -- (3,1.5);
\draw(3,1.5) to [normal open switch, l=$B$]  (3.5,1.5)--(4,1.5);
\draw(.5,1.5)--(.5,.5) to [normal open switch, l=$C$] (4,.5)--(4,1.5);
\draw(4,1)--(4.5,1)--(4.5,-0.5) to [lamp] (1.5,-0.5) to [battery] (0,-0.5)--(0,1)--(.5,1)--(.5,1.5);
\end{circuitikz}
\end{center}


\item 某种大炮击中目标的概率是0.3,只要以多少门这样的大炮同时射击一次,就可以使击中目标的概率:
\begin{multicols}{2}
\begin{enumerate}[(1)]
\item 超过95\%;    \item 不小于0.99.
\end{enumerate}
\end{multicols}

\item 某售货员在三个柜面上售货,如果在某一小时内柜面需要售货员照顾的概率,第一柜面是0.9,第二柜面为0.8,第三柜面为0.7,假定各个柜面是否需要照顾相互之间不受影响,计算在这个小时内,至少有一个柜面需要售货员的概率.
\item 甲、乙、丙三猎手向同一野猪射击,设甲射中的概率为0.4;乙射中的概率为0.5;丙射中的概率为0.7,若只有一人射中,野猪被击毙的概率为0.2;若其中二人射中,则野猪被击毙的概率为0.6;若三人同时射中,则野猪必定被击毙。求野猪被击毙的概率。
\item 某射手射击一次,击中目标的概率是0.9,他射击了四次。分别写出恰好击中4次,3次,2次,1次,0次的概率,并将它们与$(0.9+0.1)^4$的展开式的各项进行比较,你可得出什么结论?
\item 一次测量中出现正误差和负误差的概率都是$\frac{1}{2}$,在3次测量中,恰好出现2次正误差的概率是多少?恰好出现两次负误差的概率是多少?
\item 一头病牛用某药后被治愈的概率为95\%,计算服用这种药的4头病牛中至少有3头被治愈的概率.
\item 在乒乓球比赛中,甲、乙两人球艺相当,胜负机会相同,试比较下列概率哪一个较大?一方在:
\begin{enumerate}[(1)]
\item 4盘中胜3盘与8盘中胜5盘;
\item 4盘中胜3盘以上与8盘中胜5盘以上;
\item $2n$盘中胜的盘数不多于$n$与$2n$盘中胜$n$盘以上.
\end{enumerate}

\item 每一用户在一小时内向总机拔话的概率等于0.02,该总机有300个分机,试问在一小时内有不少于7个分机向总机拔话的概率是多少?

\item 某一批蚕豆种籽,如果每一粒发芽的概率为90\%,播下5粒种籽,计算:
\begin{enumerate}[(1)]
    \item 其中恰有4粒发芽的概率;
    \item 其中恰有4粒未发芽的概率.
\end{enumerate}

\item 在四次独立试验中,事件$A$至少出现一次的概率为$\frac{80}{81}$,求事件$A$在各次试验中出现的概率.
\item 某小组有10台各为7.5千瓦的机床,如果每台机床的使用情况是互相独立的,且每台机床平均每小时开动12分钟,问全部机床用电超过48千瓦的可能性有多大.
\item 某一学校有730名学生,对每一个人来说,他的生日在某一天的概率设为$\frac{1}{360}$,求:
\begin{enumerate}[(1)]
\item 恰有3名学生的生日为元月1日;
\item 生日在元月1日学生不多于3人;
\item 没有一个学生的生日在元月1日;
\item 至少有一个学生的生日在元月1日的概率.
\end{enumerate}

\item 两个篮球运动员在罚球线投球的命中率分别是0.7与0.6,每人投球3次,计算两人都投进2球的概率.
\item 一个织布工人看管12台布机、在$a$小时内一台布机需要工人照顾的概率等于$\frac{1}{3}$
\begin{enumerate}[(1)]
\item 在$a$小时内,有4台布机需要照顾的概率;
\item 在$a$小时内,需要照顾的布机不多于6台的概率.
\end{enumerate}

\item 在多次试验中,当事件$A$出现不少于3次时,事件B才出现.而事件$A$在每次试验中出现的概率等于0.3,若
\begin{multicols}{2}
    \begin{enumerate}[(1)]
    \item 作了5次试验;
    \item 作了7次试验
\end{enumerate}
\end{multicols}
试求$\Pr(B)$.

\item 一个通讯小组有两套通讯设备,只要其中有一套设备能正常工作,就能进行通讯,每套设备出3个部件组成.只要其中有一个部件出故障,这套设备就不能正常工作,如果在某段时间内每个部件不出故障的概率都是$P$,计算在这段时间内能进行通讯的概率.
\item 甲、乙两个篮球队员各投蓝3次,每次投篮时投中得分的概率各为0.6和0.7.求
\begin{multicols}{2}
\begin{enumerate}[(1)]
    \item 得分相同的概率;
    \item 甲得分比乙多的概率.
\end{enumerate}    
\end{multicols}

\end{enumerate}


% \input{4.tex}



\backmatter

\chapter{附录:统计学简介}

\section*{平均值(均值)}

给了一组数据$x_1,x_2,\ldots,x_n$, 相应地有一个平均数
(也称为均值), 它是如下的算术平均数
\[\frac{1}{n} \sum_{i=1}^nx_i\]
通常用$\bar x$来表示它,平均数是日 常生活中常见的量。例如, 工厂中某一产品的平均日产量,农村中某一作物的平均亩产量,商品的平均销售额,学生考试的平均分数等等。自然会产生这样的问题,为什么要用平均数呢?原因是平均数具有代表性,下面来解释它的意思。

对给定的一组数据$x_1,x_2,\ldots,x_n$, 如果用一个数$C$来代表这些$x_1,x_2,\ldots,x_n$, 则$x_i-C$就表达了$x_i$与$C$的偏差, $x_i-C$的绝对值小,$C$就接近$x_i$。即用$C$来代表$x_i$就代表得好. 由于$C$要代表所有数$x_1,x_2,\ldots,x_n$,因此必须考虑几个偏差$x_{1}-C,x_{2}-C,\ldots,x_{n}-C$. 为了消除正负号的影响, 用$(x_i-C)^2$来衡量$x_i$与$C$ 的差别,于是总的差异就是$\sum\limits_{i= 1}^{n}\left ( x_{i}- C\right ) ^{2}$, 当$x_{1},x_{2},\ldots,x_{n}$确定不变时,应选取$C$使$\sum\limits_{i= 1}^{n}\left ( x_{i}- C\right ) ^{2}$达到最小值,这样的$C$才是代表$x_{1},x_{2},\ldots,x_{n}$这$n$个数据最合适的数。

下面来证明这个数就是$\bar x$
\[\begin{split}
    \sum_{i=1}^{n} ( x_{i}-C)^{2} &=\sum_{i=1}^{n} ( x_{i}-\bar{x}+\bar{x}-{C} )^{2} \\
    &=\sum_{i=1}^{n}\left[\left( x_{i}-\bar{x}\right) ^{2}+2\left( x_{i}-\bar{x}\right) ( \bar{x}-C)+(\bar{x}-C)^{2}\right]\\
    &=\sum_{i=1}^{n}\left( x_{i}-\bar{x}\right)^{2}+2\left[\sum_{i=1}^{n}\left( x_{i}-\bar{x}\right)\right](\bar{x}-C)+n(\bar{x}-C)^{2}
\end{split}\]
由于
$$\sum_{i=1}^{n} ( x_{i}-\bar{x} ) =\sum_{i=1}^{*} x_{i}-n\bar{x}=n\bar{x}-n\bar{x}=0$$
因此
$$\sum_{i=1}^{n}(x_{i}-C)^{2}=\sum_{i=1}^{n}(x_{i}-\bar{x})^{2}+n(\bar{x}-C)^{2}\ge \sum_{i=1}^{n}(x_{i}-\bar{x})^{2}$$
显然,等号成立当且仅当$\bar x=C$.

这就证明了:如果以$\sum\limits_{i=1}^n(x_i-C)^2$来衡量$C$代表$x_1,
x_{2},\ldots,x_{n}$的好坏标准,则$\bar x$是代表性最好的值.

现在来看几个特例:

\begin{example}
    如果$x_1=x_2=\cdots=x_n=a$, 则 $\bar{x}=a$.
\end{example}

\begin{proof}
    $$\bar{x}= \frac{1}{n}\sum_{i=1}^{n}x_{i}= \frac{1}{n}\sum_{i= 1}^{n}a= a$$
\end{proof}

\begin{example}
    如果$x_i=a+id$, $i=0,1,2,\ldots,n-1$,
即$x_0,x_1,\ldots,x_{n-1}$是等差数列,这时
\[\begin{split}
   \bar{x}= \frac {1}{n}\sum _{i= 0}^{n-1}x_{i}&= \frac{1}{n}\sum_{i= 0}^{n- 1}( a+ id) \\
   &= \frac{1}{n}\left [ na+ \frac {(  n- 1 )n}{2}d \right ]=  a+\frac{(n-1)d}{2}
\end{split}\]
\end{example}
例如求$1,2,\ldots, 100$的平均 数,这 时$a=d=1$, $n=100$于是$\bar{x} = 1+ \frac {99}{2}= 50.5$; 又 如求$3, 6, 9, 12, \ldots, 27 $这几 个数的均值,则$a= d= 3$, $n= 0$, 于是$\bar{x}=3+\frac{8}{2}\times 3=15$.


\section*{均方差(方差)}

从上面的讨论可以看出,平均数$\bar x$代表$x_1,\ldots,x_n$的好坏程度是由$\sum\limits^n_{i=1}(x_i-\bar x)^2$来反映的,它是$n$个数据$x_i$对
$\bar x$的偏差平方的总和. 如果用$n$除一下,就得到平均的偏差平方和,它反映了每个数据$x_i$偏离$x$的“平均”状况,因此称它为均方差,简称为方差,通常用$S^2$来表示,即有
\[S^2=\frac{1}{n}\sum^n_{i=1}(x_i-\bar x)^2\]
方差表明了一组数据$x_1,x_2,\ldots,x_n$的分散程度. $S^2$越小,表示$x_i$之间的差异小,$\bar x$的代表性就好;$S^2$越大,表示$x_i$之间的差异也大,数据便很分散,此时$\bar x$代表性就不强.

有时,我们要比较的就是两组数据的分散程度. 这用方差就很方便. 例如有两班学生,甲班有$n$个学生,乙班有$m$个学生,进行教学测验后,甲班的成绩为$x_1,x_2,\ldots,x_n$,乙班的成绩为$y_1,y_2,\ldots,y_m$,平均数$\bar x$与$\bar y$反映了两个班的平均成绩,各自方差$S^2_x$和$S^2_y$反映了两班成绩不齐的程度,方差小的班,学生之间的差异就小.

如果$x_1=x_2=\cdots=x_n=a$,则$\bar x=a$,这时方差$S^2=0$. 如果$x_i=i,\; i=1,2,\ldots,n$,则$\bar x=\frac{n+1}{2}$,而方差
\[\begin{split}
    S^2=\frac{1}{n}\sum^n_{i=1}(x-\bar x)^2 &=\frac{1}{n}\sum^n_{i=1}\left(i-\frac{n+1}{2}\right)^2\\
&=\frac{1}{n}\sum^n_{i=1}i^2-\left(\frac{n+1}{2}\right)^2\\
&=\frac{n(n+1)(2n+1)}{6}-\frac{(n+1)^2}{4}\\
&=\frac{(n+1)(4n+2-3n-3)}{12}=\frac{n^2-1}{12}
\end{split}\]

平均数$\bar x$与平均方差$S^2$是一组数据$x_1,x_2,\ldots,x_{n}$的两个重要指标,给了一组数$x_1,x_2,\ldots,x_{n}$后,如何具体地算出这两个值,我们在下一小节中来详细讨论这一问题.

\section*{计算公式}
设$x_1,x_2,\ldots,x_{n}$相应的均值和方差为$\bar x$和$S^2_x$,设$y_1,y_2,\ldots,y_{n}$相应的均值和方差为$\bar y$和$S^2_y$,下面先导出几个公式,然后利用它们来具体计算。

\begin{enumerate}
    \item 如果$y_i=x_i+a,\; i=1,2,\cdots,n$。则 $\bar {y}= \bar{x} + a$.
    
\begin{proof}
     $\bar{y} = \frac 1n\sum\limits_{i= 1}^{n}y_{i}= \frac 1n\sum\limits _{i= 1}^{n} ({x_{i}+ a}) = \frac 1n\sum\limits _{i= 1}^{n}x_{i}+ a= x+ a$
\end{proof}
    
\item 如果$y_i=bx_i,i=1,2,...n$, 则$\bar{y}=b\bar{x}$

\begin{proof}
$\bar y= {\frac 1n}\sum\limits _{i= 1}^{n}y_{i}= {\frac 1n}\sum\limits _{i= 1}^{n}bx_{i}= b {\bar {x}}$
\end{proof}

    \item 如果$y_i= bx_{i}+ a,\; i= 1, 2, \ldots, n$, 则$\bar {y}=b\bar{x} + a$
  
\begin{proof}
    由1,2立即可得.
\end{proof}  
    \item $\sum\limits _{i= 1}^{n}\left (  x_{i}- \bar {x} \right ) ^{2}= \sum\limits _{i= 1}^{n}x_{i}^{2}- n\bar {x}^{2}$

\item 如果$y_i=bx_i+a,\; i=1,2,\ldots,n$, 则$S_y^{2}=b^{2}S_{x}^{2}$

\begin{proof}
\[\begin{split}
    S_{y}^{2}= {\frac 1n}\sum\limits  _{i= 1}^{n}(y_{i}- \bar{y})^2&=\frac{1}{n}\sum\limits ^n_{i=1}\left ( bx_{i}+ a- b\bar{y} - a\right ) ^{2}\\
    &=\frac{b^{2}}{n}\sum\limits _{i=1}^{n}(x_{i}-\bar{x})^{2}=b^{2}S_{x}^{2}
\end{split}\]
\end{proof}
\end{enumerate}

公式5告诉我们,当$b=\pm1$时,$S_x^2=S_y^2$. 即一组数据同
加一个常数,或同时改变符号时,它的方差是不变的.

\begin{example}
设某地历年夏季的雨量为:(单位mm)
\[248.7\quad 249.4\quad 133.2\quad 153.5\quad 211.7 \]
求它的平均值$\bar x$及方差$S^2$
\end{example}

取$y_i=x_i-200$,所以$x_i=y_i+200$,于是$\bar x=\bar y+200$,$S^2_x=S^2_y$

列表演算如下:
\begin{center}
\begin{tabular}{ccc}
\hline 
$x_i$ & $y_i=x_i-200$ & $y^2_i$\\
\hline
248.7  & 48.7  & 2371.69\\
249.4& 49.4&2440.36\\
133.2&$-66.8$&4462.24\\
153.5&$-46.5$&2162.25\\
211.7&11.7&136.89\\
\hline
$\Sigma$ &$-3.5$& $11573.43$\\
\hline
\end{tabular}
\end{center}

所以
\[\begin{split}
    \bar x&=\bar y+200=-\frac{3.5}{5}+200=199.3\\
S^2_x&=S^2_y=\frac{1}{n}\left(\sum^n_{i=1} y^2_i-n\bar y^2\right)=\frac{1}{5}(11573.43-245)=2314.196
\end{split}
    \]

从演算过程可以看出,$y_i$的平方、求和的计算量比$x_i$的平方、求和的计算量小.这里选200是为了使$x_i-200$数字变小而且减法本身也较易.

\begin{example}
    设$i=a+id,\; i=1,2,\ldots,n$,即$x_1,x_2,\ldots,x_n$为等差数列,试求方差$S^2_n$.
\end{example}

为了便于比较,下面用两种方法求解.

\textbf{解一:} 已知$\bar x=a+\frac{n+1}{2}d$,所以$x_i-x=\left(i-\frac{n+1}{2}\right)d$.
\[\sum^n_{i=1}(x_i-\bar x)^2 =\sum^n_{i=1}\left(i-\frac{n+1}{2}\right)^2 d^2=\frac{d^2n(n^2-1)}{12} \]
所以
\[S^2_x=\frac{d^2(n^2-1)}{12} \]

\textbf{解二:} 取$y_i=\frac{x_i-a}{d}=i$, $i=1,2,\ldots,n$,于是$S^2_y=\frac{1}{d^2}S^2_x$,即$S^2_x=d^2S^2_y$. 前面已算过,这时$S^2_y=\frac{n^2-1}{12}$
所以$S^2_x=\frac{d^2(n^2-1)}{12}$.

\begin{example}
    对两组数据$x_1,x_2,\ldots,x_n$及$y_1,y_2,\ldots,y_m$, 设它们的均值及均方差分别为$\bar x,\; S^2_x; \; \bar y,\;  S^2_y$,将这两组数据合并为一组数据,求合并后的数据的均值$\bar z$及均方差$S^2_z$.
\end{example}

\begin{solution}
令$\bar z=\frac{1}{n+m}\left(\sum\limits^n_{i=1}x_i+\sum\limits^m_{i=1}y_i\right)=\frac{1}{n+m}(n\bar x+m\bar y)$

又
\[\begin{split}
    S^2_x&=\frac{1}{n+m}\left[\sum\limits^n_{i=1}(x_i-\bar z)+\sum\limits^m_{i=1}(y_i-\bar z)^2\right]\\
    &=\frac{1}{n+m}\left[\sum\limits^n_{i=1}(x_i-\bar x)^2 +n(\bar x-\bar z)^2+\sum\limits^m_{i=1}(y_i-\bar y)^2+m(\bar y-\bar z)^2\right]\\
    &=\frac{1}{n+m}\left[nS^2_x+n(\bar x-\bar z)^2+mS^2_y+m(\bar y-\bar z)^2\right]
\end{split}\]
但是$\bar z=\frac{n}{n+m}\bar x+\frac{m}{n+m}\bar y$,代入有
\[\bar x-\bar z=\frac{m}{n+m}(\bar x-\bar y),\qquad \bar y-\bar z=\frac{n}{n+m}(\bar y-\bar x)\]
\[S_{y}^{2}= \frac n{n+ m} S_{x}^{2}+ \frac m{n+ m} S_{y}^{2}+ \frac n{n+ m}\left(\frac m{n+ m}\right)^{2}(\bar x- \bar{y} ) ^{2}+\frac{m}{n+m}\left(\frac{n}{m+n}\right)^{2}(\bar{y}-\bar x)^{2}\]
所以
\[S_{y}^{2}=\frac{1}{n+m}\left(nS_{x}^{2}+mS_{y}^{2}\right)+\frac{nm}{(n+m)^{2}}(\bar{x}-\bar{y})^2\]
总之有
$$z=\frac{1}{n+m}(n\bar{x}+m\bar{y})$$
\[S_{z}^{2}=\frac{1}{n+m}\left(nS_{x}^{2}+mS_{z}^{2}\right)+\frac{nm}{(n+m)^{2}}(\bar{x}-\bar{y})^2\]
\end{solution}

特别,当$m=1$时,$S^2_y=0$,记$y_1=x_{n+1}$,这就得到增加一个新数据的递推公式
\[\begin{split}
    \bar z&= \frac{1}{n+1}(n\bar x+x_{n+1})\\
    S^2_z&=\frac{n}{n+1}S^2_x+\frac{n}{(n+1)^2}(\bar x-x_{n+1})^2
\end{split}\]

不难看出,如果从$x_1,x_2,\ldots,x_n$中减少一个数据,例如删去$x_n$,那么$x_1,x_2,\ldots,x_{n-1}$的均值$\bar x_{*}$与均方差$S^2_{*}$可以用$\bar x$与$S^2_x$及$x_n$表示,即有递推公式
\[\begin{split}
    \bar x_{*}&= \frac{1}{n-1}(n\bar x-x_{n})\\
    S^2_{*}&=\frac{n}{n-1}S^2_x-\frac{n}{(n-1)^2}(\bar x-x_{n})^2
\end{split}\]

\section*{最小二乘法}
在实际工作中经常需要从实测的数据求出变量之间的关系式. 例如年龄和血压是有关系的,随着年龄的增长,血压是会增高的,调查了几百人的年龄与血压的情况,将资料按年龄分组. 用各组的均值(年龄的均值和血压的均值),得到如下的数据:

\begin{minipage}{.35\textwidth}
\begin{center}
    \begin{tabular}{cc}
\hline
        年龄$x_i$ & 心脏收缩压$y_i$\\
\hline
35&114\\
45&124\\
55&143\\
65&158\\
75&166\\
\hline
    \end{tabular}
\end{center}
\end{minipage}\hfill
\begin{minipage}{.6\textwidth}
\centering
\begin{tikzpicture}[>=stealth, scale =.8]
\draw[->](-1,0)--(6,0)node[above]{$x$(年龄)};
\draw[->](0,-1)--(0,8)node[right]{$y$(血压)};
\foreach \x/\y in {1/35,2/45,3/55,4/65,5/75}
{
    \draw(\x,0)node[below]{$\y$}--(\x,.1);
}
\foreach \x/\y in {1/110,2/120,3/130,4/140,5/150,6/160,7/170}
{
    \draw(0,\x)node[left]{$\y$}--(.1,\x);
}
\tkzDefPoints{1/1.4/A, 2/2.4/B, 3/4.3/C, 4/5.8/D, 5/6.6/E}
\tkzDrawPoints(A,B,C,D,E)
\node at (A) [left]{$P_1$};
\node at (B) [right]{$P_2$};
\node at (C) [left]{$P_3$};
\node at (D) [left]{$P_4$};
\node at (E) [right]{$P_5$};
\draw[domain=.5:5.5, smooth, thick]plot(\x, 1.35*\x+.1);
\node[below left]{$O$};
\end{tikzpicture}

\end{minipage}

从表上的数据或图上的点来看,
血压值$y$与年龄数$x$似乎有直线的关系,但是这五个点又不正好在一条直线上,于是发生了一个问题:如何求一个直线方程,使得这条线与五个点最接近.

用$P_i$表示点$(x_i,y_i)\; i=1,2,3,4,5$.对给定的直线$\ell:\; y=a+bx$. 在直线$\ell$上取五个点,使其横坐标与$P_i$的横坐标$x_i$相同,这五个点应是
\[Q_i(x_i, a+bx_i) \qquad i=1,2,3,4,5\]
很明显,这五个点的纵坐标之差为$y_i-a-bx_i$. 用数值$\sum\limits^5_{i=1}(y_i-a-bx_i)^2$
来衡量直线$\ell$上五个点$Q_1,\ldots,Q_5$和已给的五个点$P_1,\ldots,P_5$的总差距,这个差距愈小,就认为这条直线愈接近这些点. 因此,最接近的直线应使这个差距达到最小值,下
面用和以前类似的方法来求出$a$和$b$.

已知$n$个点有坐标$(x_i,y_i)\; i=1,2,\ldots,n$.求直线$\ell:\; y=a+bx$,使和这$n$个点在上述意义下最接近,也就是求数值$a$和$b$,使
\[Q=\sum^n_{i=1}(y_i-a-bx_i)^2\]
达到最小值.

\begin{blk}{定理(最小二乘法)}
使$Q$达最小值的$a,b$为
\[\hat a=\bar y-\hat b \bar x,\qquad \hat b=\frac{\sum\limits^n_{i=1}(x_i-\bar x)(y_i-\bar y)}{\sum\limits^n_{i=1}(x_i-\bar x)^2}\]
\end{blk}

\begin{proof}
对任意数$a,b,\hat a,\hat b$,有
\[\begin{split}
Q&=\sum^n_{i=1}(y_1-a-bx_i)^2 =\sum^n_{i=1}\left[y_i-\hat a-\hat b x_i+\hat a-a+(\hat b-b)x_i\right]^2\\ 
&=\sum^n_{i=1}\left(y_i-\hat a-\hat b x_i\right)^2+2\sum^n_{i=1}\left(y_i-\hat a-\hat b x_i\right)\cdot \left(\hat a-\hat a+bx_i-bx_i\right)\\
&\qquad +\sum^n_{i=1}\left[\hat a-a+(\hat b-b)x_i\right]^2
\end{split}\]
取$\hat a,\hat b$使得
\[\sum^n_{i=1}\left(y_i-\hat a-\hat b x_i\right)=0,\qquad \sum^n_{i=1}x_i\left(y_i-\hat a-\hat b x_i\right)=0\]
那么上面
\[\begin{split}
    Q&=\sum^n_{i=1}\left(y_i-\hat a-\hat b x_i\right)^2+\sum^n_{i=1}\left[\hat a-a+(\hat b-b)x_i\right]^2\ge \sum^n_{i=1}\left(y_i-\hat a-\hat b x_i\right)^2
\end{split}\]
因此若能由$\hat a$,$\hat b$之条件唯一
引出$\hat a$,$\hat b$,那么它就是所求的使$Q$达极小值的$a$,$b$了。

将条件改变为
\[\hat a+\hat b \bar x=\bar y,\quad n\hat a \bar x+\hat b\sum^n_{i=1}x^2=\sum^n_{i=1}x_iy_i\]
由$\hat a=\bar y-\hat b\bar x$,代入后式,有
\[n\left(\bar y-\hat b\bar x\right)\bar x+\hat b\sum^n_{i=1}x^2_i=\sum^n_{i=1}x_iy_i\]
即有
\[\hat b\left[\sum^n_{i=1}(x_i-\bar x)^2\right]=\sum^n_{i=1}x_iy_i-n\bar x\bar y=\sum^n_{i=1}(x_i-\bar x)\cdot (y_i-\bar y)\]
所以\[\hat b=\frac{\sum\limits^n_{i=1}(x_i-\bar x)(y_i-\bar y)}{\sum\limits^n_{i=1}(x_i-\bar x)^2}\]
于是$\hat a=\bar y-\hat b\bar x$, 这证明了定理.
\end{proof}

回到原来的例子,下面列表进行计算.
\begin{center}
\begin{tabular}{cccccc}
\hline
$x_i$ & $y_i$&$x_i-\bar x$&$y_i-\bar y$&$(x_i-\bar x)^2$&$(x_i-\bar x)(y_i-\bar y)$\\
\hline
35&114&$-20$&$-27$&400&540\\
45&124&$-10$&$-17$&100&170\\
55&143&0&2&0&0\\
65&158&10&17&100&170\\
75&166&20&25&400&500\\
\hline
$\Sigma 275$&705&&&1000&1380\\
平均55&141\\
\hline
\end{tabular}
\end{center}

\[\hat b=\frac{1380}{1000}=1.38,\qquad \hat a=141-1.38\x 55=65.1\]
于是求得最接近这五点的直线为
\[y=65.1+1.38x\]

现在,进一步来考查一下,由上述方程算得的$x=35$, 45, 55, 65, 75时$y$值是多少呢?它们与实际的观测值$y_i$的差又是多少呢?为了以示区别,由方程算出的$65.1+1.38x_i$记为$y_i$有

\begin{center}
\begin{tabular}{cccc}
\hline
实测值&方程算得的值& 偏差& 偏差平方\\
$y_i$&$\hat y_i$&$y_i-\hat y_i$&$\left(y_i-\hat y_i\right)^2$\\
\hline
114&113.4&0.6&0.36\\
124&127.2&$-3.2$&10.24\\
143&141.0&2.0&4.00\\
158&154.8&3.2&10.24\\
166&168.6&$-2.6$&6.76\\
\hline
$\Sigma$&&0&31.60\\
\hline
\end{tabular}
\end{center}

从上表可以看出
\[\sum^n_{i=1}\left(y_i-\hat y_i\right)=0\]
这不是偶然的,可以证明,在一般情况下,总有$\sum\limits^n_{i=1}\left(y_i-\hat y_i\right)=0$. 事实上,
\[\begin{split}
\sum^n_{i=1}\left(y_i-\hat y_i\right)&=\sum^n_{i=1}y_i-\sum^n_{i=1}\left(\hat a+\hat b x_i\right)    \\
&=n\bar y -n\hat a-n\hat b \bar x=n\left(\hat y-\hat a-\hat b \bar x\right)
\end{split} \]
由于已知$\hat a=\bar y-\hat b\bar x$,所以有$\sum\limits^n_{i=1}\left(y_i-\hat y_i\right)=0$


因此,这一结果常常可以作为验算所计算的结果是否正确,但是,在实际使用时,在对大量数据进行计算时,由于
舍入误差的累积会使$\sum\limits^n_{i=1}\left(y_i-\hat y_i\right)\ne 0$,不过这个$\sum\limits^n_{i=1}\left(y_i-\hat y_i\right)$不会大,而在$\sum\limits^n_{i=1}\left(y_i-\hat y_i\right)$的绝对值很大时,这往往是由于计算上的错误所造成的.











\end{document}
