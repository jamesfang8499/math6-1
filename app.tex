\chapter{附录:统计学简介}

\section*{平均值(均值)}

给了一组数据$x_1,x_2,\ldots,x_n$, 相应地有一个平均数
(也称为均值), 它是如下的算术平均数
\[\frac{1}{n} \sum_{i=1}^nx_i\]
通常用$\bar x$来表示它,平均数是日 常生活中常见的量。例如, 工厂中某一产品的平均日产量,农村中某一作物的平均亩产量,商品的平均销售额,学生考试的平均分数等等。自然会产生这样的问题,为什么要用平均数呢?原因是平均数具有代表性,下面来解释它的意思。

对给定的一组数据$x_1,x_2,\ldots,x_n$, 如果用一个数$C$来代表这些$x_1,x_2,\ldots,x_n$, 则$x_i-C$就表达了$x_i$与$C$的偏差, $x_i-C$的绝对值小,$C$就接近$x_i$。即用$C$来代表$x_i$就代表得好. 由于$C$要代表所有数$x_1,x_2,\ldots,x_n$,因此必须考虑几个偏差$x_{1}-C,x_{2}-C,\ldots,x_{n}-C$. 为了消除正负号的影响, 用$(x_i-C)^2$来衡量$x_i$与$C$ 的差别,于是总的差异就是$\sum\limits_{i= 1}^{n}\left ( x_{i}- C\right ) ^{2}$, 当$x_{1},x_{2},\ldots,x_{n}$确定不变时,应选取$C$使$\sum\limits_{i= 1}^{n}\left ( x_{i}- C\right ) ^{2}$达到最小值,这样的$C$才是代表$x_{1},x_{2},\ldots,x_{n}$这$n$个数据最合适的数。

下面来证明这个数就是$\bar x$
\[\begin{split}
    \sum_{i=1}^{n} ( x_{i}-C)^{2} &=\sum_{i=1}^{n} ( x_{i}-\bar{x}+\bar{x}-{C} )^{2} \\
    &=\sum_{i=1}^{n}\left[\left( x_{i}-\bar{x}\right) ^{2}+2\left( x_{i}-\bar{x}\right) ( \bar{x}-C)+(\bar{x}-C)^{2}\right]\\
    &=\sum_{i=1}^{n}\left( x_{i}-\bar{x}\right)^{2}+2\left[\sum_{i=1}^{n}\left( x_{i}-\bar{x}\right)\right](\bar{x}-C)+n(\bar{x}-C)^{2}
\end{split}\]
由于
$$\sum_{i=1}^{n} ( x_{i}-\bar{x} ) =\sum_{i=1}^{*} x_{i}-n\bar{x}=n\bar{x}-n\bar{x}=0$$
因此
$$\sum_{i=1}^{n}(x_{i}-C)^{2}=\sum_{i=1}^{n}(x_{i}-\bar{x})^{2}+n(\bar{x}-C)^{2}\ge \sum_{i=1}^{n}(x_{i}-\bar{x})^{2}$$
显然,等号成立当且仅当$\bar x=C$.

这就证明了:如果以$\sum\limits_{i=1}^n(x_i-C)^2$来衡量$C$代表$x_1,
x_{2},\ldots,x_{n}$的好坏标准,则$\bar x$是代表性最好的值.

现在来看几个特例:

\begin{example}
    如果$x_1=x_2=\cdots=x_n=a$, 则 $\bar{x}=a$.
\end{example}

\begin{proof}
    $$\bar{x}= \frac{1}{n}\sum_{i=1}^{n}x_{i}= \frac{1}{n}\sum_{i= 1}^{n}a= a$$
\end{proof}

\begin{example}
    如果$x_i=a+id$, $i=0,1,2,\ldots,n-1$,
即$x_0,x_1,\ldots,x_{n-1}$是等差数列,这时
\[\begin{split}
   \bar{x}= \frac {1}{n}\sum _{i= 0}^{n-1}x_{i}&= \frac{1}{n}\sum_{i= 0}^{n- 1}( a+ id) \\
   &= \frac{1}{n}\left [ na+ \frac {(  n- 1 )n}{2}d \right ]=  a+\frac{(n-1)d}{2}
\end{split}\]
\end{example}
例如求$1,2,\ldots, 100$的平均 数,这 时$a=d=1$, $n=100$于是$\bar{x} = 1+ \frac {99}{2}= 50.5$; 又 如求$3, 6, 9, 12, \ldots, 27 $这几 个数的均值,则$a= d= 3$, $n= 0$, 于是$\bar{x}=3+\frac{8}{2}\times 3=15$.


\section*{均方差(方差)}

从上面的讨论可以看出,平均数$\bar x$代表$x_1,\ldots,x_n$的好坏程度是由$\sum\limits^n_{i=1}(x_i-\bar x)^2$来反映的,它是$n$个数据$x_i$对
$\bar x$的偏差平方的总和. 如果用$n$除一下,就得到平均的偏差平方和,它反映了每个数据$x_i$偏离$x$的“平均”状况,因此称它为均方差,简称为方差,通常用$S^2$来表示,即有
\[S^2=\frac{1}{n}\sum^n_{i=1}(x_i-\bar x)^2\]
方差表明了一组数据$x_1,x_2,\ldots,x_n$的分散程度. $S^2$越小,表示$x_i$之间的差异小,$\bar x$的代表性就好;$S^2$越大,表示$x_i$之间的差异也大,数据便很分散,此时$\bar x$代表性就不强.

有时,我们要比较的就是两组数据的分散程度. 这用方差就很方便. 例如有两班学生,甲班有$n$个学生,乙班有$m$个学生,进行教学测验后,甲班的成绩为$x_1,x_2,\ldots,x_n$,乙班的成绩为$y_1,y_2,\ldots,y_m$,平均数$\bar x$与$\bar y$反映了两个班的平均成绩,各自方差$S^2_x$和$S^2_y$反映了两班成绩不齐的程度,方差小的班,学生之间的差异就小.

如果$x_1=x_2=\cdots=x_n=a$,则$\bar x=a$,这时方差$S^2=0$. 如果$x_i=i,\; i=1,2,\ldots,n$,则$\bar x=\frac{n+1}{2}$,而方差
\[\begin{split}
    S^2=\frac{1}{n}\sum^n_{i=1}(x-\bar x)^2 &=\frac{1}{n}\sum^n_{i=1}\left(i-\frac{n+1}{2}\right)^2\\
&=\frac{1}{n}\sum^n_{i=1}i^2-\left(\frac{n+1}{2}\right)^2\\
&=\frac{n(n+1)(2n+1)}{6}-\frac{(n+1)^2}{4}\\
&=\frac{(n+1)(4n+2-3n-3)}{12}=\frac{n^2-1}{12}
\end{split}\]

平均数$\bar x$与平均方差$S^2$是一组数据$x_1,x_2,\ldots,x_{n}$的两个重要指标,给了一组数$x_1,x_2,\ldots,x_{n}$后,如何具体地算出这两个值,我们在下一小节中来详细讨论这一问题.

\section*{计算公式}
设$x_1,x_2,\ldots,x_{n}$相应的均值和方差为$\bar x$和$S^2_x$,设$y_1,y_2,\ldots,y_{n}$相应的均值和方差为$\bar y$和$S^2_y$,下面先导出几个公式,然后利用它们来具体计算。

\begin{enumerate}
    \item 如果$y_i=x_i+a,\; i=1,2,\cdots,n$。则 $\bar {y}= \bar{x} + a$.
    
\begin{proof}
     $\bar{y} = \frac 1n\sum\limits_{i= 1}^{n}y_{i}= \frac 1n\sum\limits _{i= 1}^{n} ({x_{i}+ a}) = \frac 1n\sum\limits _{i= 1}^{n}x_{i}+ a= x+ a$
\end{proof}
    
\item 如果$y_i=bx_i,i=1,2,...n$, 则$\bar{y}=b\bar{x}$

\begin{proof}
$\bar y= {\frac 1n}\sum\limits _{i= 1}^{n}y_{i}= {\frac 1n}\sum\limits _{i= 1}^{n}bx_{i}= b {\bar {x}}$
\end{proof}

    \item 如果$y_i= bx_{i}+ a,\; i= 1, 2, \ldots, n$, 则$\bar {y}=b\bar{x} + a$
  
\begin{proof}
    由1,2立即可得.
\end{proof}  
    \item $\sum\limits _{i= 1}^{n}\left (  x_{i}- \bar {x} \right ) ^{2}= \sum\limits _{i= 1}^{n}x_{i}^{2}- n\bar {x}^{2}$

\item 如果$y_i=bx_i+a,\; i=1,2,\ldots,n$, 则$S_y^{2}=b^{2}S_{x}^{2}$

\begin{proof}
\[\begin{split}
    S_{y}^{2}= {\frac 1n}\sum\limits  _{i= 1}^{n}(y_{i}- \bar{y})^2&=\frac{1}{n}\sum\limits ^n_{i=1}\left ( bx_{i}+ a- b\bar{y} - a\right ) ^{2}\\
    &=\frac{b^{2}}{n}\sum\limits _{i=1}^{n}(x_{i}-\bar{x})^{2}=b^{2}S_{x}^{2}
\end{split}\]
\end{proof}
\end{enumerate}

公式5告诉我们,当$b=\pm1$时,$S_x^2=S_y^2$. 即一组数据同
加一个常数,或同时改变符号时,它的方差是不变的.

\begin{example}
设某地历年夏季的雨量为:(单位mm)
\[248.7\quad 249.4\quad 133.2\quad 153.5\quad 211.7 \]
求它的平均值$\bar x$及方差$S^2$
\end{example}

取$y_i=x_i-200$,所以$x_i=y_i+200$,于是$\bar x=\bar y+200$,$S^2_x=S^2_y$

列表演算如下:
\begin{center}
\begin{tabular}{ccc}
\hline 
$x_i$ & $y_i=x_i-200$ & $y^2_i$\\
\hline
248.7  & 48.7  & 2371.69\\
249.4& 49.4&2440.36\\
133.2&$-66.8$&4462.24\\
153.5&$-46.5$&2162.25\\
211.7&11.7&136.89\\
\hline
$\Sigma$ &$-3.5$& $11573.43$\\
\hline
\end{tabular}
\end{center}

所以
\[\begin{split}
    \bar x&=\bar y+200=-\frac{3.5}{5}+200=199.3\\
S^2_x&=S^2_y=\frac{1}{n}\left(\sum^n_{i=1} y^2_i-n\bar y^2\right)=\frac{1}{5}(11573.43-245)=2314.196
\end{split}
    \]

从演算过程可以看出,$y_i$的平方、求和的计算量比$x_i$的平方、求和的计算量小.这里选200是为了使$x_i-200$数字变小而且减法本身也较易.

\begin{example}
    设$i=a+id,\; i=1,2,\ldots,n$,即$x_1,x_2,\ldots,x_n$为等差数列,试求方差$S^2_n$.
\end{example}

为了便于比较,下面用两种方法求解.

\textbf{解一:} 已知$\bar x=a+\frac{n+1}{2}d$,所以$x_i-x=\left(i-\frac{n+1}{2}\right)d$.
\[\sum^n_{i=1}(x_i-\bar x)^2 =\sum^n_{i=1}\left(i-\frac{n+1}{2}\right)^2 d^2=\frac{d^2n(n^2-1)}{12} \]
所以
\[S^2_x=\frac{d^2(n^2-1)}{12} \]

\textbf{解二:} 取$y_i=\frac{x_i-a}{d}=i$, $i=1,2,\ldots,n$,于是$S^2_y=\frac{1}{d^2}S^2_x$,即$S^2_x=d^2S^2_y$. 前面已算过,这时$S^2_y=\frac{n^2-1}{12}$
所以$S^2_x=\frac{d^2(n^2-1)}{12}$.

\begin{example}
    对两组数据$x_1,x_2,\ldots,x_n$及$y_1,y_2,\ldots,y_m$, 设它们的均值及均方差分别为$\bar x,\; S^2_x; \; \bar y,\;  S^2_y$,将这两组数据合并为一组数据,求合并后的数据的均值$\bar z$及均方差$S^2_z$.
\end{example}

\begin{solution}
令$\bar z=\frac{1}{n+m}\left(\sum\limits^n_{i=1}x_i+\sum\limits^m_{i=1}y_i\right)=\frac{1}{n+m}(n\bar x+m\bar y)$

又
\[\begin{split}
    S^2_x&=\frac{1}{n+m}\left[\sum\limits^n_{i=1}(x_i-\bar z)+\sum\limits^m_{i=1}(y_i-\bar z)^2\right]\\
    &=\frac{1}{n+m}\left[\sum\limits^n_{i=1}(x_i-\bar x)^2 +n(\bar x-\bar z)^2+\sum\limits^m_{i=1}(y_i-\bar y)^2+m(\bar y-\bar z)^2\right]\\
    &=\frac{1}{n+m}\left[nS^2_x+n(\bar x-\bar z)^2+mS^2_y+m(\bar y-\bar z)^2\right]
\end{split}\]
但是$\bar z=\frac{n}{n+m}\bar x+\frac{m}{n+m}\bar y$,代入有
\[\bar x-\bar z=\frac{m}{n+m}(\bar x-\bar y),\qquad \bar y-\bar z=\frac{n}{n+m}(\bar y-\bar x)\]
\[S_{y}^{2}= \frac n{n+ m} S_{x}^{2}+ \frac m{n+ m} S_{y}^{2}+ \frac n{n+ m}\left(\frac m{n+ m}\right)^{2}(\bar x- \bar{y} ) ^{2}+\frac{m}{n+m}\left(\frac{n}{m+n}\right)^{2}(\bar{y}-\bar x)^{2}\]
所以
\[S_{y}^{2}=\frac{1}{n+m}\left(nS_{x}^{2}+mS_{y}^{2}\right)+\frac{nm}{(n+m)^{2}}(\bar{x}-\bar{y})^2\]
总之有
$$z=\frac{1}{n+m}(n\bar{x}+m\bar{y})$$
\[S_{z}^{2}=\frac{1}{n+m}\left(nS_{x}^{2}+mS_{z}^{2}\right)+\frac{nm}{(n+m)^{2}}(\bar{x}-\bar{y})^2\]
\end{solution}

特别,当$m=1$时,$S^2_y=0$,记$y_1=x_{n+1}$,这就得到增加一个新数据的递推公式
\[\begin{split}
    \bar z&= \frac{1}{n+1}(n\bar x+x_{n+1})\\
    S^2_z&=\frac{n}{n+1}S^2_x+\frac{n}{(n+1)^2}(\bar x-x_{n+1})^2
\end{split}\]

不难看出,如果从$x_1,x_2,\ldots,x_n$中减少一个数据,例如删去$x_n$,那么$x_1,x_2,\ldots,x_{n-1}$的均值$\bar x_{*}$与均方差$S^2_{*}$可以用$\bar x$与$S^2_x$及$x_n$表示,即有递推公式
\[\begin{split}
    \bar x_{*}&= \frac{1}{n-1}(n\bar x-x_{n})\\
    S^2_{*}&=\frac{n}{n-1}S^2_x-\frac{n}{(n-1)^2}(\bar x-x_{n})^2
\end{split}\]

\section*{最小二乘法}
在实际工作中经常需要从实测的数据求出变量之间的关系式. 例如年龄和血压是有关系的,随着年龄的增长,血压是会增高的,调查了几百人的年龄与血压的情况,将资料按年龄分组. 用各组的均值(年龄的均值和血压的均值),得到如下的数据:

\begin{minipage}{.35\textwidth}
\begin{center}
    \begin{tabular}{cc}
\hline
        年龄$x_i$ & 心脏收缩压$y_i$\\
\hline
35&114\\
45&124\\
55&143\\
65&158\\
75&166\\
\hline
    \end{tabular}
\end{center}
\end{minipage}\hfill
\begin{minipage}{.6\textwidth}
\centering
\begin{tikzpicture}[>=stealth, scale =.8]
\draw[->](-1,0)--(6,0)node[above]{$x$(年龄)};
\draw[->](0,-1)--(0,8)node[right]{$y$(血压)};
\foreach \x/\y in {1/35,2/45,3/55,4/65,5/75}
{
    \draw(\x,0)node[below]{$\y$}--(\x,.1);
}
\foreach \x/\y in {1/110,2/120,3/130,4/140,5/150,6/160,7/170}
{
    \draw(0,\x)node[left]{$\y$}--(.1,\x);
}
\tkzDefPoints{1/1.4/A, 2/2.4/B, 3/4.3/C, 4/5.8/D, 5/6.6/E}
\tkzDrawPoints(A,B,C,D,E)
\node at (A) [left]{$P_1$};
\node at (B) [right]{$P_2$};
\node at (C) [left]{$P_3$};
\node at (D) [left]{$P_4$};
\node at (E) [right]{$P_5$};
\draw[domain=.5:5.5, smooth, thick]plot(\x, 1.35*\x+.1);
\node[below left]{$O$};
\end{tikzpicture}

\end{minipage}

从表上的数据或图上的点来看,
血压值$y$与年龄数$x$似乎有直线的关系,但是这五个点又不正好在一条直线上,于是发生了一个问题:如何求一个直线方程,使得这条线与五个点最接近.

用$P_i$表示点$(x_i,y_i)\; i=1,2,3,4,5$.对给定的直线$\ell:\; y=a+bx$. 在直线$\ell$上取五个点,使其横坐标与$P_i$的横坐标$x_i$相同,这五个点应是
\[Q_i(x_i, a+bx_i) \qquad i=1,2,3,4,5\]
很明显,这五个点的纵坐标之差为$y_i-a-bx_i$. 用数值$\sum\limits^5_{i=1}(y_i-a-bx_i)^2$
来衡量直线$\ell$上五个点$Q_1,\ldots,Q_5$和已给的五个点$P_1,\ldots,P_5$的总差距,这个差距愈小,就认为这条直线愈接近这些点. 因此,最接近的直线应使这个差距达到最小值,下
面用和以前类似的方法来求出$a$和$b$.

已知$n$个点有坐标$(x_i,y_i)\; i=1,2,\ldots,n$.求直线$\ell:\; y=a+bx$,使和这$n$个点在上述意义下最接近,也就是求数值$a$和$b$,使
\[Q=\sum^n_{i=1}(y_i-a-bx_i)^2\]
达到最小值.

\begin{blk}{定理(最小二乘法)}
使$Q$达最小值的$a,b$为
\[\hat a=\bar y-\hat b \bar x,\qquad \hat b=\frac{\sum\limits^n_{i=1}(x_i-\bar x)(y_i-\bar y)}{\sum\limits^n_{i=1}(x_i-\bar x)^2}\]
\end{blk}

\begin{proof}
对任意数$a,b,\hat a,\hat b$,有
\[\begin{split}
Q&=\sum^n_{i=1}(y_1-a-bx_i)^2 =\sum^n_{i=1}\left[y_i-\hat a-\hat b x_i+\hat a-a+(\hat b-b)x_i\right]^2\\ 
&=\sum^n_{i=1}\left(y_i-\hat a-\hat b x_i\right)^2+2\sum^n_{i=1}\left(y_i-\hat a-\hat b x_i\right)\cdot \left(\hat a-\hat a+bx_i-bx_i\right)\\
&\qquad +\sum^n_{i=1}\left[\hat a-a+(\hat b-b)x_i\right]^2
\end{split}\]
取$\hat a,\hat b$使得
\[\sum^n_{i=1}\left(y_i-\hat a-\hat b x_i\right)=0,\qquad \sum^n_{i=1}x_i\left(y_i-\hat a-\hat b x_i\right)=0\]
那么上面
\[\begin{split}
    Q&=\sum^n_{i=1}\left(y_i-\hat a-\hat b x_i\right)^2+\sum^n_{i=1}\left[\hat a-a+(\hat b-b)x_i\right]^2\ge \sum^n_{i=1}\left(y_i-\hat a-\hat b x_i\right)^2
\end{split}\]
因此若能由$\hat a$,$\hat b$之条件唯一
引出$\hat a$,$\hat b$,那么它就是所求的使$Q$达极小值的$a$,$b$了。

将条件改变为
\[\hat a+\hat b \bar x=\bar y,\quad n\hat a \bar x+\hat b\sum^n_{i=1}x^2=\sum^n_{i=1}x_iy_i\]
由$\hat a=\bar y-\hat b\bar x$,代入后式,有
\[n\left(\bar y-\hat b\bar x\right)\bar x+\hat b\sum^n_{i=1}x^2_i=\sum^n_{i=1}x_iy_i\]
即有
\[\hat b\left[\sum^n_{i=1}(x_i-\bar x)^2\right]=\sum^n_{i=1}x_iy_i-n\bar x\bar y=\sum^n_{i=1}(x_i-\bar x)\cdot (y_i-\bar y)\]
所以\[\hat b=\frac{\sum\limits^n_{i=1}(x_i-\bar x)(y_i-\bar y)}{\sum\limits^n_{i=1}(x_i-\bar x)^2}\]
于是$\hat a=\bar y-\hat b\bar x$, 这证明了定理.
\end{proof}

回到原来的例子,下面列表进行计算.
\begin{center}
\begin{tabular}{cccccc}
\hline
$x_i$ & $y_i$&$x_i-\bar x$&$y_i-\bar y$&$(x_i-\bar x)^2$&$(x_i-\bar x)(y_i-\bar y)$\\
\hline
35&114&$-20$&$-27$&400&540\\
45&124&$-10$&$-17$&100&170\\
55&143&0&2&0&0\\
65&158&10&17&100&170\\
75&166&20&25&400&500\\
\hline
$\Sigma 275$&705&&&1000&1380\\
平均55&141\\
\hline
\end{tabular}
\end{center}

\[\hat b=\frac{1380}{1000}=1.38,\qquad \hat a=141-1.38\x 55=65.1\]
于是求得最接近这五点的直线为
\[y=65.1+1.38x\]

现在,进一步来考查一下,由上述方程算得的$x=35$, 45, 55, 65, 75时$y$值是多少呢?它们与实际的观测值$y_i$的差又是多少呢?为了以示区别,由方程算出的$65.1+1.38x_i$记为$y_i$有

\begin{center}
\begin{tabular}{cccc}
\hline
实测值&方程算得的值& 偏差& 偏差平方\\
$y_i$&$\hat y_i$&$y_i-\hat y_i$&$\left(y_i-\hat y_i\right)^2$\\
\hline
114&113.4&0.6&0.36\\
124&127.2&$-3.2$&10.24\\
143&141.0&2.0&4.00\\
158&154.8&3.2&10.24\\
166&168.6&$-2.6$&6.76\\
\hline
$\Sigma$&&0&31.60\\
\hline
\end{tabular}
\end{center}

从上表可以看出
\[\sum^n_{i=1}\left(y_i-\hat y_i\right)=0\]
这不是偶然的,可以证明,在一般情况下,总有$\sum\limits^n_{i=1}\left(y_i-\hat y_i\right)=0$. 事实上,
\[\begin{split}
\sum^n_{i=1}\left(y_i-\hat y_i\right)&=\sum^n_{i=1}y_i-\sum^n_{i=1}\left(\hat a+\hat b x_i\right)    \\
&=n\bar y -n\hat a-n\hat b \bar x=n\left(\hat y-\hat a-\hat b \bar x\right)
\end{split} \]
由于已知$\hat a=\bar y-\hat b\bar x$,所以有$\sum\limits^n_{i=1}\left(y_i-\hat y_i\right)=0$


因此,这一结果常常可以作为验算所计算的结果是否正确,但是,在实际使用时,在对大量数据进行计算时,由于
舍入误差的累积会使$\sum\limits^n_{i=1}\left(y_i-\hat y_i\right)\ne 0$,不过这个$\sum\limits^n_{i=1}\left(y_i-\hat y_i\right)$不会大,而在$\sum\limits^n_{i=1}\left(y_i-\hat y_i\right)$的绝对值很大时,这往往是由于计算上的错误所造成的.

